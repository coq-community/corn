% $Id: hrefl-iii.tex,v 1.13 2004/09/23 13:29:11 freek Exp $
\documentclass[numreferences]{kluwer}
\usepackage{amssymb,url}
\usepackage{enumerate}

\newdisplay{definition}{Definition}[section]
\newdisplay{theorem}[definition]{Theorem}
\newdisplay{lemma}[definition]{Lemma}
\newdisplay{corollary}[definition]{Corollary}
\newdisplay{proposition}[definition]{Proposition}
\newproof{example}{Example}
\newproof{notation}{Notation}

\newcommand{\intII}{\,]\![}
\newcommand{\intrel}{\mathbin{\intII_{\rho}}}
\newcommand{\N}{\ensuremath{\mathcal{N}}}
\newcommand{\NF}{\ensuremath{{\cal N}_{F}}}
\newcommand{\Prop}{{\textsf{Prop}}}
\newcommand{\alt}{\mathrel{|}}
\newcommand{\Z}{{\mathbb Z}}
\newcommand{\V}{{\mathbb V}}
\newcommand{\tacticname}[1]{\textsf{#1}}
\newcommand{\rational}{\tacticname{rational}}

\newcommand{\zring}{\ensuremath{\mathrm{zring}}}
\newcommand{\nexp}{\ensuremath{\mathrm{nexp}}}
\newcommand{\less}{\mathrel{\prec_A}}
\newcommand{\nat}{{\mathbb N}}
\newcommand{\axiom}[1]{\ensuremath{\mathbf{#1}}}
\newcommand{\mlfnv}[2]{\ensuremath\ulcorner\!\!\ulcorner{#1}\urcorner\!\!\urcorner_{#2}}
\newcommand{\mlfn}[2]{\ensuremath\ulcorner{#1}\urcorner_{#2}}
\newcommand{\domain}{\ensuremath{\mathrm{dom}}}
\newcommand{\ord}{\ensuremath{\mathrm{ord}}}
\newcommand{\wf}{\ensuremath{\mathrm{wf}}}
\newcommand{\nf}{\ensuremath{\mathrm{nf}}}
\newcommand{\intrels}{\mathbin{\intII_{\sigma}}}
\newcommand{\intrelt}{\mathbin{\intII_{\theta}}}

\newcommand{\renamevar}[2]{\ensuremath{{#1}^{#2}}}
\newcommand{\isrenamevar}[3]{\ensuremath{{#1}=\renamevar{#2}{#3}}}
\newcommand{\idn}{()}
\newcommand{\coeff}[2]{\ensuremath{\|#2\|_{#1}}}
\newcommand{\iter}[2]{\ensuremath{{#1}\cdot{#2}}}

\newcommand{\multMZ}{\ensuremath{\cdot_{\mathrm M\Z}}}
\newcommand{\multMV}{\ensuremath{\cdot_{\mathrm M\V}}}
\newcommand{\multMM}{\ensuremath{\cdot_{\mathrm{MM}}}}
\newcommand{\plusMM}{\ensuremath{+_{\mathrm{MM}}}}
\newcommand{\plusPM}{\ensuremath{+_{\mathrm{PM}}}}
\newcommand{\plusPP}{\ensuremath{+_{\mathrm{PP}}}}
\newcommand{\multPM}{\ensuremath{\cdot_{\mathrm{PM}}}}
\newcommand{\multPP}{\ensuremath{\cdot_{\mathrm{PP}}}}
\newcommand{\starPP}{\ensuremath{\star_{\mathrm{PP}}}}
\newcommand{\plusFF}{\ensuremath{+_{\mathrm{FF}}}}
\newcommand{\multFF}{\ensuremath{\cdot_{\mathrm{FF}}}}
\newcommand{\divFF}{\ensuremath{/_{\mathrm{FF}}}}
\newcommand{\starFF}{\ensuremath{\star_{\mathrm{FF}}}}

\begin{document}
\begin{article}
\begin{opening}
\title{A Decision Procedure for Equational Reasoning\\
in Commutative Algebraic Structures}
\author{L. \surname{Cruz-Filipe}\thanks{Work done during a stay at the
Radboud University Nijmegen}}
\institute{CLC, Lisbon, Portugal}
\author{F. \surname{Wiedijk}}
\institute{NIII, Radboud University Nijmegen, Netherlands}
\runningauthor{L. Cruz-Filipe and F. Wiedijk}
\runningtitle{A Decision Procedure for Equational Reasoning}

\begin{abstract}
We present a decision procedure for equational reasoning in abelian groups,
commutative rings and fields that checks whether a given equality can
be proven from the axioms of these structures.
This has been implemented as a tactic in Coq; here we give a mathematical
description of the decision procedure that abstracts from Coq specifics,
making the work in principle adaptable to other theorem provers.

Within Coq we prove that this decision procedure is correct.
On the meta-level we analyse its completeness, showing that it is complete
for groups and rings in the sense that the tactic succeeds in finding a
proof of an equality if and only if that equality is provable from
the group/ring axioms without any hypotheses.
Finally we characterize in what way our method is incomplete for
fields.
\end{abstract}
\keywords{Decision procedures, theorem proving, equational reasoning,
abelian groups, commutative rings, fields}
\end{opening}

\section{Introduction}

One of the main aims of the Foundations group at the Radboud
University of Nijmegen is to help making formalization of mathematics
practical and attractive.
For this reason a library of formal mathematics for the Coq
system~\cite{coqmanual} -- called the C-CoRN
library~\cite{lcf:geu:wie:04} -- has been developed to exercise the
technology of proof formalization.
This library started as a formalization of the Fundamental Theorem of
Algebra in the so-called FTA project, but then was extended with a
formalization of basic analysis up to the Fundamental Theorem of
Calculus, and currently other subjects are being added to it as well.

To support the formalization work for the C-CoRN library,
a tactic called {\rational} was implemented.
It automatically proves equations from the field axioms.
Later this tactic was generalized to prove equations in
rings and groups as well.
The tactic uses the approach of \emph{reflection} from~\cite{ACHA90},
in particular the variant of reflection called \emph{partial reflection}
described in~\cite{geu:wie:zwa:00}.
The generalization to rings and groups uses the application
of partial reflection called \emph{hierarchical reflection} in~\cite{lcf:wie:04}.
This paper studies the behavior of {\rational} from a theoretical
point of view.

Most proof assistants have automation tools that provide the
functionality of {\rational} for rings, and many also have them for
fields (for instance, Coq provides both of these
with \tacticname{ring}~\cite{coqmanual} and
\tacticname{field}~\cite{del:may:01}).  This automation is always
implemented (like {\rational}) by putting polynomials into a normal
form.
%
However, there are three different ways that this decision procedure can
be realized in the proof assistant:
\begin{enumerate}
\item
First of all the decision procedure might just take the equation,
normalize both sides, and then give a yes/no answer depending on whether
the normal forms are the same.
In this approach there is no \emph{reduction} of this judgment
to a proof on a lower level.
(It does not follow the ``de Bruijn criterion'', the approach that
each proof needs to be reduced to elementary steps that are checked
by a small ``kernel'' of the program.)

As an example, this approach is taken
by the Mizar proof assistant \cite{muz:93}.
If one puts the ``requirement'' \tacticname{ARITHM} in the environment
of a formalization, this ``ring equality'' decision procedure will be applied
automatically, even without having to mention a tactic.

\item
The second way is to have a decision procedure that generates
a proof of the equation that is checked afterwards.
However, the implementation of the decision procedure itself is
not proved correct:
if there is a bug in the implementation,
the procedure might return ``equal'', but
then the proof happens not to be correct.

This approach is taken by the HOL system \cite{har:00}.
This system supports \emph{ordered rewriting}
(straightforward rewriting with the ring axioms will not work, as
the commutativity rules like $x + y = y + x$ will cause rewriting
not to terminate; ordered rewriting is a generalization of AC-rewriting,
rewriting modulo associativity and commutativity).
Using this feature, rewriting using a suitable form of the ring axioms
will provide a decision procedure like ours.

\item
The third way (the one we do it) is to have the decision procedure
proved correct inside the system.
Then it is not necessary to check the proof for specific instances,
it is sufficient to run the procedure
and see that it returns the correct result.

This approach, called ``the two-level approach'' by Barendregt and others
in~\cite{bar:bar:ruy:96},
is also used by the versions of this decision
procedure (the tactics \tacticname{ring} and \tacticname{field})
implemented for Coq.
\end{enumerate}

The main difference between our work and other implementations
of the same idea is that the normalization is very structured
and systematic.
We define addition and multiplication functions
that are meant to operate on monomials and polynomials that are already
in normal form.
These functions are then the ``building blocks'' of our normalization
function.
This enables us to easily prove the correctness of the normalization
function, which we need to use the reflection method.

\bigskip

The mathematics in this paper has not been formalized.
Formalizing takes an order of a magnitude more work than just doing
the proofs in the informal -- old style -- way, and it is not clear
what the benefit of formalization would be in this case.
%(apart from the fact that one then would be \emph{certain} that the
%proofs are correct).
The exception regards the proofs that are essential to the
tactic.  Proofs that have been checked within Coq are
always explicitly marked.

On the other hand, the description in this paper of the {\rational}
tactic is kept as independent of Coq as possible.  The algorithms and
results that we describe are not specific to Coq or even to type
theory, they can be used with any proof assistant.
In particular, it is our opinion that {\rational} could be adapted
to the systems mentioned above; or (alternatively) that the behavior
of the tactics those systems use could be characterized by a similar
method.

This paper is self-contained in the sense that everything that
is used is defined as well.
However, it does not go into detail about partial/hierarchical reflection
or the details of the {\rational} tactic.
For this we refer to two earlier papers,~\cite{geu:wie:zwa:00}
and~\cite{lcf:wie:04}.

We begin by describing the mechanism of {\rational} in more detail.
Then we discuss the several layers of expressions we need to study
it.

Section~\ref{quoteing} formally describes the ML part of the tactic and
proves a number of results about it, among which the correctness of the
code.
In Section~\ref{normalization} we introduce the normalization function for
rings and prove its completeness.  This proof generalizes almost directly
to groups, as explained in Section~\ref{groups}.  Finally,
Section~\ref{fields} analyzes the more complex case of fields, focusing
on why the previous proof cannot be adapted to this situation, and presents
an alternative completeness result.

The tactic described here is a simplified version of that
in~\cite{lcf:wie:04}, and in Section~\ref{extensions} we explain how the
same theorems can be generalized to the implemented tactic.  We conclude
with an overview of what was achieved in Section~\ref{concl}.

\section{Background}

In this section we lay the bricks for our work.  We begin
by describing the way {\rational} works in detail, after which we
summarize the parts of~\cite{geu:pol:wie:zwa:02}, \cite{geu:wie:zwa:00}
and~\cite{lcf:wie:04} that are essential for the remainder of the paper.

\subsection{The mechanism of {\rational}}\label{tactic}

The {\rational} tactic proves equalities in an algebraic
structure $A$ through the use of a type of \emph{syntactic expressions} $E$
together with an \emph{interpretation relation}.
$$\intrel \subseteq E \times A$$
In this, $\rho$ is a \emph{valuation} that maps the variables in the
syntactic expressions to values in $A$.
The relation $e \intrel a$ means that the syntactic expression
$e$ is interpreted under the valuation $\rho$ by the object $a$.

The type $E$ is inductive, and therefore it is possible to define
a \emph{normalization function} $\N : E \to E$ recursively.
One then proves
$$
\begin{array}{c}
e \intrel a \;\Rightarrow\; \N(e) \intrel a \\
\noalign{\medskip}
e \intrel a \;\land\; e \intrel b \;\Rightarrow\; a =_A b
\end{array}
$$
and together these give a method to prove equalities between terms
that denote elements of $A$.

To prove $a =_A b$, one finds $e$, $f$ and $\rho$ with $e \intrel a$ and
$f \intrel b$,
and one checks whether $\N(e) = \N(f)$.
If this is the case then it follows that $a =_A b$:
from the first lemma we find that $\N(e) \intrel a$ and $\N(f) \intrel b$,
and then the second lemma gives this desired equality.

The tactic has two parts: the first part is an ML program that
finds the expressions $e$ and $f$ and the valuation $\rho$ and
constructs a proof term for the equation $a=_A b$; the second part is a
Coq formalization of normalization of polynomial expressions over a
field.
This means that the tactic contains two -- quite different --
programs: the program that calculates a proof term
from an equation, which is written in ML, and the program that computes the
normal form of a polynomial expression, which
is written in the Coq type theory.  Only the last one is proved correct
as part of the formalization.

The correctness of {\rational} is guaranteed by the way
it works: if it finds a proof of an equation, then that proof has
automatically been checked by Coq and is correct.
Failure, however, can arise from two different situations:
\begin{enumerate}[(1)]
\item the ML program finds $e$, $f$ and $\rho$ but $e\intrel a$ or
$f\intrel b$ does not hold;
\item $\N(e)$ and $\N(f)$ do not coincide.
\end{enumerate}
In this paper we formalize the ML program as a function $\mlfn{\cdot}{}$
and prove that situation~(1) cannot occur (Theorem~\ref{quotecorrect}).

We then characterize under what conditions the tactic is complete.
Now completeness can mean two things here:
either one can consider the set of equations that hold in all fields,
or one can consider the equations that can be proved from the field axioms.
It happens to be the case that both sets of equations are the
same~\cite{cha:key:90}.
In this paper we establish completeness for groups and rings,
meaning that in these situation~(2) means that $a$ and $b$ are not
provably equal.  Unfortunately this result extends only
partially to fields, but we can still give a simple condition that,
if fulfilled, yields the same conclusion.

As a consequence, when a call to {\rational} fails no proof of the goal
exists that follows exclusively from the structure's axioms.  This is
extremely useful in interactive proof development, since it enables the
user to detect wrong paths much earlier.

\subsection{The semantic level}\label{structures}

We now summarize the Algebraic Hierarchy of C-CoRN~\cite{geu:pol:wie:zwa:02},
on top of which {\rational} works.

\begin{definition}
A \emph{setoid structure} over $A$ is a relation
$=_A:A\to A\to\Prop$ (denoted infix) satisfying:
\begin{eqnarray*}
\axiom{Set_1} & : & \forall_{x:A}.x=_A x \\
\axiom{Set_2} & : & \forall_{x,y:A}.x=_A y\to y=_A x\\
\axiom{Set_3} & : & \forall_{x,y,z:A}.x=_A y\to y=_A z\to x=_A z
\end{eqnarray*}
Furthermore, we distinguish subtypes $[A\to A]$ and $[A\to A\to A]$ of
$A\to A$ and $A\to A\to A$, respectively, satisfying
\[
\begin{array}{l}
\displaystyle
\axiom{Set_4} :
 \forall_{f:[A\to A]}.\forall_{x,x':A}.x=_A x'\to f(x)=_A f(x')\smallskip\\
\axiom{Set_5} : \forall_{f:[A\to A\to A]}.
 \forall_{x,x',y,y'}.x=_A x'\wedge y=_A y'\to f(x,y)=_A f(x',y')
\end{array}\]
We will speak of a setoid $A$ to mean a type $A$ with a setoid structure over
$A$.
\end{definition}

\begin{definition}
A \emph{group structure} over $A$ is a setoid structure over $A$ together
with a tuple $\langle 0_A,+_A,-_A\rangle$ where $0_A:A$, $+_A:[A\to
A\to A]$ and $-_A:[A\to A]$ (we will write $+_A$ using the usual infix
notation) satisfying:
\begin{eqnarray*}
\axiom{SG} & : & \forall_{x,y,z:A}.(x+_A y)+_A z=_A x+_A(y+_A z) \\
\axiom{M_1} & : & \forall_{x:A}.x+_A 0=_A x\\
\axiom{M_2} & : & \forall_{x:A}.0+_A x=_A x\\
\axiom{G_1} & : & \forall_{x:A}.x+_A (-_A x)=_A 0\\
\axiom{G_2} & : & \forall_{x:A}.(-_A x)+_A x=_A 0\\
\axiom{AG} & : & \forall_{x,y:A}.x+_A y=_A y+_A x
\end{eqnarray*}
Notice that axiom \axiom{M_2} (respectively \axiom{G_2}) can be proved
from \axiom{M_1} (resp.\ \axiom{G_1}) and \axiom{AG}.  But in the construction
of the Algebraic Hierarchy \axiom{AG} is introduced last.

By a group $A$ we mean a type $A$ with a group structure over it.
\end{definition}

\begin{definition} Let $A$ be a group.  We define $-_A:A\to A\to A$ by
\[x-_A y:= x+_A(-_A y).\]
\end{definition}

The following is trivial, and allows us to write $-_A:[A\to A\to A]$.

\begin{proposition} $-_A$ satisfies \axiom{Set_5}.
\end{proposition}

\begin{definition}
A \emph{ring structure} over $A$ is a group structure over $A$ together with
a tuple $\langle 1_A,\times_A\rangle$ where $1_A:A$, $\times_A:[A\to
A\to A]$ (we will write $\times_A$ using the usual infix notation)
satisfying the following.
\begin{eqnarray*}
\axiom{R_1} & : & \forall_{x,y,z:A}.
 (x\times_A y)\times_A z=_A x\times_A(y\times_A z) \\
\axiom{R_2} & : & \forall_{x:A}.x\times_A 1=_A x\\
\axiom{R_3} & : & \forall_{x:A}.1\times_A x=_A x\\
\axiom{R_4} & : & \forall_{x,y:A}.x\times_A y=_A y\times_A x\\
\axiom{R_5} & : & \forall_{x,y,z:A}.
 x\times_A(y+_A z)=_A(x\times_A y)+_A(x\times_A z)
\end{eqnarray*}
As before, axiom \axiom{R_3} can be proved from \axiom{R_2} and
\axiom{R_4}.

By a ring $A$ we mean a type $A$ with a ring structure over it.
\end{definition}

\begin{definition} Let $A$ be a ring.  We define two functions
$\zring_A:\Z\to A$ and $\nexp_A:A\to\nat\to A$ inductively as follows:
\begin{eqnarray}
\label{zring:0}
 \zring_A(0) & := & 0_A\\
\label{zring:pos}
 \zring_A(n+1) & := & \zring_A(n)+_A 1_A,\mbox{ for $n\geq 0$}\\
\label{zring:neg}
 \zring_A(n-1) & := & \zring_A(n)-_A 1_A,\mbox{ for $n\leq 0$}
\end{eqnarray}
\begin{eqnarray}
\label{nexp:zero}
 \nexp_A(x,0) & := & 1_A\\
\label{nexp:suc}
 \nexp_A(x,n+1) & := & x\times_A\nexp_A(x,n)
\end{eqnarray}
We denote $\zring_A(n)$ by $\underline n_A$ and $\nexp_A(x,n)$ by $x^n$.
\end{definition}

The following is again trivial to prove.
\begin{proposition} For every $n$, the function $\cdot^n:A\to A$ satisfies
\axiom{Set_4}.
\end{proposition}

Based on this result, we will often see $x^n$ as the application of
$\cdot^n:[A\to A]$ to $x$.

\begin{definition}
A \emph{field structure} over $A$ is a ring structure over $A$ together with
an operation $\cdot^{-1}:A\to A$ defined on elements different from $0$
(that is, we can only write $x^{-1}$ if we know that $x\neq 0$)
satisfying
\[\axiom{F} : x\neq 0 \to x\times_A x^{-1}=_A 1_A.\]
By a field $A$ we mean a type $A$ with a field structure over it.
\end{definition}

\begin{definition} On a field $A$, we define
$/_A:A\to A\to A$ by \[x/_A y:= x\times_A(y^{-1}).\]
\end{definition}

The following is trivial:

\begin{proposition} $/_A$ satisfies \axiom{Set_5'}:
\[
\axiom{Set_5'} : \forall_{x,x',y,y':A}.
 y\neq 0\wedge y'\neq 0\to x=_A x'\wedge y=_A y'\to x/_A y=_A x'/_A y'
\]
\end{proposition}
We will sometimes abuse notation and refer to \axiom{Set_5'} as an
instance of \axiom{Set_5}, and refer to $/_A$ as an operation of type
$[A\to A\to A]$.

\begin{definition}\label{defn:proof} A \emph{proof} of $t_1=_A t_2$ from
the field axioms in an environment $\Gamma$ is a sequence
$\varphi_1,\ldots,\varphi_n$ of equalities such that $\varphi_n$ is
$t_1=_A t_2$ and, for $i=1,\ldots,n$, one of the following holds.
\begin{enumerate}[--]
\item $\varphi_i$ is an instance of one of the axioms \axiom{Set_1},
\axiom{SG}, \axiom{M_1}, \axiom{M_2}, \axiom{G_1}, \axiom{G_2},
\axiom{AG} or \axiom{R_1}--\axiom{R_5}.
\item $\varphi_i$ is an instance of axiom \axiom{F} with hypothesis
in $\Gamma$.
\item $\varphi_i$ is an instance of one of the axioms
\axiom{Set_2}--\axiom{Set_5} and the hypothesis(es) of the axiom are
included in $\{\varphi_1,\ldots,\varphi_{i-1}\}$.
\end{enumerate}
\end{definition}

We will often not mention $\Gamma$ explicitly, but assume that all the
proofs are done in an environment containing all the necessary inequalities.
The reason for this (and for choosing the term ``environment'' rather than
``context'') is that {\rational} only looks at the equality being proved
and assumes all needed inequalities hold anyway.

\begin{definition}\label{defn:less} Let $A$ be a type.  We define the relation
$\less$ on the terms of type $A$ as the least relation satisfying:
\begin{enumerate}
\item $t\less f(t)$ for $f:[A\to A]$ (in particular, in a group one has
$t\less -_A t$ and in a ring $t\less t^n$ for $n:\nat$);
\item $t_i\less f(t_1,t_2)$ for $f:[A\to A\to A]$ and $i=1,2$ (in
particular, $f$ can be one of $+_A$, $-_A$ or $\times_A$ in a group or
ring);
\item if $A$ is a field, then $t_i\less t_1/_A t_2$ for $i=1,2$.
\end{enumerate}
(Notice the implicit requirement $t_2\neq 0$ in the clause
$t_i\less t_1/_A t_2$.)
\end{definition}

\begin{proposition}\label{lesswff}
$\less$ is a well founded relation.
\end{proposition}
\begin{pf}
By definition, if $t_1\less t_2$ then $t_1$ is a subterm of $t_2$; since
``being a subterm of'' is a well founded relation, so is $\less$.
\end{pf}

\begin{notation} From now on, we will omit the subscript $A$ in the
symbols denoting the algebraic operations, since no ambiguity is introduced.
However, we will write $=_A$ to emphasize the distinction between this
defined equality and the one induced by $\beta\delta\iota$-reduction on
the set of lambda terms of type $A$.
\end{notation}

\subsection{The syntactic level}\label{expressions}

We now introduce the syntactic counterpart to the type of fields, which
is the type of expressions that {\rational} works with.

\begin{definition}\label{defn:expressions}
The syntactic type $E$ of expressions is the inductive type generated by the
following grammar:
\[E ::= \Z \alt \V_0 \alt \V_1(E) \alt E+E \alt E\times E \alt E/E\]
where $\V_i=\{v^i_j|j\in\nat\}$ for $i=0,1$.
\end{definition}

\begin{definition}\label{expr:abbr}
We define the following \emph{abbreviations} on expressions:
\begin{eqnarray}
\label{expr:inv} -e & := & e\times(-1)\\
\label{expr:minus} e_1-e_2 & := & e_1+(-e_2)\\
\label{expr:exp:zero} e^0 & := & 1\\
\label{expr:exp:suc} e^{n+1} & := & e\times e^n
\end{eqnarray}
These abbreviations are done only on the meta-level; when
we write e.g.\ $e_1-e_2$ we are speaking about the expression
$e_1+(e_2\times(-1))$.
\end{definition}

\begin{definition}\label{defn:exprorder}
The \emph{order} on $E$ is defined as follows, where $\star$ stands for $+$,
$\times$ or $/$.
\begin{enumerate}[(i)]
\item $v^0_i<_E v^0_j$ if $i<j$;
\item $v^0_i<_E e$ whenever $e$ is $i:\Z$, $e_1\star e_2$ or $v^1_i(e')$;
\item $i<_E j$ if $i<j$ ($i,j:\Z$);
\item $i<_E e$ whenever $e$ is $e_1\star e_2$ or $v^1_i(e')$;
\item $e_1\star e_2<_E e'_1\star e'_2$ whenever $e_1<_E e'_1$ or $e_1=e'_1$ and
$e_2<_E e'_2$ (lexicographic ordering);
\item $e_1+e_2<_E e$ whenever $e$ is $e'_1\times e'_2$, $e'_1/e'_2$
or $v^1_i(e')$;
\item $e_1\times e_2<_E e$ whenever $e$ is $e'_1/e'_2$ or $v^1_i(e')$;
\item $e_1/e_2<_E e$ whenever $e$ is $v^1_i(e')$;
\item $v^1_i(e_1)<_E v^1_j(e_2)$ whenever $i<j$ or $i=j$ and $e_1<_E e_2$.
\end{enumerate}
In other words, expressions are recursively sorted by first looking at
their outermost operator $$v^0_i <_E i <_E e+f <_E e\times f <_E e/f <_E
v^1_i(e)$$ and then sorting expressions with the same operator using a
lexicographic ordering.
For example:
$$v^0_1<_E 4<_E v^0_1/4<_E v^1_0(v^0_1+3)<_E v^1_0(2\times v^0_3)<_E v^1_7(v^0_1+3).$$
\end{definition}

\subsection{Object level and meta-level}\label{levels}

This section explains the difference between the several kinds of terms
in this paper.

We deal with algebraic structures (groups, rings
and fields).
And, as we are working with formal systems, we also have
\emph{terms} that are interpreted in these algebraic structures.
To complicate things, we use the method of \emph{reflection}
which means that the notion of ``term'' both occurs on
the meta-level as well as on the object level.
We will identify various instances of ``number zero'' as an example to explain
the situation.

Let us start with the \emph{object level}.
We have three kinds of objects that have a ``zero''.
\begin{itemize}
\item
\emph{The natural numbers and the integers.}
First of all, we have the natural numbers $\nat$.
In the natural numbers there is a unique object which is the natural number
zero.

The equality that one uses for the natural numbers
is \emph{Leibniz equality}.
This means that the zero of the natural numbers does not have
different representations:
there is only one zero.

The integers are like the natural numbers:
there is exactly one integer zero, and one uses Leibniz equality
to compare integers.

\item
\emph{The elements of the algebraic structures.}
Each group, ring, or field $A$ has a zero as well.  However, we use
setoids for these algebraic structures (so we model
quotients in a type-theoretical way), meaning that an algebraic
structure can have more than one object that \emph{represents} the
zero of that structure.  In other words, for an algebraic structure we
use a defined \emph{setoid equality} instead of Leibniz equality.

For instance, suppose we construct the real numbers as Cauchy
sequences of rational numbers.
Then \emph{every} Cauchy sequence that converges to zero represents the
zero of these ``Cauchy reals''.
These sequences are then all ``setoid equal'' to each other, but can
be distinguished using Leibniz equality.

\item
\emph{Field expressions.}
Finally we have the inductively defined set $E$.
In this set there is a unique term for ``zero''.
For these ``field expressions'' we also use Leibniz equality.

\end{itemize}

\noindent
All these entities exist
on the object level (as a set of mathematical objects, like
the natural numbers),
but we also have the \emph{meta-level},
the formal language that we use to talk about all these objects.
This means there is still another kind of zero:
\begin{itemize}
\item
\emph{Terms on the meta-level.}
For the integers there is a constant in the language that
denotes the integer zero.
This symbol is a linguistic construction that differs from
the integer zero itself, in that it exists on the meta-level
instead of at the object level.

Similarly, there is a function in the language that maps each
algebraic structure $A$ to a zero element $0_A$
of that structure.
Again, the symbol for this function is different from those zero elements itself,
in that it exists on the meta-level.
Note that this function denotes one specific zero of the structure
among all the elements that are setoid equal to it.

Finally, in the case of the field expressions, the basic terms denoting them
on the meta-level are very similar to the objects themselves.
Still one should distinguish the two.

\end{itemize}

\noindent
On the terms of the language there are two notions of equality.
There is \emph{syntactic identity}, and there is \emph{convertibility}.
(These equalities are used to talk about the language, and cannot
be expressed in the language itself.)

Consider the function ``$\zring$'' that maps the integers into a given ring.
Then if one applies this function to the integer zero, one gets 
a term that is syntactically different from the term that denotes the zero of
the ring.
However it is convertible to this term, by unfolding the definition of
$\zring$ and computing the resulting term.

We will almost everywhere use syntactic identity in this paper.
The only conversion that we will refer to is computation of a few
basic functions: subtraction, the $\zring$ function and exponentiation
with a constant natural number.
Of all other functions the definition will never be unfolded.

\medskip
\noindent
One should take care to distinguish between the field expressions
on the object level and the terms on the meta-level.
The field expressions can only involve variables and field operations,
while
the meta-level terms can involve any type of sub-term, as long as
the full term denotes an element of the field.
For instance in the field of real numbers $\sqrt{\zring(2)}$ is an acceptable
meta-level term,
but there is nothing like a square root in field expressions.

This distinction
can be illustrated with two relations that are defined above.
The relation $<_E$ is defined on the field expressions in Definition \ref{defn:exprorder}.
The relation $\less$ is defined on meta-level terms denoting
elements of the field $A$ in Definition \ref{defn:less}.
Note that this second relation does not respect convertibility:
${1_A}^0$ (``one to the power zero'') is convertible with $1_A$, but $1_A \less {1_A}^0$ while $1_A \not\less 1_A$.


\subsection{The interpretation relation}\label{interpretation}

The final ingredient for {\rational}
is an interpretation relation, described in detail
in~\cite{geu:wie:zwa:00} and~\cite{lcf:wie:04}.
It is this relation that allows us to speak of correctness and completeness
of {\rational}, which is what we want to do.

%\subsubsection*{Valuations}

The type $E$ includes families of variables so that we can
speak about arbitrary expressions in a field, besides those
that only mention the field operations.
Therefore, the interpretation of an expression is dependent on a valuation -- an assignment
of values to the variables.

\begin{definition} A \emph{valuation} from a type of variables $\V$
to a type $T$ is a finite (partial) function from $\V$ to $T$.
\end{definition}

\begin{notation}
The domain of a valuation $\rho$ will be denoted $\domain(\rho)$.
We will use the notation $[v:=t]$ for the valuation that
replaces $v$ with $t$.
\end{notation}

The following definitions and results are standard from the theory of
finite functions.

\begin{definition} Let $\rho,\sigma$ be valuations from $\V$ to $T$.
If $\rho$ and $\sigma$ coincide on the intersection of their domains
(in particular, if their domains are disjoint), we define the
\emph{union} $\rho\cup\sigma$ to be the only valuation $\theta$
with domain $\domain(\rho)\cup\domain(\sigma)$ such that $\theta(v)=\rho(v)$,
for $v\in\domain(\rho)$, and $\theta(v)=\sigma(v)$ for $v\in\domain(\sigma)$.
\end{definition}

\begin{definition}\label{val:incl} Let $\rho,\sigma$ be valuations
from $\V$ to $T$.  We say that $\sigma$ \emph{extends} $\rho$, denoted
by $\rho\subseteq\sigma$, if there is a valuation $\theta$ from $\V$
to $T$ such that $\sigma=\rho\cup\theta$.
\end{definition}

\begin{proposition}\label{val:incl:props}
For all $\V$ and $T$, $\subseteq$ is a partial order on the set of
valuations from $\V$ to $T$.
\end{proposition}
%\begin{pf} Trivial.
%\end{pf}

\begin{proposition}\label{val:incl:wdef}
Let $\rho,\sigma$ be valuations from $\V$ to $T$ such that
$\rho\subseteq\sigma$.  Then $\sigma(v)=\rho(v)$ for every
$v\in\domain(\rho)$.
\end{proposition}
%\begin{pf} By definition of $\subseteq$, there is a valuation $\theta$
%such that $\sigma=\rho\cup\theta$; by definition of $\cup$, since
%$v\in\domain(\rho)$, $\sigma(v)=(\rho\cup\theta)(v)=\rho(v)$.
%\end{pf}

\begin{definition} A valuation $\rho$ is \emph{injective} if, for
distinct variables $x$ and $y$ in $\V$, the terms $\rho(x)$
and $\rho(y)$ are syntactically distinct.
\end{definition}

From now on, we assume a fixed field $A$.

\begin{definition}\label{valpair} A \emph{valuation pair over $A$}
is a pair $\rho=\langle\rho_0,\rho_1\rangle$ where $\rho_0$ and
$\rho_1$ are injective valuations from, respectively, $\V_0$ to $A$
and $\V_1$ to $[A\to A]$.
\end{definition}

The results about valuations generalize in the obvious way to
valuation pairs.

\begin{definition}\label{valpair:incl}
Let $\rho$ be a valuation pair over $A$.  We say that $\sigma$
\emph{extends} $\rho$, denoted by $\rho\subseteq\sigma$, if
$\rho_i\subseteq\sigma_i$ for $i=0,1$.
\end{definition}

\begin{proposition}\label{valpair:incl:props}
$\subseteq$ is a reflexive and transitive relation on the set of valuation
pairs over $A$.
\end{proposition}

\begin{proposition}\label{valpair:incl:wdef}
Let $\rho,\sigma$ be valuation pairs over $A$ such that
$\rho\subseteq\sigma$.  Then $\sigma_i(v^i_k)=\rho_i(v^i_k)$ for $i=0,1$
and $v^i_k\in\domain(\rho_i)$.
\end{proposition}

%\subsubsection*{Interpretation of expressions}

\begin{definition}\label{defn:intrel}
Let $\rho$ be a valuation pair over $A$.  The \emph{interpretation
relation} $\intrel\subseteq E\times A$ is defined inductively by:
\begin{eqnarray}
\label{intrel:var0}
\rho_0(v^0_i)=_A t & \rightarrow & v^0_i\intrel t\\
\label{intrel:int}
\underline k=_A t & \rightarrow & k\intrel t\\
\label{intrel:plus}
 e_1\intrel t_1 \wedge e_2\intrel t_2 \wedge t_1+t_2=_A t
& \rightarrow & e_1+e_2\intrel t\\
\label{intrel:mult}
 e_1\intrel t_1 \wedge e_2\intrel t_2 \wedge t_1\times t_2=_A t
& \rightarrow & e_1\times e_2\intrel t\\
\label{intrel:div}
 e_1\intrel t_1 \wedge e_2\intrel t_2 \wedge t_2\neq 0 \wedge t_1/t_2=_A t
& \rightarrow & e_1/e_2\intrel t\\
\label{intrel:var1}
 e\intrel t_1 \wedge \rho_1(v^1_i)(t_1)=_A t & \rightarrow & v^1_i(e)\intrel t
\end{eqnarray}
Notice that by omitting~(\ref{intrel:div}) we obtain an interpretation relation
over rings; omitting also~(\ref{intrel:mult}) and~(\ref{intrel:int}) for
$k\neq 0$ we obtain an interpretation relation over groups.
\end{definition}

\begin{lemma}\label{intrel:abbr} The abbreviated expressions
(Definition~\ref{expr:abbr}) satisfy the following relations.
\begin{eqnarray}
\label{intrel:inv}
e\intrel t_1 \wedge -t_1=_A t & \rightarrow & -e\intrel t\\
\label{intrel:minus}
e_1\intrel t_1 \wedge e_2\intrel t_2 \wedge t_1-t_2=_A t
 & \rightarrow & e_1-e_2\intrel t\\
\label{intrel:nexp}
e\intrel t_1 \wedge t_1^n=_A t & \rightarrow & e^n\intrel t
\end{eqnarray}
\end{lemma}
\begin{pf}
The three cases are similar; we show~(\ref{intrel:inv}).
Recall that $-e=e\times(-1)$.  By \axiom{Set_1}, $-1=_A -1$;
hence $-1\intrel -1$ by~(\ref{intrel:int}).  By hypothesis $e\intrel
t_1$.  Finally, since $-t_1=_A t$, one has $t_1\times(-1)=_A t$, whence
$e\times(-1)\intrel t$ by~(\ref{intrel:mult}).
For~(\ref{intrel:nexp}) proceed by induction on $n$.
\end{pf}
%\begin{pf}
%\begin{enumerate}[(1)]%
%\setcounter{enumi}{\ref{intrel:inv}}\addtocounter{enumi}{-1}
%\item Recall that $-e=e\times(-1)$.  By \axiom{Set_1}, $-1=_A -1$;
%hence $-1\intrel -1$ by~(\ref{intrel:int}).  By hypothesis $e\intrel
%t_1$.  Finally, since $-t_1=_A t$, one has $t_1\times(-1)=_A t$, whence
%$e\times(-1)\intrel t$ by~(\ref{intrel:mult}).
%\item By definition, $e_1-e_2=e_1+(-e_2)$.  By hypothesis, $e_1\intrel
%t_1$ and $e_2\intrel t_2$.  Since $-t_2=_A -t_2$ by \axiom{Set_1}, one
%has that $-e_2\intrel -t_2$ by~(\ref{intrel:inv}).  By hypothesis
%$t_1-t_2=_A t$, that is (by definition of $-_A$), $t_1+(-t_2)=_A t$.
%Hence, by~(\ref{intrel:plus}) $e_1+(-e_2)\intrel t$.
%\item By induction on $n$.  If $n$ is $0$, then
%by~(\ref{expr:exp:zero}) $e^n=1$.  By hypothesis $t_1^0=_A t$, or simply
%$1=_A t$ since $t_1^0=1$ by~(\ref{nexp:zero}).  Hence $1\intrel t$
%by~(\ref{intrel:int}).

%Consider now $n=m+1$; then $e^n=e\times e^m$ by~(\ref{expr:exp:suc}).
%By hypothesis $t_1^{m+1}=_A t$, that is to say, $t_1\times t_1^n=_A t$
%according to~(\ref{nexp:suc}).  But $e\intrel t_1$ by hypothesis, and
%also $e^m\intrel t_1^m$ by induction hypothesis; hence $e\times
%e^m\intrel t$ by~(\ref{intrel:mult}).
%\end{enumerate}
%\end{pf}

\begin{lemma}\label{inclinterp}
Let $e:E$, $t:A$ and $\rho,\sigma$ be valuation pairs for $A$ with
$\rho\subseteq\sigma$ such that $e\intrel t$; then $e\intrels t$.
\end{lemma}
\begin{pf} By induction on the proof of $e\intrel t$.
\begin{enumerate}
\item $e=v^0_i$ and $\rho_0(v^0_i)=_A t$: since $\rho\subseteq\sigma$,
Proposition~\ref{valpair:incl:wdef} implies that $\sigma_0(v^0_i)=_A t$,
and by~(\ref{intrel:var0}) also $v^0_i\intrels t$.
\item $e=n$ and $\underline n=_A t$: then $n\intrels t$ follows
by~(\ref{intrel:int}).
\item $e=e_1+e_2$, $e_1\intrel t_1$, $e_2\intrel t_2$ and $t_1+t_2=_A t$;
by induction hypothesis $e_1\intrels t_1$ and $e_2\intrels t_2$, whence
$e_1+e_2\intrels t$ follows from~(\ref{intrel:plus}).
\item $e=e_1\times e_2$: analogous using~(\ref{intrel:mult}).
\item $e=e_1/e_2$: similar from~(\ref{intrel:div}),
since by hypothesis $t_2\neq 0$.
\item $e=v^1_i(e')$, $e'\intrel t'$ and $\rho_1(v^1_i)(t')=_A t$: since
$\sigma_1(v^1_i)=\rho_1(v^1_i)$ by Proposition~\ref{valpair:incl:wdef},
also $\sigma_1(v^1_i)(t')=_A t$.  By induction hypothesis $e'\intrels t'$,
whence $v^1_i(e')\intrels t$ by~(\ref{intrel:var1}).
\end{enumerate}
\end{pf}

\begin{lemma}\label{intrelfunction} Let $e:E$, $t,t':A$ and $\rho$ be
a valuation pair for $A$ such that $e\intrel t$ and $e\intrel t'$.
Then the following hold.
\begin{enumerate}[(i)]
\item $t=_A t'$;
\item if $e\intrel t$ and $e\intrel t'$ can be proved without
using~(\ref{intrel:div}), and no divisions occur in either $t$ or $t'$,
then $t=_A t'$ can be proved without using the axiom \axiom{F}.
\end{enumerate}
\end{lemma}
\begin{pf} By induction on $\intrel$ (Coq checked).
\end{pf}

\section{From terms to expressions}\label{quoteing}

We now start looking at the actual implementation of {\rational}, focusing
on the ML program inside it.
This program computes a partial
inverse to the interpretation relation described above, that is, given
a term $t:A$, with $A$ a field, it returns an expression $e$
and a valuation pair $\rho$ such that $e\intrel t$.
In this section we formally describe this program as a function
``quote'', $\mlfn{\cdot}{\cdot}$, and prove its correctness.

\subsection{Quoting terms to variables}

The first step is to define what to do when we meet a term that is not
built from the field operations, e.g. a variable or an expression
like $\sqrt 2$.

\begin{definition}\label{defn:quotev0}
Let $t:A$ and $\rho$ be a valuation pair over $A$.  Then
$\mlfnv{t}{\rho}$ is the pair $\langle v,\sigma\rangle$ with $v\in\V_0$
defined by:
\begin{itemize}
\item if there is an $i$ such that $\rho_0(v^0_i)=t$, then $v=v^0_i$ and
$\sigma=\rho$;
\item else, let $k$ be minimal such that $\rho_0(v^0_k)$ is not defined
and take $v=v^0_k$ and $\sigma_0=\rho_0\cup[v^0_k:=t]$, $\sigma_1=\rho_1$.
\end{itemize}
The behavior of $\mlfnv{\cdot}{}$ can be described as follows: given a term $t$ and
a valuation $\rho$, it checks whether there is a variable $v^0_i$ such that
$\rho_0(v^0_i)=t$.  In the affirmative case, it returns this variable and
$\rho$; else it extends $\rho_0$ with a fresh variable which is interpreted
to $t$ and returns this variable and the resulting valuation.  Notice
that the result is deterministic, since there is at most one variable $i$
satisfying $\rho_0(v^0_i)=t$.
\end{definition}

\begin{lemma}\label{quotev0incl}
Let $t$ and $\rho$ be as in Definition~\ref{defn:quotev0}, and suppose
that $\mlfnv{t}{\rho}=\langle v,\sigma\rangle$.  Then $\rho\subseteq\sigma$.
\end{lemma}
\begin{pf}
Straightforward by definition of $\mlfnv{\cdot}{}$.
\end{pf}
%\begin{pf}
%Immediate: either $\sigma=\rho$, and we invoke reflexivity of $\subseteq$
%(Proposition~\ref{valpair:incl:props}), or
%$\langle\sigma_0,\sigma_1\rangle=\langle\rho_0\cup[v^0_k:=t],\rho_1\rangle$
%with $v^0_k\not\in\domain(\rho_0)$, which directly satisfies the definition of
%$\subseteq$.
%\end{pf}

\begin{lemma}\label{quotev0corr}
Let $t$ and $\rho$ be as in Definition~\ref{defn:quotev0}, and suppose
that $\mlfnv{t}{\rho}=\langle v,\sigma\rangle$.  Then $v\intrels t$.
\end{lemma}
\begin{pf}
By definition of $\mlfnv{\cdot}{}$, there are two cases:
\begin{itemize}
\item There is an $i$ such that $\rho_0(v^0_i)=t$, and then also
$\rho_0(v^0_i)=_A t$ by \axiom{Set_1}, whence $v^0_i\intrel t$
by~(\ref{intrel:var0}).  Since in this case $v=v^0_i$ and
$\sigma=\rho$, the thesis follows.
\item There is no such $i$; then $v=v^0_k$ where $k\not\in\domain(\rho_0)$,
and $\sigma_0=\rho_0\cup[v^0_k:=t]$.  Then, by definition of $\cup$,
$\sigma_0(v^0_k)=t$ and we can conclude as above that $v^0_k\intrels t$
using \axiom{Set_1} and~(\ref{intrel:var0}).
\end{itemize}
\end{pf}

\begin{definition}\label{defn:quotev1}
Let $f:[A\to A]$ and $\rho$ be a valuation pair over $A$.  Then
$\mlfnv{f}{\rho}$ is the pair $\langle v,\sigma\rangle$ with $v\in\V_1$
defined by:
\begin{itemize}
\item if there is an $i$ such that $\rho_1(v^1_i)=f$, then $v=v^1_i$ and
$\sigma=\rho$;
\item else, let $k$ be minimal such that $\rho_1(v^1_k)$ is not defined
and take $v=v^1_k$ and $\sigma_0=\rho_0$, $\sigma_1=\rho_1\cup[v^1_k:=t]$.
\end{itemize}
\end{definition}

\begin{lemma}\label{quotev1incl}
Let $t$ and $\rho$ be as in Definition~\ref{defn:quotev1}, and suppose
that $\mlfnv{t}{\rho}=\langle v,\sigma\rangle$.  Then $\rho\subseteq\sigma$.
\end{lemma}
\begin{pf}
Straightforward by definition of $\mlfnv{\cdot}{}$.
%Analogous to the proof of Lemma~\ref{quotev0incl}.
\end{pf}

\begin{lemma}\label{quotev1corr}
Let $f:[A\to A]$ and $\rho$ be a valuation pair over $A$ and suppose
that $\mlfnv{f}{\rho}=\langle v,\sigma\rangle$.  Suppose that $t\intrel e$;
then $v(t)\intrels f(e)$.
\end{lemma}
\begin{pf}
By definition of $\mlfnv{\cdot}{}$, there are again two cases:
\begin{itemize}
\item There is an $i$ such that $\rho_1(v^1_i)=f$; then
$\rho_1(v^1_i)(t)=_A f(t)$ by \axiom{Set_1}, whence $v^1_i(e)\intrel
f(t)$ by~(\ref{intrel:var1}).  Since in this case $v=v^1_i$ and
$\sigma=\rho$, the thesis follows.
\item There is no such $i$; then $v=v^1_k$ where $k\not\in\domain(\rho_1)$,
and $\sigma_1=\rho_1\cup[v^1_k:=t]$.  Then, by definition of $\cup$,
$\sigma_1(v^1_k)=f$ and we can conclude as above that
$v^1_k(e)\intrels f(t)$ using \axiom{Set_1} and~(\ref{intrel:var1}).
\end{itemize}
\end{pf}

\subsection{Quoting arbitrary terms}

We can now define the quote function.

\begin{definition}\label{defn:quote}
Let $t:A$ and $\rho$ be a valuation pair over $A$.  Then
$\mlfn{t}{\rho}$ is recursively defined as follows.
\[
\begin{array}{rcll}
\mlfn{\underline n}{\rho} & = & \langle n,\rho\rangle
 & \mbox{ $n:\Z$ closed}\\
\mlfn{t_1\star t_2}{\rho} & = & \langle e_1\star e_2,\sigma\rangle
 & \mbox{where }\left\{\begin{array}l \star\in\{+,-,\times,/\}\\
                       \langle e_1,\theta\rangle=\mlfn{t_1}{\rho}\\
                       \langle e_2,\sigma\rangle=\mlfn{t_2}{\theta}
                       \end{array}\right. \\
\mlfn{-t}{\rho} & = & \langle -e,\sigma\rangle
 & \mbox{ where $\langle e,\sigma\rangle=\mlfn{t}{\rho}$}\\
\mlfn{t^n}{\rho} & = & \langle e^n,\sigma\rangle
 & \mbox{where }\left\{\begin{array}l n:\nat\mbox{ is closed}\\
                       \langle e,\sigma\rangle=\mlfn{t}{\rho}
                       \end{array}\right. \\
\mlfn{f(t)}{\rho} & = & \langle v^1_i(e),\theta\rangle
 & \mbox{where }\left\{\begin{array}l \langle e,\sigma\rangle=\mlfn{t}{\rho}\\
                       \langle v^1_i,\theta\rangle=\mlfnv{f}{\sigma}
                       \end{array}\right. \\
\mlfn{t}{\rho} & = & \mlfnv{t}{\rho} & \mbox{ otherwise}
\end{array}
\]
The two last clauses also define $\mlfn{\underline n}{\rho}$ and
$\mlfn{t^n}{\rho}$ when $n$ is not a closed term.

Notice that on every recursive call of $\mlfn{\cdot}{}$ the argument
decreases w.r.t\ $\less$; since $\less$ is well founded by
Proposition~\ref{lesswff}, this is a valid definition.
\end{definition}

\begin{lemma}\label{quoteincl}
Let $\rho$ be a valuation pair for $A$.  For all $t:A$ and $e:E$, if
$\langle e,\sigma\rangle=\mlfn{t}{\rho}$, then $\rho\subseteq\sigma$.
\end{lemma}
\begin{pf} By induction on $\less$ using Lemmas~\ref{quotev0incl}
and~\ref{quotev1incl} and Proposition~\ref{valpair:incl:props}.
\end{pf}

The following is the main result so far: it expresses the correctness
of the quote function.
Given a term $t$ and a valuation $\rho$, the program computes an
expression $e$ and a new valuation $\sigma$ such that $e\intrels t$.

\begin{theorem}\label{quotecorrect}
Let $t:A$ and $\rho$ be a valuation pair over $A$, and take
$\langle e,\sigma\rangle=\mlfn{t}{\rho}$.  Then $e\intrels t$.
\end{theorem}
\begin{pf} By induction on $\less$.
\begin{enumerate}
\item $t$ is minimal for $\less$:
\begin{enumerate}
\item $t=\underline n$ with $n:\Z$ closed.  Then, by definition of
quote, $e=n$ and $\sigma=\rho$.  By \axiom{Set_1},
$\underline n=_A\underline n$, and by~(\ref{intrel:int})
$n\intrel\underline n$.
\item otherwise $\langle e,\sigma\rangle=\mlfnv{t}{\rho}$.  By
Lemma~\ref{quotev0corr}, it follows that $e\intrels t$.
\end{enumerate}
\item $t=f(t')$ with $f:[A\to A]$.
\begin{enumerate}
\item $f$ is $-_A$: then
$\langle e,\sigma\rangle=\langle -e',\sigma\rangle$ with
$\langle e',\sigma\rangle=\mlfn{t'}{\rho}$.  By induction hypothesis,
$e'\intrels t'$; since $-t'=-t'$ by \axiom{Set_1}, $-e'\intrels -t'$
by~(\ref{intrel:inv}).
\item $f$ is $\cdot^n$ with $n$ closed: then
$\langle e,\sigma\rangle=\langle (e')^n,\sigma\rangle$ with
$\langle e',\sigma\rangle=\mlfn{t'}{\rho}$.  By induction hypothesis,
$e'\intrels t'$; since $(t')^n=(t')^n$ by \axiom{Set_1},
$(e')^n\intrels(t')^n$ by~(\ref{intrel:nexp}).
\item otherwise $e=v^1_i(e')$ with
$\langle e',\theta\rangle=\mlfn{t'}{\rho}$ and
$\langle v^1_i,\sigma\rangle=\mlfnv{f}{\theta}$.  By induction
hypothesis $e'\intrelt t'$; Lemma~\ref{quotev1corr} allows us to
conclude that $f(e')\intrels v^1_i(t')$.
\end{enumerate}
\item $t=t_1\star t_2$ with $\star\in\{+,-,\times,/\}$ (if $\star-/$, then
also $t_2\neq 0$): then $e=e_1\star e_2$ with
$\langle e_1,\theta\rangle=\mlfn{t_1}{\rho}$ and
$\langle e_2,\sigma\rangle=\mlfn{t_1}{\theta}$.

By induction hypothesis, $e_1\intrelt t_1$; by Lemma~\ref{quoteincl},
$\theta\subseteq\sigma$, whence by Lemma~\ref{inclinterp} also
$e_1\intrels t_1$.

Also by induction hypothesis, $e_2\intrelt t_2$.  Furthermore,
\axiom{Set_1} implies $t_1\star t_2=_A t_1\star t_2$, hence $e_1\star
e_2\intrelt t_1\star t_2$ by either~(\ref{intrel:plus}),
(\ref{intrel:minus}), (\ref{intrel:mult}) or~(\ref{intrel:div}),
according to whether $\star$ is respectively $+$, $-$, $\times$ or
$/$; in the last case, the extra condition $t_2\neq 0$ also holds.
\end{enumerate}
\end{pf}

Notice that this result is still valid if $A$ is a group or
a ring, as can be seen by removing the corresponding cases in this proof
and checking that it remains valid.

\subsection{Properties of quoting}

The remainder of this section is concerned with other properties of
the quote that are needed for the rest of the paper.

First, quote is idempotent on the second component: if the quote of $e$ with
$\rho$ is $t$ and $\sigma$, and $\theta$ is any valuation extending
$\sigma$ (in particular, $\sigma$ itself), then
$\mlfn{t}{\theta}=\langle e,\theta\rangle$.

\begin{lemma}\label{quoteidempotent}
Let $t:A$, $e:E$ and $\rho,\sigma,\theta$ be valuation pairs over $A$
such that $\langle e,\sigma\rangle=\mlfn{t}{\rho}$ and $\sigma\subseteq\theta$.
Then $\mlfn{t}{\theta}=\langle e,\theta\rangle$.
\end{lemma}
\begin{pf} By induction on $\less$ using Lemmas~\ref{inclinterp},
\ref{quotev0corr}, \ref{quotev1corr} and~\ref{quoteincl}.
\end{pf}

All variables in the expression output by a quote can be
interpreted by the corresponding valuation.

\begin{lemma}\label{quotedom}
Let $t:A$ and $\rho$ be a valuation pair over $A$.
For all variables $v^i_k$ occurring in $e$, if
$\mlfn{t}{\rho}=\langle e,\sigma\rangle$, then $v^i_k\in\domain(\sigma_k)$.
\end{lemma}
\begin{pf}
By induction on $t$ according to $\less$.  All cases follow directly from
the induction hypothesis except for the case when $t=f(t')$ with $f:[A\to A]$
not $-_A$ or $\cdot^n$ with $n$ closed.  In this situation
there exist a natural number $j$, an expression $e'$
and a valuation pair $\theta$ such that
$\mlfn{t'}{\rho}=\langle e',\theta\rangle$,
$\mlfnv{f}{\theta}=\langle v^1_j,\sigma\rangle$ and $e=v^1_j(e')$.
If $v^i_k$ is a variable occurring in $e$, then either $v^i_k=v^1_j$ and
the result holds by definition of $\mlfnv{\cdot}{}$ or $v^i_k$ occurs in $e'$;
in the latter case, by induction hypothesis $v^i_k\in\domain(\theta_i)$, and
since $\theta\subseteq\sigma$ also $v^i_k\in\domain(\sigma_i)$.
\end{pf}

\subsection{Permutation of quotes}

To prove the main properties of {\rational}, we need to consider situations
when the same terms are quoted in different orders.
This section proves some results about the corresponding outputs: under
quite general hypotheses, they differ only in the names of the variables.

\begin{definition}\label{renamevars}
Let $\rho,\sigma$ be valuation pairs for $A$.  We say that $\sigma$ is
obtained from $\rho$ \emph{by a renaming of variables} if there is a pair
$\xi=\langle\xi_0,\xi_1\rangle$ of permutations of $\nat$ such that,
for $i=0,1$, the following conditions hold:
\begin{itemize}
\item $\xi_i(k)\neq k\rightarrow v^i_k\in\domain(\rho_i)$;
\item for all $k$, $\sigma_i\left(v^i_{\xi_i(k)}\right)\simeq\rho_i(v^i_k)$
(that is, either they are both undefined or they are both defined and
coincide).
\end{itemize}
We denote this situation by {\isrenamevar\sigma\rho\xi} and say that $\xi$ is
a renaming of variables for $\rho$ (or simply $\xi$ is a renaming of
variables, if the $\rho$ is not relevant).
Also, we will abuse notation and write $\domain(\xi_i)=\{k|\xi_i(k)\neq k\}$
for $i=0,1$; it follows that $\xi_i(j)=j$ if $j\not\in\domain(\xi_i)$.  The
first condition then becomes simply $\domain(\xi_i)\subseteq\domain(\rho_i)$.
\end{definition}

Notice that the second condition totally defines $\sigma$, since
each $\xi_i$ is a permutation.  For this reason, we will also use the
notation {\isrenamevar\sigma\rho\xi} as a definition of $\sigma$.
Note also that, if {\isrenamevar\sigma\rho\xi},
then $\domain(\sigma_i)=\domain(\rho_i)$ for $i=0,1$.

The following is immediate.
\begin{proposition}\label{renamevarsequiv}
The relation $R(\rho,\sigma)$ defined as ``$\sigma$ is obtained from $\rho$
by a renaming of variables'' is an equivalence relation.
\end{proposition}
%\begin{pf}
%Straightforward.
%\end{pf}

\begin{definition}
Let $\xi$ be a renaming of variables and $e,e':E$.  We say that $e'$
is obtained from $e$ by $\xi$, denoted {\isrenamevar{(e')}e\xi}, if $e'$
is obtained from $e$ by replacing each occurrence of $v^i_k$ by
$v^i_{\xi_i(k)}$, $i=0,1$.
\end{definition}

\begin{lemma}\label{renamequote}
Let $\rho$ be a valuation pair for $A$ and $\xi$ be a renaming of
variables for $\rho$.  For every $t:A$, if $\mlfn{t}{\rho}=\langle
e,\sigma\rangle$ then $\mlfn{t}{\renamevar\rho\xi}=%
\langle\renamevar{e}\xi,\renamevar\sigma\xi\rangle$.
\end{lemma}
\begin{pf} By induction on $\less$.
\begin{enumerate}
\item $t$ is minimal for $\less$:
\begin{enumerate}
\item $t=\underline n$: then $e=n$, $\sigma=\rho$,
$\mlfn{t}{{\renamevar\rho\xi}}=\langle n,{\renamevar\rho\xi}\rangle$ and the 
conclusion trivially holds.
\item otherwise, $e=\mlfnv{t}{\rho}$ and we have to distinguish two
cases.

Suppose there is an $i$ such that $\rho_0(v^0_i)=t$.  Then $e=v^0_i$ and
$\sigma=\rho$; but by Definition~\ref{renamevars}
$\renamevar\rho\xi_0\left(v^0_{\xi_0(i)}\right)=\rho_0(v^0_i)$, so
$\mlfn{t}{\renamevar\rho\xi}=\langle v^0_{\xi_0(i)},\renamevar\rho\xi\rangle$,
and by definition {\isrenamevar{v^0_{\xi_0(i)}}{(v^0_i)}\xi}.

Otherwise, pick $k$ minimal such that $v^0_k\not\in\domain(\rho_0)$.
Then $e=v^0_k$ and $\sigma=\langle\rho_0\cup[v^0_k:=t],\rho_1\rangle$.
But then $\mlfn{t}{{\renamevar\rho\xi}}=\langle v^0_k,\sigma'\rangle$ with
$\sigma'=\langle\renamevar\rho\xi_0\cup[v^0_k:=t],\renamevar\rho\xi_1\rangle$: since
$\domain(\rho_0)=\domain(\renamevar\rho\xi_0)$ (see remark after
Definition~\ref{renamevars}), $k$ is also the minimal natural number
satisfying $v^0_k\not\in\domain(\renamevar\rho\xi_0)$; furthermore, there can be no
$i$ such that $\renamevar\rho\xi_0(v^0_i)=t$ because
$\renamevar\rho\xi_0(v^0_i)=\rho_0(v^0_{\xi^{-1}_0(i)})$.  But $k\not\in\domain(\xi_0)$,
so {\isrenamevar{v^0_k}{v^0_k}\xi} and {\isrenamevar{\sigma'}\sigma\xi}.
\end{enumerate}
\item $t=f(t')$ with $f:[A\to A]$.
\begin{enumerate}
\item $f$ is $-_A$: then there is an expression $e'$ such that
$\mlfn{t'}{\rho}=\langle e',\sigma\rangle$ and $e=-e'$.
By induction hypothesis,
$\mlfn{t'}{{\renamevar\rho\xi}}=\langle\renamevar{(e')}\xi,\renamevar\sigma\xi\rangle$
and hence $\mlfn{t}{{\renamevar\rho\xi}}=\langle\renamevar e\xi,\renamevar\sigma\xi\rangle$.
\item $f$ is $\cdot^n$ with $n$ closed: analogous.
\item otherwise there exist an expression $e'$, an index $i$ and
a valuation pair $\theta$ such that
$\mlfn{t'}{\rho}=\langle e',\theta\rangle$,
$\mlfnv{f}{\theta}=\langle v^1_i,\sigma\rangle$ and $e=v^1_i(e')$.
By induction hypothesis, $\mlfn{t'}{{\renamevar\rho\xi}}=%
\langle\renamevar{(e')}\xi,\renamevar\theta\xi\rangle$.

Suppose there is an $k$ such that $\theta_1(v^1_k)=f$.  Then $i=k$ and
$\sigma=\theta$; but by Definition~\ref{renamevars}
$\renamevar\theta\xi_1\left(v^1_{\xi_1(i)}\right)=\theta_1(v^1_i)=f$, so
$\mlfnv{f}{\renamevar\theta\xi}=\langle v^1_{\xi_1(i)},\renamevar\theta\xi\rangle$.
Trivially {\isrenamevar{v^1_{\xi_1(i)}}{(v^1_i)}\xi}; since $\theta=\sigma$,
$\mlfn{t}{{\renamevar\rho\xi}}=\langle\renamevar{v^1_i(e')}\xi,\renamevar\sigma\xi\rangle$,
which establishes the result.

Otherwise, $i$ is the minimal $k$ such that $v^1_k\not\in\domain(\theta_1)$
and $\sigma=\langle\theta_0,\theta_1\cup[v^1_i:=f]\rangle$.
But then $\mlfnv{f}{\renamevar\theta\xi}=\langle v^1_i,\sigma'\rangle$ with
$\sigma'=\langle\renamevar\theta\xi_0,\renamevar\theta\xi_1\cup[v^1_i:=f]\rangle$:
since $\domain(\theta_1)=\domain(\renamevar\theta\xi_1)$ (second condition in
Definition~\ref{renamevars}), $i$ is also the minimal $k$
satisfying $v^1_k\not\in\domain(\renamevar\theta\xi_1)$; furthermore, there can be no
$k$ such that $\renamevar\rho\xi_1(v^1_k)=f$, since
$\renamevar\theta\xi_1(v^1_k)=\theta_1(v^1_{\xi^{-1}_1(k)})$.
But then {\isrenamevar{\sigma'}\sigma\xi}; since $i=\xi_1(i)$,
we also have in this situation that
$\mlfn{t}{\renamevar\rho\xi}=\langle\renamevar{v^1_i(e')}\xi,\renamevar\sigma\xi\rangle$.
\end{enumerate}
\item $t=t_1\star t_2$ with $\star\in\{+,-,\times,/\}$: then there are
expressions $e_1,e_2$ and a valuation pair $\theta$ such that
$\mlfn{t_1}{\rho}=\langle e_1,\theta\rangle$,
$\mlfn{t_2}{\theta}=\langle e_2,\sigma\rangle$ and $e=e_1\star e_2$.

By induction hypothesis $\mlfn{t_1}{\renamevar\rho\xi}=%
\langle\renamevar{e_1}\xi,\renamevar\theta\xi\rangle$.
The induction hypothesis applies again, and
$\mlfn{t_2}{\renamevar\theta\xi}=\langle\renamevar{e_2}\xi,%
\renamevar\sigma\xi\rangle$.  Hence, $\mlfn{t_1\star t_2}{\renamevar\rho\xi}=%
\langle(\renamevar{e_1}\xi)\star(\renamevar{e_2}\xi),%
\renamevar\sigma\xi\rangle$ and trivially
$(\renamevar{e_1}\xi)\star(\renamevar{e_2}\xi)=\renamevar e\xi$.
\end{enumerate}
\end{pf}

The next step is to prove the following result: if the order in which two
terms are quoted is reversed, the expressions and valuations obtained
will differ only by a renaming of variables.
\begin{lemma}\label{quotecommutes}
Let $t_1,t_2:A$, $e_1,e'_1,e_2,e'_2:E$ and
$\rho,\sigma,\sigma',\theta,\theta'$ be valuation pairs for $A$
satisfying the following relations.
\begin{eqnarray*}
\mlfn{t_1}{\rho} = \langle e_1,\sigma\rangle
 & \hspace{2cm} & \mlfn{t_2}{\rho} = \langle e'_2,\sigma'\rangle\\
\mlfn{t_2}{\sigma} = \langle e_2,\theta\rangle
 && \mlfn{t_1}{\sigma'} = \langle e'_1,\theta'\rangle
\end{eqnarray*}
Then there is a renaming of variables $\xi$ for $\theta$ such that
 for $i=1,2$, $\domain(\xi_i)\cap\domain(\rho_i)=\emptyset$,
{\isrenamevar{\theta'}\theta\xi} and {\isrenamevar{e'_i}{e_i}\xi}.
\end{lemma}

The proof of this (intuitive) result is by induction, but the
multitude of cases makes it somewhat long and not extremely
interesting.  It is detailed in~\cite{lcf:wie:04b}; the following lemma
is used in its proof, and will be needed elsewhere.
\begin{lemma}\label{quotecommv1}
Let $t:A$, $f:[A\to A]$, $e,e':E$, $i,i':\nat$ and
$\rho,\sigma,\sigma',\theta,\theta'$ be valuation pairs for $A$
satisfying the following relations.
\[
\begin{array}{ccc}
\displaystyle
\mlfnv{f}{\rho} = \langle v^1_i,\sigma\rangle
 & \hspace{2cm} & \mlfn{t}{\rho} = \langle e',\sigma'\rangle\medskip\\
\mlfn{t}{\sigma} = \langle e,\theta\rangle
 && \mlfnv{f}{\sigma'} = \langle v^1_{i'},\theta'\rangle
\end{array}
\]
Then there is a renaming of variables $\xi$ for $\theta$ such that
$\xi_0=\idn$, $\domain(\xi_1)\cap\domain(\rho_1)=\emptyset$,
{\isrenamevar{\theta'}\theta\xi}, {\isrenamevar{e'}e\xi} and $i'=\xi_1(i)$.
\end{lemma}


The following corollary will be essential in the proof of the Completeness
Theorem.

\begin{lemma}\label{quoteaddpred}
Let $t_1,t_2,t_3:A$, $e_1,e_2,e'_2,e_3,e'_3:E$ and
$\rho,\sigma,\sigma',\theta,\theta'$ be valuation pairs for $A$
satisfying the following relations.
\begin{eqnarray*}
\mlfn{t_2}{\emptyset} = \langle e_2,\sigma\rangle
 && \mlfn{t_1}{\emptyset} = \langle e_1,\rho\rangle\\
\mlfn{t_3}{\sigma} = \langle e_3,\theta\rangle
 && \mlfn{t_2}{\rho} = \langle e'_2,\sigma'\rangle\\
 &\hspace{2cm}&\mlfn{t_3}{\sigma'} = \langle e'_3,\theta'\rangle
\end{eqnarray*}
Then there exist a valuation pair $\tau$ and a renaming of variables
$\xi$ for $\tau$ such that $\theta\subseteq\tau$,
{\isrenamevar{\theta'}\tau\xi} and {\isrenamevar{e'_i}{e_i}\xi} for $i=2,3$.
\end{lemma}
\begin{pf}
Consider $\mlfn{t_1}{\sigma}=\langle e^\ast_1,\rho^\ast\rangle$,
$\mlfn{t_3}{\rho^\ast}=\langle e^\ast_3,\theta^\ast\rangle$ and
$\mlfn{t_1}{\theta}=\langle e'_1,\tau\rangle$
% (see Figure~\ref{fig:quoteaddpred})
and apply Lemmas~\ref{quotecommutes} and~\ref{renamequote}.
%\begin{figure}[htb]
%\[\xymatrix{
% & \emptyset \ar[dl]_{t_2} \ar[dr]^{t_1} \\
% \sigma \ar[d]_{t_3} \ar[dr]^{t_1} & & \rho \ar[d]_{t_2} \\
% \theta \ar[d]_{t_1} & \rho^\ast \ar@{.>}[r]^{\xi} \ar[d]_{t_3}
% & \sigma' \ar[d]_{t_3} \\
% \tau \ar@{.>}[r]^{\xi'} & \theta^\ast \ar@{.>}[r]^{\xi} & \theta'
%}\]
%\caption{Proof of Lemma~\ref{quoteaddpred}.  An arrow from $\rho$ to
%$\rho'$ with label $t$ means that $\rho'$ is obtained by $\mlfn{t}{\rho}$.}
%\label{fig:quoteaddpred}
%\end{figure}
\end{pf}

\section{Completeness of {\rational}: rings}\label{normalization}

We now move to the Coq portion of the tactic.
We identify a subset of the set of expressions which we call
\emph{normal forms}.  Then we define a normalization function {\N}
that assigns to any expression $e$ an expression $\N(e)$ in normal
form.  In this section we prove the fundamental properties of this function.

In this first stage we will forget about division and work only with
the subset of expressions interpretable in a ring.
Section~\ref{fields} discusses how these definitions can be
generalized for fields and how the results we show here can be
transposed to the general case.

\subsection{Normal forms}\label{nf}

The intuition for the normal forms is as follows.  A normal form is a
polynomial where all terms have been multiplied, so that it is written
as a sum of products of atomic terms (integers, variables of arity $0$ or
variables of arity $1$ applied to a normal form).  To guarantee uniqueness
of the normal form we further require that these terms be ordered.

We begin by defining monomials and polynomials.  These can be seen in
a precise way as lists of expressions; hence we can identify the
subset of monomials and polynomials whose lists are ordered.  These
will be our normal forms.

\begin{definition}\label{def:prenf}
The sets of \emph{monomials} and \emph{polynomials} are inductively
defined by the following grammar.
\begin{eqnarray*}
M' & ::= & \Z \alt \V_0\times M' \alt \V_1(P')\times M' \\
P' & ::= & \Z \alt M'+P'
\end{eqnarray*}
\end{definition}
Notice that $M'\subseteq E$ and $P'\subseteq E$.

\begin{definition}\label{def:monabs}
For every $m:M'$ we define the \emph{list of variables} of $m$, $|m|$, and
the \emph{coefficient} of $m$, $\|m\|$.
\[\begin{array}{rclcrcl}
|\cdot| : M' & \to & \mathrm{list}(E) & & \|\cdot\| : M' & \to & \Z \\
 i & \mapsto & [] & & i & \mapsto & i \\
 v^0_i\times m & \mapsto & v^0_i :: m & & v^0_i\times m & \mapsto & \|m\| \\
 v^1_i(p')\times m & \mapsto & v^1_i(p') :: m
  & & v^1_i(p')\times m & \mapsto & \|m\|
\end{array}\]
\end{definition}

\begin{definition}\label{def:polabs}
For every $p:P'$ we define the \emph{list of monomials} of $p$ as follows:
\begin{eqnarray*}
|\cdot| : P' & \to & \mathrm{list}(\mathrm{list}(E)) \\
 i & \mapsto & [] \\
 m+p & \mapsto & |m| :: |p|
\end{eqnarray*}
\end{definition}

\begin{definition}\label{def:prenftonf}
We define the following mutually recursive predicates over $M'$ and $P'$.
\begin{enumerate}[(i)]
\item $\ord_{M'}(m)$ holds if $|m|$ is an ordered list (with the ordering
from Definition~\ref{defn:exprorder}).
\item $\ord_{P'}(p)$ holds if $|p|$ is ordered (using the
lexicographic ordering for each element of $|p|$) and $|p|$ does not
contain repetitions.
\item $\wf_{M'}$ is defined recursively as follows:
\begin{itemize}
\item $\wf_{M'}(i)$ holds for $i\neq 0$;
\item $\wf_{M'}(v^0_i\times m)\iff\wf_{M'}(m)$;
\item $\wf_{M'}(v^1_i(p)\times m)\iff(\wf_{M'}(m)\wedge\nf_{P'}(p))$.
\end{itemize}
\item $\nf_{M'}(m)$ holds if either $m=0$ or $\wf_{M'}(m)\wedge\ord_{M'}(m)$
holds.
\item $\wf_{P'}$ is defined recursively as follows:
\begin{itemize}
\item $\wf_{P'}(i)$ holds for $i\in\Z$;
\item $\wf_{P'}(m+p)\iff(\wf_{P'}(p)\wedge\nf_{M'}(m))$.
\end{itemize}
\item $\nf_{P'}(p)$ holds iff $\wf_{P'}(p)\wedge\ord_{P'}(p)$ holds.
\end{enumerate}
\end{definition}

\begin{definition}\label{defn:nf}
The set of monomials \emph{in normal form} is defined as
\[M = \{m:M' \alt \nf_{M'}(m)\}.\]
The set of polynomials \emph{in normal form}, or simply of normal forms,
is defined as \[P = \{p:P' \alt \nf_{P'}(p)\}.\]
\end{definition}

We will use the definitions of $\|m\|$, $|m|$ and $|p|$ above also for
monomials and polynomials in normal form.

\begin{definition}\label{defn:coeff}
Let $m:M$ and $p:P$.  The \emph{coefficient} of $m$ in $p$, denoted by
{\coeff mp}, is recursively defined as follows.
\begin{eqnarray*}
\coeff{\underline{\ }}\cdot : P\times M & \to & \Z\\
 i, j & \mapsto & i \\
 i, m & \mapsto & 0 \\
 m'+p, m & \mapsto & \left\{%
\begin{array}{lll}\|m'\| & \mbox{if $|m'|=|m|$} \\ 
 \coeff mp & \mbox{else}\end{array}\right.
\end{eqnarray*}
The first clause in this definition may look somewhat strange; the idea is
that we only look at $|m|$ to define {\coeff mp}, and thus any integer
should correspond to the independent term of $p$.
\end{definition}

The reason for introducing the operations $|\cdot|$ and $\|\cdot\|$ is
that they totally characterize normal forms.

\begin{lemma}\label{monchar}
If $m,m':M$, then $m=m'$ iff $\|m\|=\|m'\| \wedge |m|=|m'|$.
\end{lemma}
\begin{pf}
Straightforward.
\end{pf}

\begin{lemma}\label{polchar}
If $p,q:P$, then $p=q\iff \forall m:M.\coeff pm=\coeff qm$.
\end{lemma}
\begin{pf}
The direct implication is immediate.  For the converse, assume that
$\coeff pm=\coeff qm$ for all $m$; then every monomial occurring in
$p$ also occurs in $q$ with the same coefficient, and reciprocally.
But $|p|$ and $|q|$ are both ordered, hence $p=q$.
\end{pf}

\subsection{The normalization function}\label{normfn}

The normalization function is not defined directly, but by means of a
number of auxiliary functions.  This makes it easier to state and prove
results about it.

\begin{definition}\label{defn:multMZ} {\multMZ} is defined by:
\begin{eqnarray*}
\multMZ : M \times \Z & \to & M \\
 m, 0 & \mapsto & 0 \\
 i, j & \mapsto & i\times j \\
 x\times m, j & \mapsto & x\times (m\multMZ j)
\end{eqnarray*}
\end{definition}

\begin{proposition}\label{multMZ}
{\multMZ} satisfies the following properties:
\begin{enumerate}[(i)]
\item $\|m\multMZ i\|=\|m\|\times i$;
\item $|m\multMZ 0|=[]$;
\item $|m\multMZ i|=|m|$ for $i\neq 0$;
\item {\multMZ} is well defined, i.e.\ its output is in $M$;
\item if $m\intrel t$, then $m\multMZ i\intrel t\times i$.
\end{enumerate}
\end{proposition}
\begin{pf}
The first two properties follow directly from the definition; for the third,
just notice that, if $i\neq 0$, then $\times_{M \Z}$ translates to the
identity on the list of variables of $m$.
From these three properties, the fourth then follows: if $i=0$, then this
is a consequence of $0:M$; else only the coefficient of $m$ changes, hence
$\nf_{M'}(m\multMZ i)$ still holds.
The last property is proved by straightforward induction (Coq checked).
\end{pf}

\begin{definition}\label{defn:multMV} {\multMV} is defined by:
\begin{eqnarray*}
\multMV : M \times (\V_0\cup\V_1(P)) & \to & M \\
 i, y & \mapsto & (y\times 1)\multMZ i \\
 x\times m, y & \mapsto & 
\left\{\begin{array}{ll}x\times(m\multMV y) & x<_E y \\
 y\times x\times m & \mbox{otherwise}\end{array}\right.
\end{eqnarray*}
\end{definition}

\begin{proposition}\label{multMV}
{\multMV} satisfies the following properties:
\begin{enumerate}[(i)]
\item $\|m\multMV x\|=\|m\|$;
\item if $m\neq 0$, then $|m\multMV x|$ is the sorted
list obtained from $m$ and $x$;
\item {\multMV} is well defined;
\item if $m\intrel t$ and $x\intrel t'$, then
$m\multMV x\intrel t\times t'$.
\end{enumerate}
\end{proposition}
\begin{pf}
If $m=0$ these properties follow from Proposition~\ref{multMZ}, so assume
$m\neq 0$.
The first property follows directly from the definition; for the
second, just notice that {\multMV} translates to the
algorithm of straight insertion on lists.
From these two properties, the third then follows: the elements of $|m|$
are not changed by {\multMV} and $x$ is either $v^0_i$ or
$v^1_i(p)$ with $p:P$, hence $m\multMV x$ satisfies $\wf_{M'}$.
Also the correctness of straight insertion guarantees that
$|m\multMV x|$ is sorted if $m$ is.
The last property is proved by induction using
Proposition~\ref{multMZ} (Coq checked).
\end{pf}

\begin{definition}\label{defn:multMM} {\multMM} is defined by:
\begin{eqnarray*}
\multMM : M \times M & \to & M \\
 i, m & \mapsto & m\multMZ i \\
 x\times m, m' & \mapsto & (m\multMM m')\multMV x
\end{eqnarray*}
\end{definition}

\begin{proposition}\label{multMM}
{\multMM} satisfies the following properties:
\begin{enumerate}[(i)]
\item $\|m\multMM m'\|=\|m\|\times\|m'\|$;
\item if $m,m'\neq 0$, then $|m\multMM m'|$ is the sorted
list obtained by merging $|m|$ with $|m'|$;
\item {\multMM} is well defined;
\item if $m\intrel t$ and $m'\intrel t'$, then
$m\multMM m'\intrel t\times t'$.
\end{enumerate}
\end{proposition}
\begin{pf}
The first property again follows directly from the definition of {\multMM}
and Propositions~\ref{multMZ} and~\ref{multMV}.
The second holds because {\multMM} simply implements straight insertion sort
on the list obtained by appending $|m|$ to $|m'|$.
From these two the third property follows, and the last one is again proved by
straightforward induction using Propositions~\ref{multMZ} and~\ref{multMV}
(Coq checked).
\end{pf}

The next function is of a different nature: it takes two monomials $m$
and $m'$ that coincide as lists (that is, $|m|=|m'|$) and returns the
monomial obtained by adding them.  Obviously this is only well defined
under the assumption that $|m|=|m'|$.

\begin{definition}\label{defn:plusMM} Let $\Delta_M$ denote the subset of
$M\times M$ defined by
\[\Delta_M=\{\langle m,m'\rangle\in M\times M \alt |m|=|m'|\}.\]
{\plusMM} is defined as follows.
\begin{eqnarray*}
\plusMM : \Delta_M & \to & M \\
 i, j & \mapsto & i+j \\
 x\times m, x\times m' & \mapsto & (m\plusMM m')\multMV x
\end{eqnarray*}
The structure of $\Delta_M$ ensures that this definition covers all cases.
\end{definition}

\begin{proposition}\label{plusMM}
{\plusMM} satisfies the following properties:
\begin{enumerate}[(i)]
\item $\|m\plusMM m'\|=\|m\|+\|m'\|$;
\item $m\plusMM m'=0$ if $\|m\|+\|m'\|=0$;
\item $|m\plusMM m'|=|m|=|m'|$ otherwise;
\item {\plusMM} is well defined;
\item if $m\intrel t$ and $m'\intrel t'$, then $m\plusMM m'\intrel t+t'$.
\end{enumerate}
\end{proposition}
\begin{pf}
The first condition is straightforward from the definition of {\plusMM}.
The second and third follow from this definition and
Proposition~\ref{multMV}; and from these the fourth is a direct consequence.
The last point is proved by induction using
Proposition~\ref{multMV} (Coq checked).
\end{pf}

In the sequence we will need the following notations.
We will denote by $<_M$ the lexicographic ordering on $\mathrm{list}(E)$
obtained from $<_E$.
Given two lists $l,w$ of expressions, we write $l\subseteq w$ to mean
that $l$ is a sublist of $w$, i.e.\ all elements of $l$ occur in $w$ and
in the same order.

\begin{definition}\label{defn:plusPM} {\plusPM} is defined as follows.
\begin{eqnarray*}
\plusPM : P\times M & \to & P \\
 i, j & \mapsto & i+j \\
 i, m & \mapsto & m+i \\
 m+p, j & \mapsto & m+(p\plusPM j) \\
 m+p, m' & \mapsto &
\left\{\begin{array}{ll} m+(p\plusPM m') & |m|<_M|m'| \\
 p\plusPM(m\plusMM m') & |m|=|m'| \\
 m'+m+p & \mbox{else}\end{array}\right.
\end{eqnarray*}
\end{definition}

\begin{proposition}\label{plusPM}
{\plusPM} satisfies the following properties:
\begin{enumerate}[(i)]
\item if $|m|=|m'|$, then $\coeff{m'}{p\plusPM m}=\coeff{m'}p+\|m\|$;
\item if $|m|\neq|m'|$, then $\coeff{m'}{p\plusPM m}=\coeff{m'}p$;
\item $|p\plusPM m|\subseteq l$, where $l$ is the list obtained by appending
$|m|$ to $|p|$ and sorting the result;
\item {\plusPM} is well defined;
\item if $p\intrel t$ and $m'\intrel t'$, then $p\plusPM m'\intrel t+t'$.
\end{enumerate}
\end{proposition}
\begin{pf}
The two first properties follow from the definition of {\plusPM} (in the
first case also appealing to Proposition~\ref{plusMM}).

The third property is proved by induction.  The basis is trivial; for
the induction step we need to consider two cases.  Let $p=m'+p'$; if
$|m|\neq|m'|$, then the algorithm reduces again to straight insertion
of an element in a list (since the only difference is in the case
$|m|=|m'|$).  If $|m|=|m'|$, then $|m'\plusMM m|=|m|$ by
Proposition~\ref{plusMM}, so we can use the induction hypothesis to
conclude that this call returns a $q$ such that $|q|$ is the straight
insertion of $|m'|$ in $|p'|$, which is $|m'|::|p'|$ (since $m'+p:P$),
and this is a sublist of $|m|::|m|::|p'|$, which would be the outcome of
the straight insertion of $|m|$ in $|m'|::|p'|$ (since $|m|=|m'|$).
Hence also in this case the thesis holds.

The fourth property is a consequence of the previous ones, since a sublist
of an ordered list is ordered.  The last property is proved by induction
(Coq checked).
\end{pf}

\begin{definition}\label{defn:plusPP} {\plusPP} is defined as follows.
\begin{eqnarray*}
\plusPP : P\times P & \to & P \\
 i, q & \mapsto & q\plusPM i \\
 m+p, q & \mapsto & (p\plusPP q)\plusPM m
\end{eqnarray*}
\end{definition}

\begin{proposition}\label{plusPP}
{\plusPP} satisfies the following properties:
\begin{enumerate}[(i)]
\item for all $m$, $\coeff m{p\plusPP q}=\coeff mp+\coeff mq$;
\item $|p\plusPP q|\subseteq l$, where $l$ is the list obtained by appending
$|q|$ to $|p|$ and sorting the result;
\item {\plusPP} is well defined;
\item if $p\intrel t$ and $q\intrel t'$, then $p\plusPP q\intrel t+t'$.
\end{enumerate}
\end{proposition}
\begin{pf}
The first property is proved by induction on $p$.  If $p=i$, then
either $m=j$ for some $j\in\Z$ and the thesis holds by the first part
of Proposition~\ref{plusPM} or else $|m|\neq|i|$ and the thesis holds
by the second part of Proposition~\ref{plusPM}.  If $p=m'+p'$, then
by induction hypothesis $\coeff m{p'\plusPP q}=\coeff m{p'}+\coeff mq$
and there are two cases.  If $|m'|=|m|$, then $\coeff mp=\|m'\|$ and
$\coeff m{p'}=0$ (since $|p|$ does not have repetitions), and by the
first part of Proposition~\ref{plusPM}
$\coeff m{(p'\plusPP q)\plusPM m'}%\
=\coeff m{p'\plusPP q}+\|m'\|%
=\coeff m{p'}+\coeff mq+\|m'\|%
=\coeff mq+\|m'\|%
=\coeff mq+\coeff mp$.
If $|m'|\neq|m|$ then $\coeff mp=\coeff m{p'}$ and by the second part
of Proposition~\ref{plusPM}
$\coeff m{(p'\plusPP q)\plusPM m'}%
=\coeff m{p'}+\coeff mq=%
\coeff mp+\coeff mq$.

The second and third properties are proved from
Proposition~\ref{plusPM} by straightforward induction.  The last
property is similar (Coq checked).
\end{pf}

The last operations have no analogue in sorting algorithms.
We will use juxtaposition to denote the sorted merge of two lists.

\begin{definition}\label{defn:multPM} {\multPM} is defined as follows.
\begin{eqnarray*}
\multPM : P\times M & \to & P \\
 i, m' & \mapsto & 0\plusPM (m'\multMZ i) \\
 m+p, m' & \mapsto & (p\multPM m')\plusPM (m\multMM m')
\end{eqnarray*}
\end{definition}

\begin{proposition}\label{multPM}
{\multPM} satisfies the following properties:
\begin{enumerate}[(i)]
\item for all $m$, $\coeff m{p\multPM m'}=\coeff{m^\ast}p\times\|m'\|$ if
there is
an $m^\ast$ such that $|m|=|m^\ast||m'|$ (there may exist at most one such $m^\ast$) and $0$ otherwise;
\item $p\multPM 0=0$;
\item if $m'\neq 0$, then $|p\multPM m'|$ is the sorted list whose
elements are obtained by appending $|m'|$ to each element of $p$ and
sorting the result;
\item {\multPM} is well defined;
\item if $p\intrel t$ and $m'\intrel t'$, then $p\multPM m'\intrel t\times t'$.
\end{enumerate}
\end{proposition}
\begin{pf}
The first property follows by induction on $p$ using
Propositions~\ref{plusPM} and~\ref{multMM} (since {\multMZ} is
a special case of {\multMM}).  The second property also follows
by induction, since $0\multMM0=0\plusPM 0=0$.

The third property is also proved by induction on $p$.  If $p=i$
then the result follows from Propositions~\ref{multMZ} and~\ref{plusPM}.
If $p=m+p'$, then $(m+p')\multPM m'=(p\multPM m')\plusPM (m\multMM m')$.
Since $|p'|$ does not have any repeated elements, by induction hypothesis
neither does $|p\multPM m'|$ (since its elements are the image of the
elements of $|p|$ via an injective function).  By Proposition~\ref{multMM},
$|m\multMM m'|$ is the sorted list whose elements are either in $|m|$ or in
$|m'|$, and this does not occur in $|p\multPM m'|$.  Hence the thesis
follows from Proposition~\ref{plusPM}.

The fourth property is straightforward since {\multMM} and {\plusPM} are
both well defined.  The last one is proved by induction on $p$
(Coq checked).
\end{pf}

\begin{definition}\label{defn:multPP} {\multPP} is defined as follows.
\begin{eqnarray*}
\multPP : P\times P & \to & P \\
 i, q & \mapsto & q\multPM i \\
 m+p, q & \mapsto & (q\multPM m)\plusPP (p\multPP q)
\end{eqnarray*}
\end{definition}

\begin{proposition}\label{multPP}
{\multPP} satisfies the following properties:
\begin{enumerate}[(i)]
\item for all $m\in M$, $\coeff m{p\multPP q}=\sum\coeff{m_1}p\coeff{m_2}q$,
where the sum ranges over all $m_1\in|p|\cup\{1\}$ and
$m_2\in|q|\cup\{1\}$ for which $|m|=|m_1||m_2|$;
\item {\multPP} is well defined;
\item if $p\intrel t$ and $q\intrel t'$, then $p\multPP q\intrel t\times t'$.
\end{enumerate}
\end{proposition}
\begin{pf}
We prove the first property by induction.  If $p=i$ then the result follows
from Proposition~\ref{multPM}, since then $m_1$ can only be $1$ ($|p|$ is the
empty list).
If $p=m'+p'$, then by Proposition~\ref{plusPP}
$\coeff m{(q\multPM m')\plusPP(p'\multPP q)}=%
\coeff m{q\multPM m'}+\coeff m{p'\multPP q}$; the result now follows from
induction hypothesis and Proposition~\ref{multPM}.
The second property is trivial; the last is proved by induction (Coq checked).
\end{pf}

\begin{definition}\label{defn:NormR}
The normalization function $\N$ is defined as follows, where $E^\ast$ denotes
the type of expressions that do not use division.
\begin{eqnarray*}
\N : E^\ast & \to & P \\
 i & \mapsto & i \\
 v^0_i & \mapsto & v^0_i\times 1+0 \\
 e+f & \mapsto & \N(e)\plusPP\N(f) \\
 e\times f& \mapsto & \N(e)\multPP\N(f) \\
 v^1_i(e) & \mapsto & v^1_i(\N(e))\times 1+0
\end{eqnarray*}
\end{definition}
\begin{proposition}\label{NormR} {\N} satisfies the following properties:
\begin{enumerate}[(i)]
\item {\N} is well defined;
\item\label{normpresint} if $e\intrel t$ then $\N(e)\intrel t$.
\end{enumerate}
\end{proposition}
\begin{pf}
Both properties are proved by induction, the first one using
Propositions~\ref{plusPP} and~\ref{multPP}
(the second one is Coq checked).
\end{pf}

\begin{corollary}\label{correctness}
Let $t,t':A$ and define $\langle e,\rho\rangle=\mlfn{t}{\emptyset}$ and
$\langle e',\sigma\rangle=\mlfn{t'}{\rho}$.  If $\N(e)=\N(e')$, then
$t=_A t'$ can be proved from the ring axioms and unfolding of
the definitions of $-$, {\zring} and {\nexp}.
\end{corollary}
\begin{pf}
Let $e$ and $e'$ be as defined above and suppose that $\N(e)=\N(e')$.
By Lemma~\ref{quotecorrect}, both $e\intrel t$ and $e'\intrel t'$.
By Proposition~\ref{NormR} also $\N(e)\intrel t$ and $\N(e')\intrel t'$.
Since $\N(e)=\N(e')$, we have that $\N(e)\intrel t$ and $\N(e)\intrel t'$,
whence $t=_A t'$ by Lemma~\ref{intrelfunction}.
\end{pf}

\subsection{Properties of $P$ and $\N$}\label{PNprops}

We now show that $\langle P,\plusPP,0,\multPP,1\rangle$ is a ring
(w.r.t.\ syntactic equality).
This will be essential later on, where we will use the properties
of these operations without comment.

\begin{lemma}\label{multMMcomm}
For all $m,m':M$, $m\multMM m'=m'\multMM m$.
\end{lemma}
\begin{pf}
By Lemma~\ref{monchar}, it is sufficient to show that 
$\|m\multMM m'\|=\|m'\multMM m\|$ and $|m\multMM m'|=|m'\multMM m|$.  But
both of these are consequences of Proposition~\ref{multMM}, commutativity
of addition and uniqueness of sort.
\end{pf}

\begin{lemma}\label{plusPPprops}
Let $p,q,r:P$.  Then the following hold:
\begin{enumerate}[(i)]
\item $p\plusPP0=p$
\item $p\plusPP (q\plusPP r)=(p\plusPP q)\plusPP r$
\item $p\plusPP q=q\plusPP p$
\item $p\plusPP (p\multPP(-1))=0$
\end{enumerate}
\end{lemma}
\begin{pf} Remembering that $p=q\iff \forall m:M.\coeff pm=\coeff qm$
(Proposition~\ref{polchar}), the first three properties are immediate.
For the fourth, given $m:M$,
$\coeff m{p\plusPP(p\multPP(-1))}=\coeff mp+\coeff m{p\multPP(-1)}$, so
it suffices to show that $\coeff m{p\multPP(-1)}=-\coeff mp$.
By Proposition~\ref{multPP},
$\coeff m{p\multPP(-1)}=\sum\coeff{m_1}p\coeff{m_2}{-1}$.
Now in this sum $m_2$ can only assume value $1$, whence $m_1=m$ and
the previous expression reduces to $\coeff mp(-1)=-\coeff mp$.
\end{pf}

\begin{lemma}\label{multPPprops}
Let $p,q,r:P$.  Then the following hold:
\begin{enumerate}[(i)]
\item $p\multPP0=0$
\item $p\multPP1=p$
\item $p\multPP (q\multPP r)=(p\multPP q)\multPP r$
\item $p\multPP q=q\multPP p$
\item $p\multPP(q\plusPP r)=(p\multPP q)\plusPP(p\multPP r)$
\end{enumerate}
\end{lemma}
\begin{pf} Again we appeal to Proposition~\ref{polchar}.

The first property is proved straightforwardly by induction using
Proposition~\ref{multPM}.

To prove $p\multPP1=1\multPP p=p\multPM1$ take any $m:M$; then
$|m|=|m||1|$, hence by Proposition~\ref{multPM}
$\coeff m{p\multPM1}=\coeff mp\times\|1\|=\coeff mp$, hence $p\multPP1=p$.

To prove commutativity, again take any $m:M$; then
$\coeff m{p\multPP q}=%
\sum\coeff{m_1}p\coeff{m_2}q=%
\sum\coeff{m_2}q\coeff{m_1}p=%
\coeff m{q\multPP p}$
where the sums range over all $m_1\in|p|\cup\{1\}$ and $m_2\in|q|\cup\{1\}$
for which $|m|=|m_1||m_2|$; the equalities hold by Proposition~\ref{multPP}.

For associativity, we again take an arbitrary $m:M$ and conclude from
Proposition~\ref{multPP} that
$\coeff m{p\multPP(q\multPP r)}=%
\sum\coeff{m_1}p\coeff{m_2}{q\multPP r}=%
\sum\coeff{m_1}p\left(\sum\coeff{m_2^1}q\coeff{m_2^2}r\right)=%
\sum\coeff{m_1}p\coeff{m_2^1}q\coeff{m_2^2}r$.
This last expression is completely symmetric on $p$, $q$ and $r$, since the
last sum in fact ranges over all $m_1$, $m_2^1$ and $m_2^2$ such that
$|m|=|m_1||m_2^1||m_2^2|$ with $m_1\in|p|\cup\{1\}$, $m_2^1\in|q|\cup\{1\}$
and $m_2^2\in|r|\cup\{1\}$.
Therefore, from associativity and commutativity of sums and products of
integers, we immediately get that
$\coeff m{p\multPP(q\multPP r)}=\coeff m{r\multPP(p\multPP q)}$, and
applying commutativity of {\multPP} twice we get the desired result.

For the last property, again take $m:M$; then
$\coeff m{p\multPP(q\plusPP r)}=%
\sum\coeff{m_1}p\coeff{m_2}{q\plusPP r}=%
\sum\coeff{m_1}p(\coeff{m_2}q+\coeff{m_2}r)=%
\sum\coeff{m_1}p\coeff{m_2}q+\sum\coeff{m_1}p\coeff{m_2}r=%
\coeff m{(p\multPP q)\plusPP(p\multPP r)}$
\end{pf}

\begin{lemma}\label{NormRmon} Let $m:M\setminus\Z$ and $p:P$.
Then $\N(m)=m+0$ and $\N(p)=p$.
\end{lemma}
\begin{pf}
By simultaneous induction on $m$ and $p$.

The case $m=v^0_i\times i$ is proved by computation; also
$m=v^1_i(p)\times i$ follows from computation and the induction
hypothesis for $p$.

If $m=v^0_i\times m'$, then notice first that $m'\multMM(v^0_i\times 1)=m$:
$\|m'\multMM(v^0_i\times1)\|=\|m'\|\times1=\|m'\|$ by Proposition~\ref{multMM}
and $|m'\multMM(v^0_i\times 1)|$ is the list obtained by inserting $v^0_i$ at
the right position in $|m'|$, which is by definition $|m|$ (since this list is
sorted), hence by Lemma~\ref{monchar} the result holds.
Using this fact, the thesis can be seen to hold by computing $\N(m)$,
applying the induction hypothesis and Lemmas~\ref{multPPprops}
and~\ref{plusPPprops}.
The case $m=v^1_i(p)\times m'$ is similar.

Now suppose that $p$ is an integer; then $\N(p)=p$ by definition.
Else take $p=m+q$; by induction hypothesis $\N(q)=q$ and $\N(m)=m+0$, hence
$\N(m+q)=q\plusPM m$ by computation and Lemma~\ref{plusPPprops}; but by
definition of $P$, $|m|$ cannot occur in $|q|$ and must be smaller than $|q|$
(w.r.t.\ $<_M$), hence the last expression reduces to $m+q$, or $p$.
\end{pf}

\begin{corollary}\label{NormRidemp} $\N$ is idempotent: for every
$e:E^\ast$, $\N(\N(e))=\N(e)$.
\end{corollary}
\begin{pf}
Since $\N(e):P$, the previous lemma yields the result.
\end{pf}

\subsection{The substitution lemma}

In this subsection, we show that the following ``substitution lemma''
holds: if, in two expressions that normalize to the same, some variables
get uniformly renamed, then the resulting expressions also normalize to
the same term.
This is proven in two steps.
\begin{lemma}\label{substlemmaaux} Let $e:E$ and $\xi$ be a renaming of
variables.  Then $\N(\renamevar e\xi)=\N(\renamevar{\N(e)}\xi)$.
\end{lemma}
\begin{pf}
By induction on $e$.

Suppose $e=i$; then
$\N(\renamevar{\N(i)}\xi)%
=\N(\renamevar i\xi)%
=\N(i)%
=\N(\renamevar i\xi)$.

Now let $e=v^0_i$.  Then
$\N(\renamevar{\N(e)}\xi)%
=\N(\renamevar{(v^0_i\times1+0)}\xi)%
=\N(v^0_{\xi_0(i)}\times1+0)%
=\N(v^0_{\xi_0(i)})\multPP1\plusPP0%
=\N(v^0_{\xi_0(i)}))%
=\N(\renamevar{(v^0_i)}\xi)$
by virtue of Propositions~\ref{plusPPprops} and~\ref{multPPprops}.

If $e=v^1_i(e')$, then we use the induction hypothesis to show that
\begin{eqnarray*}
\N(\renamevar{\N(v^1_i(e'))}\xi)
 & = & \N(\renamevar{(v^1_i(\N(e'))\times1+0)}\xi)\\
 & = & \N(v^1_{\xi_1(i)}(\renamevar{\N(e')}\xi)\times1+0) \\
 & = & \N(v^1_{\xi_1(i)}(\renamevar{\N(e')}\xi))\multPP1\plusPP0 \\
 & = & \N(v^1_{\xi_1(i)}(\renamevar{\N(e')}\xi)) \\
 & = & v^1_{\xi_1(i)}(\N(\renamevar{\N(e')}\xi))\times1+0 \\
 & \stackrel{\mathrm{IH}}= & v^1_{\xi_1(i)}(\N(\renamevar{e'}\xi))\times1+0 \\
 & = & \N(v^1_{\xi_1(i)}(\renamevar{e'}\xi)) \\
 & = & \N(\renamevar{(v^1_i(e'))}\xi)
\end{eqnarray*}

For the case $e=e_1\star e_2$, with $\star=+,\times$, we also need
the equality
$\N(\renamevar{(p\star q)}\xi)=\N(\renamevar{(p\starPP q)}\xi)$
for all $p,q:P$.  The proof is included in the Appendix
of~\cite{lcf:wie:04b}; its use is marked here by $\ast$.
\begin{eqnarray*}
\N(\renamevar{\N(e_1\star e_2)}\xi)
 & = & \N(\renamevar{(\N(e_1)\starPP\N(e_2))}\xi) \\
 & \stackrel\ast= & \N(\renamevar{(\N(e_1)\star\N(e_2))}\xi) \\
 & = & \N(\renamevar{\N(e_1)}\xi\star\renamevar{\N(e_2)}\xi) \\
 & = & \N(\renamevar{\N(e_1)}\xi)\starPP\N(\renamevar{\N(e_2)}\xi) \\
 & \stackrel{\mathrm{IH}}= & \N(\renamevar{e_1}\xi)\starPP\N(\renamevar{e_2}\xi) \\
 & = & \N(\renamevar{e_1}\xi\star\renamevar{e_2}\xi) \\
 & = & \N(\renamevar{(e_1\star e_2)}\xi)
\end{eqnarray*}
\end{pf}

\begin{theorem}\label{substlemma}
Let $e,e':E$ be expressions such that $\N(e)=\N(e')$ and let $\xi$ be
a renaming of variables.  Then $\N(\renamevar e\xi)=\N(\renamevar{e'}\xi)$.
\end{theorem}
\begin{pf}
By Lemma~\ref{substlemmaaux},
$\N(\renamevar e\xi)%
=\N(\renamevar{N(e)}\xi)%
=\N(\renamevar{N(e')}\xi)%
=\N(\renamevar{e'}\xi)$.
\end{pf}

\subsection{Completeness}

We are now ready to state our main result.

\begin{theorem}\label{completeness}
Let $t,t':A$ be such that the equality $t=_A t'$ can be proved (in the
sense of Definition~\ref{defn:proof}) only from the ring axioms and
unfolding of the definitions of $-$, $\zring$ and $\nexp$ in $t$ and
$t'$.  Define $\langle e,\rho\rangle=\mlfn{t}{\emptyset}$ and
$\langle e',\sigma\rangle=\mlfn{t'}{\rho}$.  Then $\N(e)=\N(e')$.
\end{theorem}

For presentation, we split the proof of this in several stages.

\begin{lemma}\label{unfolding}
Let $t,t':A$ be terms such that $t'$ is obtained from $t$ by unfolding
the definitions of $-$, $\zring$ and $\cdot^n$ ($n$ closed) in $t$.
If $\mlfn{t}{\rho}=\langle e,\sigma\rangle$ and
$\mlfn{t'}{\rho}=\langle e',\sigma'\rangle$, then $\sigma=\sigma'$ and
$\N(e)=\N(e')$.
\end{lemma}
\begin{pf}
For $-$ and $\cdot^n$ this is immediate,
since terms using these constructors are quoted to expressions using the
corresponding abbreviations whose definition coincides with those of $-$
and $\cdot^n$.

For $\zring$ the proof is by induction\footnote{In this paragraph we
write $+_\Z$ to emphasize the distinction between addition of integers and
addition of expressions.}: $\underline0$ unfolds
to $0$, both of which are quoted to $0$; $\underline{n+1}$ is quoted to
$n+_\Z1$, which is in normal form, whereas $\underline{n}+1$ is quoted
to $n+1$ which normalizes to $\N(n)\plusPP1=n+_\Z1$; finally,
$\underline{n-1}$ is quoted to $n+1\times(-1)$, which normalizes to
$\N(n)\plusPP1\multPP(-1)=n-_\Z1$.
\end{pf}

\begin{lemma}\label{axioms}
Let $t,t':A$ be such that $t=_A t'$ is an instance of one of the axioms
\axiom{Set_1}, \axiom{SG}, \axiom{M_1}, \axiom{M_2}, \axiom{G_1}, \axiom{G_2},
\axiom{AG} or \axiom{R_i} with $1\leq i\leq5$.
Define
$\langle e,\tau\rangle=\mlfn{t}{\emptyset}$ and
$\langle e',\tau'\rangle=\mlfn{t'}{\tau}$.  Then $\N(e)=\N(e')$.
\end{lemma}
\begin{pf}
All these proofs are very similar, being a consequence of
Lemmas~\ref{plusPPprops} and~\ref{multPPprops}.  We detail a few.
\begin{description}
\item[\axiom{Set_1}] Then $t=t'$; by Lemma~\ref{quoteidempotent}
$e'=e$, and obviously $\N(e)=\N(e)$.
\item[\axiom{SG}] Then $t=(t_1+t_2)+t_3$ and $t'=t_1+(t_2+t_3)$.
Let $\mlfn{t_1}{\emptyset}=\langle e_1,\rho\rangle$,
$\mlfn{t_2}{\rho}=\langle e_2,\sigma\rangle$ and
$\mlfn{t_3}{\sigma}=\langle e_3,\theta\rangle$.  Then
$\mlfn{t_1+t_2}{\emptyset}=\langle e_1+e_2,\sigma\rangle$ and
$\mlfn{(t_1+t_2)+t_3}{\emptyset}=\langle (e_1+e_2)+e_3,\theta\rangle$.

Furthermore, since $\rho\subseteq\sigma\subseteq\theta$ by
Lemma~\ref{quoteincl}, Lemma~\ref{quoteidempotent} yields
$\mlfn{t_1}{\theta}=\langle e_1,\theta\rangle$,
$\mlfn{t_2}{\theta}=\langle e_2,\theta\rangle$ and
$\mlfn{t_3}{\sigma}=\langle e_3,\theta\rangle$, and therefore
$\mlfn{t_2+t_3}{\theta}=\langle e_2+e_3,\theta\rangle$ and
$\mlfn{t_1+(t_2+t_3)}{\theta}=\langle e_1+(e_2+e_3),\theta\rangle$.

Then
$\N((e_1+e_2)+e_3)%
=\N(e_1+e_2)\plusPP\N(e_3)%
=(\N(e_1)\plusPP\N(e_2))\plusPP\N(e_3)%
=\N(e_1)\plusPP(\N(e_2)\plusPP\N(e_3))%
=\N(e_1)\plusPP\N(e_2+e_3)%
=\N(e_1+(e_2+e_3))$.
\item[\axiom{G_1}] Then $t=t_1+(-t_1)$ and $t'=0$.  Let
$\mlfn{t_1}{\emptyset}=\langle e_1,\rho\rangle$; then by
Lemma~\ref{quoteidempotent} also $\mlfn{t_1}{\rho}=\langle e_1,\rho\rangle$,
hence $e=e_1+(e_1\times(-1))$; by definition
$\mlfn{0}{\rho}=\langle 0,\rho\rangle$, so $e'=0$.

Now
$\N(e_1+(e_1\times(-1)))%
=\N(e_1)\plusPP(\N(e_1)\multPP(-1))%
=0%
=\N(0)$,
according to Lemma~\ref{plusPPprops}.
\item[\axiom{R_5}] In this case $t=t_1\times(t_2+t_3)$ and
$t'=(t_1\times t_2)+(t_1\times t_3)$.  Reasoning like in
the case of \axiom{SG} above, we conclude that $e=e_1\times(e_2+e_3)$ and
$e'=(e_1\times e_2)+(e_1\times e_3)$.  By Lemma~\ref{multPPprops},
$\N(e_1\times(e_2+e_3))%
=\N(e_1)\multPP(\N(e_2)\plusPP\N(e_3))%
=\N(e_1)\multPP\N(e_2)\plusPP\N(e_1)\multPP\N(e_3)%
=\N((e_1\times e_2)+(e_1\times e_3))$.
\end{description}
\end{pf}

\begin{lemma}\label{set2}
Let $t_1,t_2:A$ be such that, if $\langle e_1,\rho\rangle=\mlfn{t_1}{\emptyset}$
and $\langle e_2,\sigma\rangle=\mlfn{t_2}{\rho}$, then $\N(e_1)=\N(e_2)$.
Define $\langle e'_2,\sigma'\rangle=\mlfn{t_2}{\emptyset}$
and $\langle e'_1,\rho'\rangle=\mlfn{t_1}{\sigma'}$.  Then $\N(e'_1)=\N(e'_2)$.
\end{lemma}
\begin{pf}
Let $e_1,e'_1,e_2$ and $e'_2$ be as given.  By
Lemma~\ref{quotecommutes} there is a renaming of variables $\xi$ such
that {\isrenamevar{e'_i}{e_i}\xi} for $i=1,2$; but then
$\N(e'_1)=\N\left(\renamevar{e_1}\xi\right)=%
\N\left(\renamevar{e_2}\xi\right)=\N(e'_2)$ using the hypothesis
$\N(e_1)=\N(e_2)$ and Theorem~\ref{substlemma}.
\end{pf}

\begin{lemma}\label{set3}
Let $t_1,t_2,t_3:A$ and define
\[\begin{array}{ccc}
\langle e_1,\rho\rangle=\mlfn{t_1}{\emptyset}
 & & \langle e'_2,\sigma'\rangle=\mlfn{t_2}{\emptyset} \\
\langle e_2,\sigma\rangle=\mlfn{t_2}{\rho}
 & & \langle e'_3,\theta'\rangle=\mlfn{t_3}{\sigma'}
\end{array}\]
Assume that $\N(e_1)=\N(e_2)$ and $\N(e'_2)=\N(e'_3)$.  Define
$\langle e_3,\theta\rangle=\mlfn{t_3}{\rho}$.  Then $\N(e_1)=\N(e_3)$.
\end{lemma}
\begin{pf}
Let $e_1,e_2,e'_2,e_3$ and $e'_3$ be as given and define
$\langle e''_3,\theta''\rangle=\mlfn{t_3}{\sigma}$.
By Lemma~\ref{quoteaddpred},
there exists a renaming of variables $\xi$ such that
$\isrenamevar{e''_3}{e'_3}\xi$ and $\isrenamevar{e_2}{e'_2}\xi$.
By Lemma~\ref{quotecommutes} there is another renaming of variables $\xi'$
such that $\isrenamevar{e_3}{e''_3}{\xi'}$ and
$\domain(\xi'_i)\cap\domain(\rho_i)=\emptyset$. %(see Figure~\ref{fig:set3})
%\begin{figure}[htb]
%\[\xymatrix{
% & \emptyset \ar[dl]_{t_2} \ar[dr]^{t_1} \\
% \sigma' \ar[d]_{t_3}\ar@{.>}[dr]^{\xi} & & \rho \ar[dl]_{t_2} \ar[d]_{t_3} \\
% \theta'\ar@{.>}[dr]^{\xi} & \sigma \ar[d]_{t_3} & \theta \\
% & \theta''\ar@{.>}[ur]^{\xi'}
%}\]
%\caption{Proof of Lemma~\ref{set3}}\label{fig:set3}
%\end{figure}
Then
$\N(e_3)%
=\N\left(\renamevar{e''_3}{\xi'}\right)%
=\N\left(\renamevar{e'_3}{\xi'\circ\xi}\right)%
=\N\left(\renamevar{e'_2}{\xi'\circ\xi}\right)%
=\N\left(\renamevar{e_2}{\xi'}\right)%
=\N\left(\renamevar{e_1}{\xi'}\right)%
=\N(e_1)$
using the hypotheses $\N(e_1)=\N(e_2)$ and $\N(e'_2)=\N(e'_3)$ together with 
Theorem~\ref{substlemma} and the equalities above stated.  The last equality
follows from the fact that $\domain(\xi'_i)\cap\domain(\rho_i)=\emptyset$: by
Lemma~\ref{quotedom} every variable $v^i_k$ occurring in $e_1$ is in
$\domain(\rho_i)$, so {\isrenamevar{e_1}{e_1}{\xi'}}.
\end{pf}

\begin{lemma}\label{set4}
Let $t_1,t_2:A$ be such that, if $\langle e_1,\rho\rangle=\mlfn{t_1}{\emptyset}$
and $\langle e_2,\sigma\rangle=\mlfn{t_2}{\rho}$, then $\N(e_1)=\N(e_2)$.
Let $f:[A\to A]$ be other than $\cdot^n$ with $n$ closed and define
$\langle e'_1,\rho'\rangle=\mlfn{f(t_1)}{\emptyset}$ and
$\langle e'_2,\sigma'\rangle=\mlfn{f(t_2)}{\emptyset}$.
Then $\N(e'_1)=\N(e'_2)$.
\end{lemma}
\begin{pf}
We have to consider two cases.  If $f$ is the unary inverse ($-$), then
immediately $e'_1=-e_1$, $\rho'=\rho$ and hence $e'_2=-e_2$; in this case,
$\N(e'_1)=\N(-e_1)=\N(e_1\times(-1))=\N(e_1)\multPP(-1)=\N(e_2)\multPP(-1)=%
\N(e_2\times(-1))=\N(-e_2)=\N(e'_2)$.

Else, $e'_1=v^1_i(e_1)$ with $\mlfnv{f}{\rho}=\langle v^1_i,\rho'\rangle$
and $e'_2=v^1_i(e''_2)$, with $\mlfn{t_2}{\rho'}=\langle e''_2,\sigma'\rangle$
(since by Lemma~\ref{quoteincl} $\rho'\subseteq\sigma'$, $\sigma'_1(v^1_i)=f$
and thus $\mlfnv{f}{\sigma'}=\langle v^1_i,\sigma\rangle$).
By Lemma~\ref{quotecommv1} there is a renaming of variables $\xi$ such
that {\isrenamevar{e''_2}{e_2}\xi} and $\domain(\xi_i)\cap\domain(\rho_i)=\emptyset$.
Hence
$\N(e'_2)%
=\N(v^1_i(e''_2))%
=v^1_i(\N(e''_2))\times1+0%
=v^1_i(\N(\renamevar{e_2}\xi))\times1+0%
=v^1_i(\N(\renamevar{e_1}\xi))\times1+0%
=v^1_i(\N(e_1))\times1+0%
=\N(v^1_i(e_1))%
=\N(e'_1)$,
using Theorem~\ref{substlemma} together with the assumption
$\N(e_1)=\N(e_2)$ and the fact that {\isrenamevar{e_1}{e_1}\xi} by
virtue of Lemma~\ref{quotedom} and
$\domain(\xi_i)\cap\domain(\rho_i)=\emptyset$.
\end{pf}

\begin{lemma}\label{set5}
Let $t_1,t_2,t_3,t_4:A$ and define
\[\begin{array}{ccc}
\langle e_1,\rho\rangle=\mlfn{t_1}{\emptyset}
 & & \langle e_3,\theta\rangle=\mlfn{t_3}{\emptyset} \\
\langle e_2,\sigma\rangle=\mlfn{t_2}{\rho}
 & & \langle e_4,\tau\rangle=\mlfn{t_4}{\theta}
\end{array}\]
Assume that $\N(e_1)=\N(e_2)$ and $\N(e_3)=\N(e_4)$ and let
\[\begin{array}{ccc}
\langle e,\gamma\rangle=\mlfn{t_1\star t_3}{\emptyset}
 & & \langle e',\gamma'\rangle=\mlfn{t_2\star t_4}{\gamma}
\end{array}\]
where $\star$ is $+$ or $\times$.  Then $\N(e)=\N(e')$.
\end{lemma}
\begin{pf}
By definition of quote, $e=e_1\star e'_3$ with
$\langle e'_3,\gamma\rangle=\mlfn{t_3}{\rho}$.
Also, $e'=e'_2\star e'_4$, with
$\langle e'_2,\gamma''\rangle=\mlfn{t_2}{\gamma}$
and $\langle e'_4,\gamma'\rangle=\mlfn{t_4}{\gamma''}$.


Take $\mlfn{t_4}{\gamma}=\langle e''_4,\gamma'''\rangle$.

According to Lemma~\ref{quoteaddpred}, there exists a renaming of variables
$\xi$ such that {\isrenamevar{e'_3}{e_3}\xi} and
{\isrenamevar{e''_4}{e_4}\xi}.  By Lemma~\ref{quotecommutes}, there
is a renaming of variables $\xi'$ such that {\isrenamevar{e'_2}{e_2}{\xi'}}
and $\domain(\xi'_i)\cap\domain(\rho_i)=\emptyset$ (and hence
{\isrenamevar{e_1}{e_1}{\xi'}} due to Lemma~\ref{quotedom}).
Again by Lemma~\ref{quotecommutes}, there exists a renaming of
variables $\xi''$ such that {\isrenamevar{e'_4}{e''_4}{\xi''}}
and $\domain(\xi''_i)\cap\domain(\gamma_i)=\emptyset$ (so that
{\isrenamevar{e'_3}{e'_3}{\xi''}}). %(see Figure~\ref{fig:set5})
%\begin{figure}
%\[\xymatrix{
% & & \emptyset \ar[dl]_{t_1} \ar[dr]^{t_3} \\
% & \rho \ar[dl]_{t_2} \ar[dr]^{t_3}
% & & \theta \ar[dr]^{t_4} \ar@{.>}[dl]_{\xi} \\
% \sigma \ar@{.>}[dr]^{\xi'}
% & & \gamma \ar[dl]_{t_2} \ar[dr]^{t_4}
% & & \tau \ar@{.>}[dl]_{\xi} \\
% & \gamma'' \ar[dr]^{t^4}
% & & \gamma''' \ar@{.>}[dl]_{\xi''} \\
% & & \gamma'
%}\]
%\caption{Proof of Lemma~\ref{set5}}\label{fig:set5}
%\end{figure}

Then
$\N(e')%
=\N(e'_2\star e'_4)%
=\N(e'_2)\starPP\N(e'_4)%
=\N(\renamevar{e_2}{\xi'})\starPP\N(\renamevar{e''_4}{\xi''})%
=\N(\renamevar{e_2}{\xi'})\starPP\N(\renamevar{e_4}{\xi''\circ\xi})%
=\N(\renamevar{e_1}{\xi'})\starPP\N(\renamevar{e_3}{\xi''\circ\xi})%
=\N(e_1)\starPP\N(\renamevar{e'_3}{\xi''})=\N(e_1)\starPP\N(e'_3)%
=\N(e_1\star e'_3)%
=\N(e)$
using the hypotheses and Theorem~\ref{substlemma}.
\end{pf}

\begin{definition}\label{defn:normalproof} A \emph{normal proof} of $t=_A t'$
is a proof of $t=_A t'$ where \axiom{Set_4} is not applied with
$\cdot^n$ ($n$ closed) and \axiom{Set_5} is not applied with $-$.
\end{definition}

\begin{lemma}\label{normalproof} Suppose that $t=_A t'$ can be proved
only from the ring axioms and unfolding of the definitions of $-$,
$\zring$ and $\nexp$ in $t$ and $t'$.  Then there exists a normal proof
of $t=_A t'$.
\end{lemma}
\begin{pf}
By induction on the length of the proof of $t=_A t'$.  The only non-trivial
cases are those when the last axiom to be applied is \axiom{Set_4} with
$\cdot^n$ ($n$ closed) or \axiom{Set_5} with $-$.

Suppose we prove $t_1^n=_A t_2^n$ from $t_1=_A t_2$ using
\axiom{Set_4}.  We proceed by induction.  If $n=0$, then we can
replace the whole proof by $(\axiom{Set_1}\ 1)$, and folding produces
$t_1^0=_A t_2^0$.  If $n=k+1$, we first find a normal proof of
$t_1^k=_A t_2^k$ using the induction hypothesis (for $n$) and a normal
proof of $t_1=_A t_2$ using the induction hypothesis for the lemma.
Then we apply \axiom{Set_5} to get $t_1\times t_1^k=_A t_2\times
t_2^k$; folding $\cdot^{k+1}$ on the last equality produces the
desired proof.

Finally, if we prove $t_1-t_3=_A t_2-t_4$ from $t_1=_A t_2$ and
$t_3=_A t_4$ using \axiom{Set_5}, we first find normal proofs of
$t_1=_A t_2$ and $t_3=_A t_4$ (induction hypothesis), apply \axiom{Set_4}
to the latter to get $-t_3=_A -t_4$ and apply \axiom{Set_5} to get
$t_1+(-t_3)=_A t_2+(-t_4)$, which is the desired equality with the
definition of $-$ unfolded.
\end{pf}

\begin{theorem}\label{completenormal} 
Let $t,t':A$ and $e,e':E$ be as in Theorem~\ref{completeness} and
assume that there is a normal proof of $t=_A t'$.
Then $\N(e)=\N(e')$.
\end{theorem}
\begin{pf}
By induction on the length of the normal proof of $t=_A t'$.

If $t=_A t'$ is an instance of one of the axioms \axiom{Set_1},
\axiom{SG}, \axiom{M_1}, \axiom{M_2}, \axiom{G_1}, \axiom{G_2},
\axiom{AG} or \axiom{R_i} with $1\leq i\leq5$, then by Lemma~\ref{axioms}
$\N(e)=\N(e')$.

If $t=_A t'$ is proved by \axiom{Set_2} from $t'=_A t$, then the
thesis holds by Lemma~\ref{set2} and the induction hypothesis.

If $t=_A t'$ is proved by \axiom{Set_3} from $t=_A t_1$ and
$t_1=_A t'$, then the thesis holds by Lemma~\ref{set3} and the induction
hypothesis.

If $t=_A t'$ is proved by \axiom{Set_4} from $t_1=_A t_2$ and
$f$ is not $\cdot^n$ with $n$ closed, then the thesis holds by
Lemma~\ref{set4} and induction hypothesis.

If $t=_A t'$ is proved by \axiom{Set_5} from $t_1=_A t_2$ and $t_3=_A t_4$
and $f$ is not $-$, then the thesis holds by
Lemma~\ref{set5} and the induction hypothesis.

If $t_1$ and $t_2$ can be obtained from $t$ and $t'$ by unfolding the
definitions of $-$, $\cdot^n$ and $\zring$, then by Lemma~\ref{unfolding}
$\mlfn{t_1}{\emptyset}=\mlfn{t}{\emptyset}=\langle e,\rho\rangle$
and $\mlfn{t_2}{\rho}=\mlfn{t'}{\rho}=\langle e',\sigma\rangle$.
The induction hypothesis asserts the thesis.
\end{pf}

We are now ready to prove Theorem~\ref{completeness}.

\begin{pf*}{Proof (Completeness Theorem~\ref{completeness})}
Assume there is a proof of $t=_A t'$.  By Lemma~\ref{normalproof} there is
also a normal proof of $t=_A t'$, so by Theorem~\ref{completenormal}
$\N(e)=\N(e')$.
\end{pf*}

\section{Completeness of {\rational}: groups}\label{groups}

We now prove a completeness theorem for groups similar
to Theorem~\ref{completeness}.
The theory developed above is not enough as is:
if $G$ is a group, $a:G$ and
$v^0_0\intrel a$, then $v^0_0+v^0_0\intrel a+a$, but
$\N(v^0_0+v^0_0)=v^0_0\times 2+0$, which cannot be interpreted in $G$,
so part~(\ref{normpresint}) of Lemma~\ref{NormR} fails to hold.
Hence we first extend the interpretation relation conservatively.

\begin{definition}\label{defn:iter} Let $G$ be a group, $n:\Z$ and $a:G$.
Then {\iter na} is inductively defined as follows.
\begin{eqnarray}
\label{iter:0}
 \iter0a & := & 0\\
\label{iter:pos}
 \iter{(n+1)}a & := & \iter na+a, \mbox{ for $n\geq 0$}\\
\label{iter:neg}
 \iter{(n-1)}a & := & \iter na-a, \mbox{ for $n\leq 0$}
\end{eqnarray}
\end{definition}

\begin{proposition} Let $R$ be a ring.  Then, for all $n:\Z$ and $a:R$,
$\iter na=_R \underline n\times a$ is provable.
\end{proposition}
\begin{pf}
Straightforward induction.
\end{pf}

\begin{lemma}\label{intrel:iterprop} Let $\rho$ be a valuation pair for
a ring $A$.  The interpretation relation satisfies the following rule,
where $n:\Z$.
\begin{eqnarray*}
e\intrel t_1 \wedge \iter n{t_1}=_A t & \rightarrow & e\times n\intrel t.
\end{eqnarray*}
\end{lemma}
\begin{pf} By the previous proposition
$\iter n{t_1}=_A \underline n\times t_1$ is provable, whence
$\underline n\times t_1=_A t$ is provable by hypothesis, \axiom{Set_2} and
\axiom{Set_3}.  Furthermore, $\underline n\intrel n$ by \axiom{Set_1}
and~(\ref{intrel:int}).  By hypothesis $e\intrel t_1$.  Therefore,
by~(\ref{intrel:mult}), $e\times n\intrel t$.
\end{pf}

Hence, this clause can be added to the inductive definition of
the interpretation relation without changing it when defined over a
ring or field but extending it in the case of groups.
We also need the case $k=0$ of~(\ref{intrel:int}).  That is, we
consider the interpretation relation as defined in
Definition~\ref{defn:intrel} extended with the two following clauses.
\begin{eqnarray}
\label{intrel:zero}
0=_G t & \rightarrow & 0\intrel t\\
\label{intrel:iter}
e\intrel t_1 \wedge \iter n{t_1}=_G t & \rightarrow & e\times n\intrel t
\end{eqnarray}
Notice that conditions~(\ref{intrel:inv}) and~(\ref{intrel:minus}) in
Lemma~(\ref{intrel:abbr}) can be proved from these clauses, so that
they also hold for groups with this extended interpretation relation.
The following results are then easily proved by induction (Coq checked);
they are analogues of Lemma~\ref{intrelfunction} and the lemmas of
Subsection~\ref{normfn}.

\begin{lemma}\label{intrelfunction:group} Let $e:E$, $t,t':G$ and $\rho$ be
a valuation pair for $G$ such that $e\intrel t$ and $e\intrel t'$.
Then $t=_G t'$.
\end{lemma}
%\begin{pf} By induction on $\intrel$ (Coq checked).
%\end{pf}

\begin{lemma}\label{groupprops}
Let $G$ be a group and $\rho$ be a valuation pair for $G$.
The auxiliary normalization functions satisfy the following properties.
\begin{enumerate}[(i)]
\item if $m\intrel t$ then $m\multMZ i\intrel\iter it$;
\item if $x\times m\intrel t$ then $m\multMV x\intrel t$;
\item if $m\times m'\intrel t$ or $m'\times m\intrel t$ then
$m\multMM m'\intrel t$;
\item if $m\intrel t$ and $m'\intrel t'$ then $m\plusMM m'\intrel t+t'$;
\item if $p\intrel t$ and $m'\intrel t'$ then $p\plusPM m'\intrel t+t'$;
\item if $p\intrel t$ and $p'\intrel t'$ then $p\plusPP p'\intrel t+t'$;
\item if $p\times m'\intrel t$ or $m'\times p\intrel t$ then
$p\multPM m'\intrel t$;
\item if $p\times p'\intrel t$ then $p\multPP p'\intrel t$.
\end{enumerate}
\end{lemma}
%\begin{pf} By induction (Coq checked).
%\end{pf}

Some of the hypotheses in the previous lemma may seem a bit strange.  The
problem is, we cannot say as before that ``if $m\intrel t$ and $m'\intrel t'$
then $m\multMM m'\intrel t\times t'$'' because in $G$ there is no
multiplication.  Hence, we replace this by the equivalent (in a ring) form
``if $m\times m'\intrel t$ then $m\multMM m'\intrel t$''.  However, this
is still not enough, since {\multMM} may switch the order of its arguments;
hence the disjunction in the actual lemma, which in fact says that one of
the arguments to {\multMM} is an integer.

Similar remarks hold for {\multPM}.  In the case of {\multMV} we already
know that the second argument is a variable, so one of the clauses of the
disjunction never holds and we can erase it.  As for {\multPP}, it will only
be called by $\N$ when a product appears in the original expression, which
is clearly impossible if this is the result of quoting a term in $G$;
it is however needed in the proof of the following lemma.

\begin{lemma}\label{NormG} Let $e:E$ and $t:G$.  If $e\intrel t$ then
$\N(e)\intrel t$.
\end{lemma}
\begin{pf}
By induction (Coq checked).  Since products in expressions can now only be
interpreted by means of~(\ref{intrel:iter}), the stronger hypotheses
in Lemma~\ref{groupprops} are seen to be satisfied by analyzing the proof
of $e\intrel t$.
\end{pf}

\begin{corollary}\label{correctness:group}
Let $t,t':G$ and define $\langle e,\rho\rangle=\mlfn{t}{\emptyset}$ and
$\langle e',\sigma\rangle=\mlfn{t'}{\rho}$.  If $\N(e)=\N(e')$, then
$t=_G t'$ can be proved from the group axioms and unfolding of
the definition of $-$.
\end{corollary}
\begin{pf}
Let $e$ and $e'$ be as defined above and suppose that $\N(e)=\N(e')$.
By Lemma~\ref{quotecorrect}, both $e\intrel t$ and $e'\intrel t'$.
By Proposition~\ref{NormG} also $\N(e)\intrel t$ and $\N(e')\intrel t'$.
Since $\N(e)=\N(e')$, we have that $\N(e)\intrel t$ and $\N(e)\intrel t'$,
whence $t=_G t'$ by Lemma~\ref{intrelfunction:group}.
\end{pf}

\begin{theorem}\label{completeness:group}
Let $t,t':G$ be such that the equality $t=_G t'$ can be proved
only from the group axioms and unfolding of the definition of $-$.  Define
$\langle e,\rho\rangle=\mlfn{t}{\emptyset}$ and
$\langle e',\sigma\rangle=\mlfn{t'}{\rho}$.  Then $\N(e)=\N(e')$.
\end{theorem}
\begin{pf}
Immediate from Theorem~\ref{completeness}, since the group axioms are a
proper subset of the ring axioms.
\end{pf}

\section{Partial completeness of {\rational}: fields}\label{fields}

We now generalize the previous results to an
arbitrary field structure $A$
by extending the type of normal forms and the normalization function.

\begin{definition}\label{defn:NF} The set $F$ of field expressions in
normal form is the set $\{p/q|p,q\in P\}$.
\end{definition}

\begin{definition}\label{defn:plusFF}\label{defn:multFF}\label{defn:divFF}
{\plusFF}, {\multFF} and $\divFF:F\times F\to F$ are defined as follows.
\begin{eqnarray*}
e_1/e_2 \plusFF f_1/f_2 & := &
 \left((e_1\multPP f_2)\plusPP(e_2\multPP f_1)\right)/(e_2\multPP f_2) \\
e_1/e_2 \multFF f_1/f_2 & := & (e_1\multPP f_1)/(e_2\multPP f_2) \\
e_1/e_2 \divFF f_1/f_2 & := & (e_1\multPP f_2)/(e_2\multPP f_1)
\end{eqnarray*}
\end{definition}

\begin{proposition}\label{plusFF}\label{multFF}\label{divFF}
These functions satisfy the following properties, where $\star\in\{+,\cdot,/\}$:
\begin{enumerate}[(i)]
\item {\starFF} is well defined;
\item if $p\intrel t$ and $q\intrel t'$, then $p\starFF q\intrel t\star t'$.
\end{enumerate}
\end{proposition}
\begin{pf}
Direct consequence of the definitions and Propositions~\ref{plusPP}
and~\ref{multPP}.  Parts (iv)--(vi) are Coq checked.
\end{pf}

\begin{definition}\label{defn:NormF}
The normalization function $\NF$ is defined as follows.
\begin{eqnarray*}
\NF : E & \to & P \\
 i & \mapsto & i \\
 v^0_i & \mapsto & \left(v^0_i\times 1+0\right)/1 \\
 e+f & \mapsto & \NF(e)\plusFF\NF(f) \\
 e\times f& \mapsto & \NF(e)\multFF\NF(f) \\
 e/f& \mapsto & \NF(e)\divFF\NF(f) \\
 v^1_i(e) & \mapsto & \left(v^1_i(\NF(e))\times 1+0\right)/1
\end{eqnarray*}
\end{definition}
\begin{proposition}\label{NormF} {\NF} satisfies the following properties:
\begin{enumerate}[(i)]
\item {\NF} is well defined;
\item\label{NFpresint} if $e\intrel t$ then $\NF(e)\intrel t$.
\end{enumerate}
\end{proposition}
\begin{pf}
As before, the first part is an induction similar to
Proposition~\ref{NormR}.
The second property is proved by induction (Coq checked); notice that the
hypothesis $e\intrel t$ is essential to guarantee that {\NF} will not
introduce divisions by zero (compare with the situation for groups).
\end{pf}

\subsection{Correctness and completeness}

Unfortunately, {\rational} as described above does not work so well
with this normalization function as before, as the following 
example shows.

\begin{example} Let $x:A$ be a variable such that $x\neq 0$.
Then $x\times 1/x =_A1$ is a special case of axiom \axiom{F}.

A simple calculation shows that
\begin{eqnarray*}
\mlfn{x\times 1/x}{\emptyset} & = &
  \langle v^0_0\times 1/v^0_0,[v^0_0:=x]\rangle \\
\mlfn{1}{[v^0_0:=x]} & = &
  \langle 1,[v^0_0:=x]\rangle
\end{eqnarray*}
but $\NF(v^0_0\times 1/v^0_0)=(v^0\times 1+0)/(v^0\times 1+0)$, while
$\NF(1)=1$.
\end{example}

The problem lies in the algebraic structure of $F$ with the operations
above defined, and in trying to generalize the properties in
Section~\ref{PNprops}.
Although $\langle F,\plusFF,0/1,\multFF,1/1\rangle$ is a ring, it is not
an integral domain: \emph{any} expression of the form $0/e$ is an additive
unit, and therefore $F$ does not become a field when we add $\divFF$ as
a division operator.

The crux of the matter is that terms in $F$ are not restricted to
irreducible fractions (with the intuitive meaning of what ``irreducible''
should mean).
Adding this restriction is also far from trivial: rewriting quotients of
polynomials to irreducible fractions is known to be an extremely difficult
problem, and implementing such an algorithm in {\NF} would make {\rational}
extremely slow.

Therefore, we will use a different approach.
Going back to the example, it is easy to check that
$\NF(v^0_0\times 1/v^0_0-1)=0/(v^0_0\times 1+0)$.
Therefore, we will use the following modified version of {\rational}
for fields: instead of comparing the normal form of the two expressions
$e$ and $f$, we compute the normal form of $e-f$ and check that it has
the form $0/e'$.
This is correct.

\begin{corollary}\label{correctnessF1}
Let $t,t':A$ and define $\langle e,\rho\rangle=\mlfn{t}{\emptyset}$ and
$\langle e',\sigma\rangle=\mlfn{t'}{\rho}$.  If $\NF(e-e')=0/e''$, where
$e''$ is an arbitrary expression, then $t=_A t'$ can be proved from the
field axioms and unfolding of the definitions of $-$, {\zring} and {\nexp}.
\end{corollary}
\begin{pf}
Let $e$ and $e'$ be as defined above and suppose that $\N(e-e')=0/e''$ for
some $e''$.
By Lemma~\ref{quotecorrect}, both $e\intrel t$ and $e'\intrel t'$.
By Proposition~\ref{NormR} also $\N(e)\intrel t$ and $\N(e')\intrel t'$.
Since $\N(e-e')=\N(e)\plusFF \N(e')\multFF(-1)$, Lemmas~\ref{plusFF}
and~\ref{multFF} together with~(\ref{intrel:int}) imply that
$\N(e-e')\intrel t+t'\times -1$; but $\N(e-e')=0/e''$, hence $0/e''$ can
be interpreted, and therefore $0/e''\intrel 0$ by~(\ref{intrel:div})
and~(\ref{intrel:int}).  By Lemma~\ref{intrelfunction} it then follows
that $t+t'\times -1=_A 0$, from which the thesis follows.
\end{pf}

The only drawback of this approach is that the completeness proof does not
go through.
A proof analogous to that of Theorem~\ref{completeness}, obtained by
replacing ``$\N(e)=\N(e')$'' with ``$\N(e-e')=0/e''$'' everywhere, fails
on the induction step for \axiom{Set_4}, since the induction hypothesis
will not be strong enough to prove an equivalent of Lemma~\ref{set4}.
All other proofs can be adapted, though, thus yielding the following
(partial) completeness result (proved like Theorem~\ref{completeness}).

\begin{theorem}\label{completeness:field}
Let $t,t':A$ be such that the equality $t=_A t'$ can be proved
only from the field axioms and unfolding of the definitions of $-$,
$\zring$ and $\nexp$ in $t$ and $t'$, without using \axiom{Set_4} except
for $-$ and $\cdot^n$ ($n$ closed).  Define
$\langle e,\rho\rangle=\mlfn{t}{\emptyset}$ and
$\langle e',\sigma\rangle=\mlfn{t'}{\rho}$.  Then $\N(e-e')$ has the form
$0/e''$ for some expression $e''$.
\end{theorem}

\begin{example}
To see that the extra hypothesis is really needed, consider the equality
$f(x/2+x/2)=_Af(x)$, which is clearly provable.  Then
\begin{eqnarray*}
\mlfn{f(x/2+x/2)}{\emptyset} & = &
  \langle v^1_0(v^0_0/2+v^0_0/2),\rho\rangle \\
\mlfn{f(x)}{\rho} & = &
  \langle v^1_0(v^0_0),\rho\rangle
\end{eqnarray*}
with $\rho=[v^0_0:=x],[v^1_0:=f]$, but 
$\NF(v^1_0(v^0_0/2+v^0_0/2)-v^1_0(v^0_0))$ is
\[(v^1_0((v^0_0\times 1+0)/1)\times(-1)+v^1_0((v^0_0\times 4+0)/4)\times1+0)/1%
\neq0.\]
\end{example}

In practice, though, the difference in strength between
Theorem~\ref{completeness} and Theorem~\ref{completeness:field} is not
very serious.  In most situations axiom \axiom{Set_4} is not needed,
and in several where it is used {\rational} still works (this will be
the case whenever the hypothesis of this axiom can be proved just from
the ring axioms).  In particular, {\rational} still works on goals
like $f(x+x)/4+f(2x)/4=f(x+x+0)/2$, and this is usually enough.
Throughout C-CoRN, less than five situations were found where this
limitation of {\rational} made an alternative proof necessary.

\section{Remarks on the implementation}\label{extensions}

As mentioned in the Introduction, our theoretical treatment of {\rational}
considers a simplified version of the tactic.
The actual implementation covers all the expressions considered in the
type of setoids in C-CoRN, in particular binary functions and partial unary
functions, the latter being implemented as functions of a setoid element
and a proof term.

These extra cases pose no new difficulties.
There is one new axiom \axiom{Set'_4}, stating a variant of
\axiom{Set_4} for partial functions.
The type of expressions needs to be extended with two new constructors
for variables representing these functions, and the interpretation and
quote must be adapted accordingly.
As a consequence several new cases need to be considered when proving
results about the order in which terms are quoted, but these can be
treated in a similar way to the like cases for variables in $\V_1$
(unlike variables in $\V_0$, which behave differently).
The completeness results for groups and rings still hold; as for
fields, as would be expected, the hypothesis of
Theorem~\ref{completeness:field} has to be strengthened since
\axiom{Set'_4} and \axiom{Set_5} cannot be used either in the proof of
$t=t'$.

\section{Conclusions}\label{concl}

In this paper we formally described {\rational} and undertook a study
of its behavior as a decision procedure.  It was shown to be correct
and complete for groups and rings, which is very useful information in
interactive proof development, and correct and partially complete for
fields, which is also very useful, as explained in Section~\ref{fields}.

Furthermore, we hope to use the completeness of {\rational} to take the
tactic a step further.  By the completeness of the theory of fields we
also know that for every equality $t=t'$ that cannot be proved from the
axioms there is a field where $t\neq t'$ holds.  We hope to use the
information provided by {\rational} upon failure (namely, an expression
in normal form that does not equal zero) to construct such a model
\emph{within} Coq, thus reflecting completeness within the system.

Although {\rational} is designed for Coq, we conducted this study at a
level of abstraction that should make it easy to develop similar
(partially) complete tactics for other proof assistants based on type
theory.  It is also possible that {\rational} can be adapted to other
systems.

%\appendix
%\section{Proof of Lemma~\ref{quotecommutes}}

%We will now present the proof of Lemma~\ref{quotecommutes}: if the order in
%which two terms are quoted is reversed, the resulting expressions will
%differ only in the names of the variables.

%The proof is split in several lemmas.
%We first look at the cases when the expressions are quoted to variables.
%\begin{lemma}\label{quotev0commv1}
%Let $t:A$, $f:[A\to A]$ and $\rho$ be a valuation pair for $A$.
%Suppose $\mlfnv{t}{\rho}=\langle v^0_i,\sigma\rangle$ and
%$\mlfnv{f}{\rho}=\langle v^1_j,\theta\rangle$.
%Then
%$\mlfnv{f}{\sigma}=\langle v^1_j,\langle\sigma_0,\theta_1\rangle\rangle$ and
%$\mlfnv{t}{\theta}=\langle v^0_i,\langle\sigma_0,\theta_1\rangle\rangle$.
%\end{lemma}
%\begin{pf}
%Staightforward, since each operator acts on a different component of $\rho$.
%\end{pf}

%\begin{lemma}\label{quotev0commv0}
%Let $t_1,t_2:A$ and $\rho$ be a valuation pair for $A$.
%Suppose that
%\begin{eqnarray*}
%\mlfnv{t_1}{\rho} & = & \langle v^0_i,\sigma\rangle\\
%\mlfnv{t_2}{\sigma} & = & \langle v^0_j,\theta\rangle\\
%\mlfnv{t_2}{\rho} & = & \langle v^0_{j'},\sigma'\rangle\\
%\mlfnv{t_1,\sigma') & = & \langle v^0_{i'},\theta'\rangle
%\end{eqnarray*}
%Then there is a renaming of variables $\xi$ for $\theta$ such that
%$\xi_1=\idn$, $\domain(\xi_0)\cap\domain(\rho_0)=\emptyset$,
%{\isrenamevar{\theta'}\theta\xi}, $i'=\xi_0(i)$ and $j'=\xi_0(j)$.
%\end{lemma}
%\begin{pf}
%There are four cases to consider, corresponding to whether or not there are
%variables that are already mapped into $t_1$ and/or $t_2$.  The only
%non-trivial case is when for no $k$ either $\rho_0(v^0_k)=t_1$ or
%$\rho_0(v^0_k)=t_2$, and it can easily be seen that the renaming of
%variables $\langle(i\ j),\idn\rangle$ satisfies all the desired conditions.
%\end{pf}

%\begin{lemma}\label{quotev1commv1}
%Let $f_1,f_2:[A\to A]$ and $\rho$ be a valuation pair for $A$.
%Suppose that
%\begin{eqnarray*}
%\mlfnv{f_1}{\rho} & = & \langle v^1_i,\sigma\rangle\\
%\mlfnv{f_2}{\sigma} & = & \langle v^1_j,\theta\rangle\\
%\mlfnv{f_2}{\rho} & = & \langle v^1_{j'},\sigma'\rangle\\
%\mlfnv{f_1,\sigma') & = & \langle v^1_{i'},\theta'\rangle
%\end{eqnarray*}
%Then there is a renaming of variables $\xi$ for $\theta$ such that
%$\xi_0=\idn$, $\domain(\xi_1)\cap\domain(\rho_1)=\emptyset$,
%{\isrenamevar{\theta'}\theta\xi}, $i'=\xi_1(i)$ and $j'=\xi_1(j)$.
%\end{lemma}
%\begin{pf}
%The proof is exactly analogous to that of Lemma~\ref{quotev0commv0},
%interchanging the indices $0$ and $1$ of the variables and valuations.
%\end{pf}

%The next two lemmas concern the commutation of quoting a term to a variable
%and quoting a term to an expression.  Their proofs are by induction and
%very similar to the proof of Lemma~\ref{quotecommutes}, so they are omitted.

%\begin{lemma}\label{quotecommv0}
%Let $t_1,t_2:A$, $e_1,e'_1,e_2,e'_2:E$ and
%$\rho,\sigma,\sigma',\theta,\theta'$ be valuation pairs for $A$
%satisfying the following relations.
%\begin{eqnarray*}
%\mlfnv{t_1}{\rho} & = & \langle e_1,\sigma\rangle\\
%\mlfn{t_2}{\sigma} & = & \langle e_2,\theta\rangle\\
%\mlfn{t_2}{\rho} & = & \langle e'_2,\sigma'\rangle\\
%\mlfnv{t_1}{\sigma'} & = & \langle e'_1,\theta'\rangle
%\end{eqnarray*}
%Then there is a renaming of variables $\xi$ for $\theta$ such that:
%\begin{enumerate}[(i)]
%\item\label{domxi0} $\xi_1=\idn$ and
%$\domain(\xi_0)\cap\domain(\rho_0)=\emptyset$;
%\item\label{replxi0} {\isrenamevar{\theta'}\theta\xi} and
%{\isrenamevar{e'_i}{e_i}\xi} for $i=1,2$.
%\end{enumerate}
%\end{lemma}

%\begin{lemma}\label{quotecommv1}
%Let $t:A$, $f:[A\to A]$, $e,e':E$, $i,i':\nat$ and
%$\rho,\sigma,\sigma',\theta,\theta'$ be valuation pairs for $A$
%satisfying the following relations.
%\begin{eqnarray*}
%\mlfnv{f}{\rho} & = & \langle v^1_i,\sigma\rangle\\
%\mlfn{t}{\sigma} & = & \langle e,\theta\rangle\\
%\mlfn{t}{\rho} & = & \langle e',\sigma'\rangle\\
%\mlfnv{f}{\sigma'} & = & \langle v^1_{i'},\theta'\rangle
%\end{eqnarray*}
%Then there is a renaming of variables $\xi$ for $\theta$ such that:
%\begin{enumerate}[(i)]
%\item\label{domxi1} $\xi_0=\idn$ and
%$\domain(\xi_1)\cap\domain(\rho_1)=\emptyset$;
%\item\label{replxi1} {\isrenamevar{\theta'}\theta\xi},
%{\isrenamevar{e'}e\xi} and $i'=\xi_1(i)$.
%\end{enumerate}
%\end{lemma}

%Now we can prove Lemma~\ref{quotecommutes}.

%\noindent\textsc{Lemma (\ref{quotecommutes})} Let $t_1,t_2:A$, $e_1,e'_1,e_2,e'_2:E$
%and $\rho,\sigma,\sigma',\theta,\theta'$ be valuation pairs for $A$
%satisfying the following relations.
%\begin{eqnarray*}
%\mlfn{t_1}{\rho} & = & \langle e_1,\sigma\rangle\\
%\mlfn{t_2}{\sigma} & = & \langle e_2,\theta\rangle\\
%\mlfn{t_2}{\rho} & = & \langle e'_2,\sigma'\rangle\\
%\mlfn{t_1}{\sigma'} & = & \langle e'_1,\theta'\rangle
%\end{eqnarray*}
%Then there is a renaming of variables $\xi$ for $\theta$ such that:
%\begin{enumerate}[(i)]
%\item\label{domxi} $\domain(\xi_i)\cap\domain(\rho_i)=\emptyset$ for $i=1,2$;
%\item\label{replxi} {\isrenamevar{\theta'}\theta\xi} and
%{\isrenamevar{e'_i}{e_i}\xi} for $i=1,2$.
%\end{enumerate}
%\begin{pf}
%By induction on $t_1$ with respect to $\less$
%\begin{enumerate}
%\item $t_1$ is minimal for $\less$.
%\begin{enumerate}
%\item $t_1=\underline n$: then $e_1=n$, $\sigma=\rho$ whence trivially
%$e'_2=e_2$, $\sigma'=\theta$ and also $e'_1=e_1$ and $\theta'=\theta$;
%thus $\xi=\langle\idn,\idn\rangle$ trivially
%satisfies~(\ref{domxi0}) and~(\ref{replxi0}).
%\item otherwise $\mlfn{t_1}{\rho}=\mlfnv{t_1}{\rho}$ and
%$\mlfn{t_1}{\sigma'}=\mlfnv{t_1}{\sigma'}$.  Lemma~\ref{quotecommv0}
%then allows us to conclude that there is a renaming of variables $\xi$
%for $\theta'$ such that $\xi_1=\idn$,
%$\domain(\xi_0)\cap\domain(\rho_0)=\emptyset$ and
%{\isrenamevar\theta{\theta'}\xi} and {\isrenamevar{e_i}{e'_i}\xi} for $i=1,2$.
%Then $\xi^{-1}$ satisfies our requirements: since $\xi_0$ is a permutation,
%$\domain(\xi^{-1}_0)=\domain(\xi_0)$ and therefore
%$\domain(\xi^{-1}_0)\cap\domain(\rho_0)=\emptyset$; $\domain(\idn)=\emptyset$, hence
%$\domain(\xi^{-1}_0)\cap\domain(\rho_0)=\emptyset$;
%{\isrenamevar{\theta'}\theta{\xi^{-1}}} follows from the second case in the
%proof of Lemma~\ref{renamevarsequiv}; and {\isrenamevar{e'_i}{e_i}{\xi^{-1}}}
%follows from the definition of inverse.
%\end{enumerate}
%\item $t_1=f(t)$ with $f:[A\to A]$.
%\begin{enumerate}
%\item $f$ is $-_A$: then $e_1=-e$ where
%$\langle e,\sigma\rangle=\mlfn{t}{\rho}$; similarly $e'_1=-e'$ where
%$\langle e',\theta'\rangle=\mlfn{t}{\sigma'}$.
%By induction hypothesis there is a renaming of variables $\xi$ for
%$\theta$ such that $\xi_1=\idn$,
%$\domain(\xi_0)\cap\domain(\rho_0)=\emptyset$,
%{\isrenamevar{\theta'}\theta\xi}, {\isrenamevar{e'}e\xi} and
%{\isrenamevar{e'_2}{e_2}\xi}.
%The only thing left to show is that {\isrenamevar{e'_1}{e_1}\xi}; but
%this is trivial, since $e_1=e$ and $e'_1=-e'$.  Therefore $\xi$ is the
%desired renaming of variables.
%\item $f$ is $\cdot^n$ with $n$ is closed: analogous.
%\item otherwise $e_1=v^1_i(e)$ with $\mlfn{t}{\rho}=\langle e,\tau\rangle$
%and $\mlfnv{f}{\tau}=\langle v^1_i,\sigma\rangle$;
%also $e'_1=v^1_j(e')$ with $\mlfn{t}{\sigma'}=\langle
%e',\tau'\rangle$ and $\mlfnv{f}{\tau'}=\langle v^1_j,\theta'\rangle$.

%Take $\mlfn{t_2}{\tau}=\langle e^\ast_2,\tau^\ast\rangle$ and
%$\mlfnv{f}{\tau^\ast}=\langle v^1_{i^\ast},\theta^\ast\rangle$.
%% (see Figure~\ref{fig:quotecommutes}).
%By induction hypothesis there is a renaming of variables $\xi$ for
%$\tau^\ast$ such that $\domain(\xi_i)\cap\domain(\rho_i)=\emptyset$ for $i=1,2$,
%{\isrenamevar{\tau'}{\tau^\ast}\xi}, {\isrenamevar{e'}e\xi} and
%{\isrenamevar{e'_2}{e_2}\xi}.  By Lemma~\ref{quotecommv1} there is a
%renaming of variables $\xi'$ for $\theta$ such that $\xi'_0=\idn$,
%$\domain(\xi'_1)\cap\domain(\tau_1)=\emptyset$,
%{\isrenamevar{\theta^\ast}\theta{\xi'}},
%{\isrenamevar{e^\ast_2}{e_2}{\xi'}} and $i^\ast=\xi'_1(i)$.

%%\begin{figure}[htb]
%%\[\xymatrix{
%% & \rho \ar[dl]_{t} \ar[dr]^{t_2} \\
%% \tau \ar[d]_{f} \ar[dr]^{t_2} & & \sigma' \ar[d]_{t} \\
%% \sigma \ar[d]_{t_2} & \tau^\ast \ar@{.>}[r]^{\xi} \ar[d]_{f}
%% & \tau' \ar[d]_{f} \\
%% \theta \ar@{.>}[r]^{\xi'} & \theta^\ast \ar@{.>}[r]^{\xi} & \theta' 
%%}\]
%%\caption{Case 2.(c) of the proof of Lemma~\ref{quotecommutes}.  An arrow
%%from $\rho$ to $\rho'$ with label $t$ means that $\rho'$ is obtained
%%by either $\mlfn{t}{\rho}$ or $\mlfnv{t}{\rho}$.}
%%\label{fig:quotecommutes}
%%\end{figure}

%We now show that $\xi\circ\xi'$ is the desired renaming of variables.
%\begin{itemize}
%\item Since $\rho\subseteq\tau$, if $v^i_k\in\domain(\rho_i)$, then
%$\xi'_i(k)=k$ and $\xi(k)=k$, whence $\domain(\xi_i)\cap\domain(\rho_i)=\emptyset$.
%\item By definition of $\mlfnv{\cdot}{}$, either $\tau^\ast_1(i^\ast)=f$ or
%$i^\ast$ is the smallest number not in $\domain(\tau^\ast_1)$; by definition
%of renaming of bound variables, in the first case $\tau'(\xi_1(i^\ast))=f$,
%hence $i'=\xi_1(i^\ast)$, while in the second $i'=i^\ast$ and hence again
%$i'=\xi_1(i^\ast)$, since $\domain(\tau^\ast_1)\supseteq\domain(\tau_1)$
%and the latter is disjoint from $\domain(\xi')$.
%By hypothesis $i^\ast=\xi'_1(i)$, so $i'=(\xi\circ\xi')_1(i)$.
%Also {\isrenamevar{e'}{e}\xi}: since
%$\domain(\xi')\cap\domain(\tau)=\emptyset$, no variable occurring in
%$e$ is in the domain of $\xi'$ by Lemma~\ref{quotedom}.  Therefore,
%$e'=\renamevar{e}\xi=\renamevar{(\renamevar{e}{\xi'})}\xi=%
%\renamevar{e}{\xi\circ\xi'}$.
%Hence {\isrenamevar{v^1_{i'}(e')}{v^1_i(e)}{\xi\circ\xi'}}.
%\item By a similar reasoning it is easy to see that
%{\isrenamevar{\theta'}{\theta^\ast}\xi}; since also
%{\isrenamevar{\theta^\ast}\theta{\xi'}} we conclude that
%{\isrenamevar{\theta'}\theta{\xi\circ\xi'}}.
%\item From {\isrenamevar{e'_2}{e_2^\ast}\xi} and
%{\isrenamevar{e_2^\ast}{e_2}{\xi'}} we obtain
%{\isrenamevar{e_2'}{e_2}{\xi\circ\xi'}}.
%\end{itemize}
%Hence $\xi\circ\xi'$ is the desired renaming of variables.
%\end{enumerate}
%\item $t_1=u\star v$ with $\star\in\{+,-,\times,/\}$: then there are
%expressions $e_u$ and $e_v$ and a valuation pair $\tau$ such that
%$\mlfn{u}{\rho}=\langle e_u,\tau\rangle$, $\mlfn{v}{\tau}=\langle
%e_v,\sigma\rangle$ and $e=e_u\star e_v$.  Also, there are expressions
%$e'_u$ and $e'_v$ and a valuation pair $\tau'$ such that
%$\mlfn{u}{\sigma'}=\langle e'_u,\tau'\rangle$, $\mlfn{v,\tau')=\langle
%e'_v,\theta'\rangle$ and $e'=e'_u\star e'_v$.

%Considering the auxiliary expressions
%$\mlfnv{t_2}{\tau}=\langle e^\ast_2,\tau^\ast\rangle$
%and $\mlfn{v}{\tau^\ast}=\langle e^\ast_v,\theta^\ast\rangle$,
%applying the induction hypothesis twice and reasoning as in the previous
%case yields the thesis.
%\end{enumerate}
%\end{pf}

\acknowledgements

The authors wish to thank Henk Barendregt and Herman Geuvers for their
helpful suggestions and support.

This work was partially supported by the European Project IST-2001-33562
MoWGLI and by FCT and FEDER under POCTI, namely through the CLC project FibLog
FEDER POCTI / 2001 / MAT / 37239 and the QuantLog initiative.

\begin{thebibliography}{99}
\bibitem{ACHA90}
S.F. Allen, R.L. Constable, D.J. Howe, and W.~Aitken.
\newblock {The Semantics of Reflected Proof}.
\newblock In {\em Proceedings of the 5th Symposium on Logic in Computer
  Science}, pages 95--197, Philadelphia, Pennsylvania, June 1990. IEEE, IEEE
  Computer Society Press.

\bibitem{bar:bar:ruy:96}
G.~Barthe, M.~Ruys, and H.~Barendregt.
\newblock A two-level approach towards lean proof-checking.
\newblock In S.~Berardi and M.~Coppo, editors, {\em Types for proofs and
  programs (Torino, 1995)}, volume 1158 of {\em LNCS}, pages 16--35. Springer
  Verlag, 1996.

\bibitem{bar:raa:00}
G.~Barthe and F.~van Raamsdonk.
\newblock {Constructor subtyping in the Calculus of Inductive Constructions}.
\newblock In J.~Tiuryn, editor, {\em Proceedings 3rd Int.\ Conf.\ on
  Foundations of Software Science and Computation Structures, FoSSaCS'2000,
  Berlin, Germany, 25 March -- 2 Apr 2000}, volume 1784, pages 17--34, Berlin,
  2000. Springer-Verlag.

\bibitem{ccorn}
{Constructive Coq Repository at Nijmegen}.
\newblock \url{http://c-corn.cs.ru.nl/}.

\bibitem{cha:key:90}
C.C. Chang and H.J.~Keisler.
\newblock {\em Model Theory}.
\newblock North-Holland, 1990.

\bibitem{coqmanual}
The {C}oq Development Team.
\newblock {\em {The {C}oq Proof Assistant Reference Manual}}, April 2004.
\newblock Version $8.0$.

\bibitem{lcf:geu:wie:04}
L.~Cruz-Filipe, H.~Geuvers, and F.~Wiedijk.
\newblock {C-CoRN: the Constructive Coq Repository at Nijmegen}.
\newblock In A.~Asperti, G.~Bancerek, and A.~Trybulec, editors,
  {\em Mathematical
  Knowledge Management, Third International Conference, MKM 2004},
  volume 3119 of {\em LNCS}, pages 88--103. Springer Verlag, 2004.

\bibitem{lcf:wie:04b}
L.~Cruz-Filipe and F.~Wiedijk.
\newblock Equational reasoning in algebraic structures: a complete tactic.
\newblock Technical Report NIII-R0431, NIII, Nijmegen, July 2004.

\bibitem{lcf:wie:04}
L.~Cruz-Filipe and F.~Wiedijk.
\newblock {Hierarchical Reflection}.
\newblock In K.~Slind, A.~Bunker, and G.~Gopalakrishnan, editors, {\em Theorem
  Proving in Higher Order Logics, 17th International Conference, TPHOLs 2004},
  volume 3223 of {\em LNCS}, pages 66--81. Springer Verlag, 2004.

\bibitem{del:00}
D.~Delahaye.
\newblock {A Tactic Language for the System Coq}.
\newblock In M.~Parigot and A.~Voronkov, editors, {\em Proceedings of
  Logic for Programming and Automated Reasoning (LPAR), Reunion Island}, volume
  1955 of {\em LNCS}, pages 85--95. Springer-Verlag, 2000.

\bibitem{del:may:01}
D.~Delahaye and M.~Mayero.
\newblock {Field: une proc{\'e}dure de d{\'e}cision pour les nombres r{\'e}els
  en Coq}.
\newblock {\em Journ{\'e}es Francophones des Langages Applicatifs}, January
  2001.

\bibitem{dyb:set:03}
P.~Dybjer and A.~Setzer.
\newblock {Induction-recursion and initial algebras}.
\newblock {\em Annals of Pure and Applied Logic}, 124:1--47, 2003.

\bibitem{geu:pol:wie:zwa:02}
H.~Geuvers, R.~Pollack, F.~Wiedijk, and J.~Zwanenburg.
\newblock {The Algebraic Hierarchy of the {FTA} {P}roject}.
\newblock {\em Journal of Symbolic Computation, Special Issue on the
  Integration of Automated Reasoning and Computer Algebra Systems}, pages
  271--286, 2002.

\bibitem{geu:wie:zwa:00}
H.~Geuvers, F.~Wiedijk, and J.~Zwanenburg.
\newblock {Equational Reasoning via Partial Reflection}.
\newblock In M.~Aagaard and J.~Harrison, editors, {\em Theorem Proving in
  Higher Order Logics, 13th International Conference, TPHOLs 2000}, volume 1869
  of {\em LNCS}, pages 162--178, Berlin, Heidelberg, New York, 2000. Springer
  Verlag.

\bibitem{har:00}
J.R.~Harrison.
\newblock {\em The HOL Light manual (1.1)}, 2000.

\bibitem{HNC+03}
J.~Hickey, et al.
\newblock {{MetaPRL} --- {A} Modular Logical Environment}.
\newblock In D.~Basin and B.~Wolff, editors, {\em Proceedings of the
  16th International Conference on Theorem Proving in Higher Order Logics
  (TPHOLs 2003)}, volume 2758 of {\em LNCS}, pages 287--303. Springer-Verlag,
  2003.

\bibitem{hof:str:96}
M.~Hoffman and T.~Streicher.
\newblock {The Groupoid Interpretation of Type Theory}.
\newblock In G.~Sambin and J.~Smith, editors, {\em Proceedings of the
  meeting of Twenty-five years of constructive type theory, Venice}. Oxford
  University Press, 1996.

\bibitem{muz:93}
M.~Muzalewski.
\newblock {\em An Outline of PC Mizar}.
\newblock Fondation Philippe le Hodey, Brussels, 1993.

\bibitem{str:93}
T.~Streicher.
\newblock {\em {Semantical Investigations into Intensional Type Theory}}.
\newblock {LMU M\"unchen}, 1993.
\newblock Habilitationsschrift.

\bibitem{YNKH03}
X.~Yu, A.~Nogin, A.~Kopylov, and J.~Hickey.
\newblock {Formalizing Abstract Algebra in Type Theory with Dependent Records}.
\newblock In D.~Basin and B.~Wolff, editors, {\em 16th International
  Conference on Theorem Proving in Higher Order Logics (TPHOLs 2003). Emerging
  Trends Proceedings}, pages 13--27. Universit{\"a}t Freiburg, 2003.
\end{thebibliography}

\end{article}
\end{document}
