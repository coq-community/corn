%&LaTeX
\documentclass[a4paper]{article}
% \usepackage[active]{srcltx} % search in dvi
% \usepackage[swedish]{babel}
% \usepackage[T1]{fontenc} % ger d�liga pdf-er!
\usepackage[latin1]{inputenc}

% \usepackage{icomma} % ``a,b'' typeset without space, but ``a, b'' with

% \usepackage{graphicx}
% \newcommand{\thee}{\rotatebox[origin=c]{180}{$\iota$}}

% \usepackage{amsmath}
% \usepackage{amsthm}
% \usepackage{amssymb}
% \usepackage{latexsym}
% \usepackage{hyperref}
% \usepackage[PS]{diagrams}
\usepackage{jespermacros}

\author{Jesper Carlstr\"om}

%\author{Jesper Carlstr\"om\thanks{\url{http://www.math.su.se/$\sim$jesper/}}}

\title{Subsets in Coq\\Report to Erik Palmgren}

% \parindent0pt%
% \parskip2ex%

\begin{document}
\nonfrenchspacing
\maketitle


\noindent
This is a report about the files {\sf subsets.v} and {\sf
  subsetspacked.v}, which in short can be expressed thus:
\begin{itemize}
\item In {\sf subsets.v}, you find ideas from my article in the
  proceedings of TYPES 2002
  \cite{carlstromsubsets}. However, some things have been made
  differently for technical reasons and this text describes mainly the
  differences.
\item In {\sf subsetspacked.v}, you find ''quotients'', packed in
  records -- and a lot of records for other uses as well. The idea is
  in part absurd because the mere reason for doing the stuff in the
  previous file is the belief that packing and unpacking should be
  avoided. I created this file to investigate how {\sf subsets.v} can
  be used in a packing oriented environment. Some interesting
  aspects were observed. I have tried to point these out later in this
  report.
\end{itemize}

\tableofcontents

\pagebreak
\section{The file {\sf subsets.v}}
\label{sec:filen-subsets.v}

Let me emphasise that I use implicit arguments heavily, it became more
or less necessary to get readable code.
\begin{verbatim}
Set Implicit Arguments.
Unset Strict Implicit.
\end{verbatim}
This means that Coq figures out the typing in many cases, so that one
can write in a polymorphic style. The line `Unset Strict Implicit'
makes Coq more cleaver. To be precise, some syntactic information is
used even if it is not invariant under definitional equality. For
instance, if $P(x)=\bot$, Coq could still (sometimes) derive the type
of $x$ from the expression $P(x)$.

\subsection{Advantage of subsets-as-propositional-functions}
\label{sec:delm-som-past}

The paradigm of subsets-as-propositional-functions 
\cite{toolboxsubsets,carlstromsubsets} is based on the observation
that subsets are defined by saying
\begin{equation}
  \label{eq:1}
  x\insubset U \defeq P(x)
\end{equation}
for a propositional function $P$, whence one can conceive 
''$x\insubset U$'' as just another way of writing $P(x)$. In other
words, there is no reason for distinguishing the subset $U$ from the
propositional function $P$.

This solution, however, makes it necessary to distinguish between sets
and subsets. If this distinction is to be avoided, subsets has to be
viewed as $\Sigma$-sets instead. A drawback of this approach is that
$\Sigma$-sets are more complicated than propositional functions and
requires more reductions. The counterpart to (\ref{eq:1}) becomes to let
$U\defeq\Sigma(A,P)$ and conclude that
\begin{equation}
  \label{eq:2}
  x\in U \iff \pi_r(x)\in P(\pi_\ell(x))
\end{equation}
with both sides being \emph{judgements} rather than
\emph{propositions} and the $\iff$ is \emph{not} definitional
equality. Rather, (\ref{eq:2}) should be thought of as expressing
certain introduction and elimination rules for $\Sigma$.

The solution with $\Sigma$ becomes particularly heavy when many
subsets are considered at one on a given underlying set. This is often
the case in formal topology, where an open set is a subset of the set
of basic opens. This might explain why the
subsets-as-propositional-functions approach is so popular among
topologysts. In the expression $\Sigma(A,U)$, only $U$ is varied,
hence it is easier to consider only $U$ itself.


\subsection{The drawback}
\label{sec:nackd-med-delm}

In a big formalisation project, like the one in Nijmegen, where the
goal is to produce a library for mathematics in type theory, it is not
always self-evident that the logically most natural solution is the
best one. It is also necessary to take into consideration:
\begin{itemize}
\item Communication with users -- the solution should be easy to
  understand. A new solution is bad even if it is better from a
  strictly logical/mathematical point of view.
\item Readability -- working with a computer, you care that Coq can
  present the material in a nice way. If Coq is better at presenting a
  certain form of formalisation, then this is an argument for this form.
\item Modularity -- one way of viewing the rationals is as a subset of
  the set of formal fractions. But if $\Q$ is a subset or a set should
  not be visible outside the module for rationals. This is complicated
  because type difference is always visible. By sticking to sets
  rather than subsets all the time, it is easier to use the modules.
\item Technical details -- however silly, you will certainly take into
  account things that have to do with the current Coq implementation.
\end{itemize}


\subsection{The content of {\sf subsets.v}}
\label{sec:inneh-subsets.v}

I first define binary relations on types
\begin{verbatim}
Definition bin (X:Type) := (X->X->CProp).
\end{verbatim}
and then what it is to be reflexive, symmetric and
transitive. The concept of equivalence relation is avoided, because I
want to try to work without conjuncted concepts.

We then come to the section about subsets. Coq has a mechanism called
''sections''. Within such a section one can make assumptions about
variables, which are then abstracted when the section is closed.

\subsubsection{Section subsets.}
\label{sec:section-subsets}

We thus write:
\begin{verbatim}
Section subsets.

Variable (carrier:Type).

Definition subset (X:Type) : Type := X->CProp.
\end{verbatim}
and get a family ''subset X'' of subsets of $X$. We can tell what it
is to be an element in a subset and introduce infix notation for this:
\begin{verbatim}
Definition insubset (x:carrier) (U:subset carrier) := U x.
Infix "in" := insubset (at level 60).
\end{verbatim}
Then we may write ''x in U'' to say that $x$ is in the subset
$U$. 
Then we introduce some algebraic operations on subsets:
\begin{verbatim}
Definition intersection (U V:subset carrier) x := x in U and x in V.
Definition union (U V:subset carrier) x := x in U or x in V.
Definition complement (U:subset carrier) x := Not (x in U).

Definition intersection_family (I:Type) (U:I->(subset carrier)) x :=
  forall i:I, x in U i.

Definition union_family (I:Type) (U:I->(subset carrier)) x :=
  {i:I | x in U i}.
\end{verbatim}
The notation \verb/{i:I | x in U i}/ is a bit unfortunate here: it is
the standard notation for existence in Coq.

Further, relations between subsets are defined, as inclusion and
extensional equality:
\begin{verbatim}
Definition includedin (U V:subset carrier):= forall x:carrier, 
    x in U -> x in V.

Infix "<=" := includedin (at level 70).

Definition extequal (U V:subset carrier) := (U <= V) and (V <= U).

Infix "=_ext" := extequal (at level 70).
\end{verbatim}
and it is showed that this equality is an equivalence relation.
\begin{verbatim}
Lemma extequal_refl: refl (extequal).
Proof.
intros U.
split; intro p; trivial.
Qed.

Lemma extequal_sym: sym extequal.
Proof.
intros U V [p q].
split; auto.
Qed.

Lemma extequal_trans: trans extequal.
Proof.
intros U V W [p q] [r s].
split; intros a t; auto.
Qed.
\end{verbatim}

Finally some results about binary intersections being equal to
intersections of a binary family. I omit this.

After the section we must tell which notation should be available
after it:
\begin{verbatim}
Infix "in" := insubset (at level 60).
Infix "/\" := intersection.
Infix "\/" := union.
Notation "~ U" := (complement U).
Notation "'for' x 'in' U , P" :=
    (forall (x':_) (x:x' in U), P)
    (at level 200, x ident).
Notation "'exists' x 'in' U , P" :=
    {x':_ |{ x:(x' in U) | P}}
    (at level 200, x ident).
Infix "<=" := includedin.
Infix "=_ext" := extequal (at level 70).
\end{verbatim}


\subsubsection{Section subsetalgebra}
\label{sec:sect-subsetalg}

In this section it is assumed that we deal with four subsets and it
should be proved that operations on them are extensional.
\begin{verbatim}
Section subsetalgebra.

Variable carrier:Type.
Variables U V U' V':subset carrier.
\end{verbatim}

Here are the results. I include them because they show rather good how
one deals with subsets.
\begin{verbatim}
Lemma intersection_ext:
   (U =_ext U') -> (V =_ext V') -> ((U /\ V) =_ext (U' /\ V')).
Proof.
intros [p q] [r s].
split; intros a [u v]; split; auto.
Qed.

Lemma union_ext:
   (U =_ext U') -> (V =_ext V') -> ((U \/ V) =_ext (U' \/ V')).
Proof.
intros [p q] [r s].
split; (intros a [u | v]; [left | right]; auto).
Qed.

Lemma complement_ext: (U =_ext V) -> ((~ U) =_ext (~ V)).
Proof.
intros [p q].
split; intros a r s; auto.
Qed.

Lemma deMorgan: (~ (U \/ V)) =_ext (~ U /\ ~V).
Proof.
split.
intros a p.
split; intro q; apply p; [left|right]; assumption.
intros a [p q] [r|s]; auto.
Qed.
\end{verbatim}


\subsubsection{Section quotients.}
\label{sec:section-quotients}

The purpose of this section is to introduce equivalence relations on
subsets. We therefore assume that we deal with a certain subset, here
called `domain', and define what is meant by a binary relation on
it. Note the use of the previously defined notation
`for x in domain, ...'.

\begin{verbatim}
Variables carrier:Type.
Variable domain:subset carrier.

Definition bin_on := (for x in domain, for y in domain, CProp).

Variable R:bin_on.

Infix "~" := R (at level 70). 
\end{verbatim}
In fact, $R$ takes four arguments $x p y q$, where $p$ and $q$ proves that
$x$ and $y$ belongs to the domain. Because of the implicit arguments
it is not necessary to spell out $x$ and $y$, as Coq can determine them
from $p$ and
$q$. Notice that this requires the option `unstrict' because $x$ is
determined from `p:x in domain', but if `x in domain' is
definitionally equal to `True', the judgement 
`p:x in domain' could be converted to `p:True', which does not contain
information about $x$. 

Because  Coq can extract info about $x$ and $y$ from $p$ and $q$, I
used the following trick:
\begin{quote}
  I use the notation \verb/x/ for $p$ and \verb/y/ for $q$, while $x$
  and $y$ is denoted by \verb/x'/ and \verb/y'/, respectively.
\end{quote}
This leads to the nice illusion that one is working with the
underlying objects, while Coq keeps track of the required proof
objects. In reality it is the opposite, which is unintuitive to
me. Therefore the trick gave more readable code.

We may in this way define reflexivity, symmetry and transitivity thus:
\begin{verbatim}
Definition refl_part := for x in domain, x~x.

Definition sym_part := for x in domain, for y in domain, x~y -> y~x.

Definition trans_part := 
   for x in domain,
   for y in domain,
   for z in domain,   x~y -> y~z -> x~z.
\end{verbatim}

We also define a property which is important to require in order for
equivalence relations on subsets not to be too fine, namely a form of
proof irrelevance. The arguments for this can be found in
\cite{carlstromsubsets}.
\begin{verbatim}
Definition proofirr :=
   for x in domain,
   for y in domain,
   forall z:(underlying x) in domain,
   forall w:(underlying y) in domain,
   z~w -> x~y.
\end{verbatim}
where \verb/underlying/ is defined thus:
\begin{verbatim}
Definition underlying (x':carrier) (_:x' in domain) := x'.
\end{verbatim}

The fines of all reflexive and proof irrelevant relations is called
\verb/same/. It means that the underlying objects are Leibniz equal.
\begin{verbatim}
Definition same:bin_on := fun
   (x':_) (x:x' in domain)
   (y':_) (y:y' in domain) =>
    x'=y'.

Lemma strongrefl: proofirr ->
                  refl_part -> 
                  forall a:carrier,
                  forall (x y:a in domain),
                  x~y.
Proof.
intros Hpirr Hrefl a x y.
apply Hpirr with x x.
apply Hrefl.
Qed.

Lemma same_finest: proofirr ->
                   refl_part ->
                   for x in domain,
                   for y in domain,
                   same x y -> x ~ y.
Proof.
intros Hpirr Hrefl x' x y' y; compute.
destruct 1; apply strongrefl; assumption.
Qed.
\end{verbatim}

Every total equivalence relation can be cut down to a proof irrelevant
equivalence relation on a subset and conversely every such relation
can be extended to a total equivalence relation by the reflexive closure.

\begin{verbatim}
Definition reflexive_closure : bin carrier := 
    fun x' y':carrier => 
   (x'=y') or {x:(x' in domain) | {y:(y' in domain) | x~y}}.

(** reflexive relations on subsets extend to reflexive relations *)

Lemma reflexive_closure_refl: refl reflexive_closure.
Proof.
intros x'.
left; auto.
Qed.

(** symmetric relations on subsets extend to symmetric relations *)

Lemma reflexive_closure_of_sym_sym: sym_part -> sym reflexive_closure.
Proof.
intros s x' y' [eq|H].
 left; symmetry; assumption.
existse H x H.
existse H y H.
right; exists y; exists x; auto.
Qed.


(** Interestingly, we don't have the analogous lemma for transitivity,
since it might happen that we have 

   - [R x' x y' y]
   - [R y' u z' z]

and so the reflexive closure will identify x' with y' and y' with z',
but not necessarily x' with z' (because y need not be the same as
u). We need also proofirr to prove transitivity. *)

Lemma reflexive_closure_of_pirr_trans_trans: 
   proofirr -> trans_part -> trans reflexive_closure.
Proof.
intros Hpirr Htr x' y' z' [id| p].
  destruct id.
  intros [id| p].
    destruct id.
    left; reflexivity.
  existse p x H.
  existse H z e.
  right; exists x; exists z; assumption.
existse p x H.
existse H y e.
intros [id| p].
  destruct id.
  right; exists x; exists y; assumption.
existse p y2 H.
existse H z e2.
right; exists x; exists z.
apply Htr with y' y; auto.
apply Hpirr with y2 z; auto.
Qed.
\end{verbatim}



\subsubsection{Section partialfunctions.}
\label{sec:sect-part}

A nice notation for partial functions:
\begin{verbatim}
Notation "U ~> B" := (for x in U, B).
\end{verbatim}
Let's make some assumptions. We deal with two subsets with relations
on and a partial function between, We define what it means that it
maps into the codomain.
\begin{verbatim}
Variables carrier cocarrier:Type.
Variable domain:subset carrier.
Variable codomain:subset cocarrier.
Variable R:bin_on domain.
Variable R_co:bin_on codomain.
Variable f:domain ~> cocarrier.

Infix "~" := R (at level 70).
Infix "~~" := R_co (at level 70).

Definition into := for x in domain, f x in codomain.
\end{verbatim}

Notice that a partial function in itself does not carry a proof that
it maps into the codomain. This is instead something we can say about
it. This means that we may say things like `The partial function $f$
maps into the subset $U$ but not into the subset $V$'.

The definitions of extensionality look as follows:
\begin{verbatim}
Definition extensional :=
   for x in domain, 
   for x' in domain, 
   forall y:f x in codomain,
   forall y':f x' in codomain,
   x~x' -> y~~y'.
\end{verbatim}
Dually we have the concept of injectivity:
\begin{verbatim}
Definition injective :=
   for x in domain, 
   for x' in domain,
   forall y:f x in codomain,
   forall y':f x' in codomain,
   y ~~ y' -> x ~ x'.
\end{verbatim}

The definition of surjectivity follows Bishop: a function is
surjective if it has a (not necessarily extensional) right inverse
(this example shows that it is sometimes good not to require all
functions to be extensional). That $g$ is right inverse of $f$ is
expressed with \verb/isinv R_co f g/.
\begin{verbatim}
Definition isinv : CProp :=
                 for y in codomain,
                 forall x:(g y in domain),
                 forall y':(f x in codomain), y' ~~ y.
\end{verbatim}
Notice that this concept depends on \verb/R_co/.


\subsubsection{Section bijections.}
\label{sec:section-bijections}

This section contains some results about the right inverse of a
surjective function $f$ is extensional if $f$ is injective, etc. I
skip this, you may have a look in the file.


\subsubsection{Section compositions.}
\label{sec:sect-comp}

Now a drawback of the `flat' approach is apparent: we must assume a
lot in the beginning of this section:
\begin{verbatim}
Variables A B C:Type.
Variable X:subset A.
Variable Y:subset B.
Variable Z:subset C.

Variable RX:bin_on X.
Variable RY:bin_on Y.
Variable RZ:bin_on Z.

Hypothesis Xpirr:proofirr RX.
Hypothesis Ypirr:proofirr RY.
Hypothesis Zpirr:proofirr RZ.

Hypothesis Xrefl:refl_part RX.
Hypothesis Yrefl:refl_part RY.
Hypothesis Zrefl:refl_part RZ.

Hypothesis Xsym:sym_part RX.
Hypothesis Ysym:sym_part RY.
Hypothesis Zsym:sym_part RZ.

Hypothesis Xtr:trans_part RX.
Hypothesis Ytr:trans_part RY.
Hypothesis Ztr:trans_part RZ.

Variable f:X ~> B.
Variable g:Y ~> C.

Hypothesis f_i:into Y f.
\end{verbatim}

The advantage is that we may use the assumptions freely within the
section but when we close it each lemma will depend only on the ones
which are actually used in the proof of the lemma. In this way some
lemmas are very general without any extra effort.

We define what compositions are and introduce nice notation for the
equivalence relations.
\begin{verbatim}
Definition composition (a:A) (x:a in X) := g (f_i x).

Infix "~X" := RX (at level 70).
Infix "~Y" := RY (at level 70).
Infix "~Z" := RZ (at level 70).
\end{verbatim}

Then we show things like `compositions are extensional if the parts
are' (likewise for injectivity and surjectivity).


\subsubsection{Section finiteness.}
\label{sec:section-finiteness}

We treat finite subsets. We want finiteness to be a predicate which we
can use on subsets with equivalence relations. We define a
standard family of finite subsets as initial segments of the natural
numbers and say that a subset is finite iff it is in bijective
correspondence with such a segment.
\begin{verbatim}
Definition fin n : subset nat := fun k => {l:nat | k + S l = n}.
\end{verbatim}

Then some results about finite subsets being discrete etc follows.

Hofmann's theorem (with Hedberg's proof) about equality of natural
numbers having at most one proof is used to show that proofs of k
belonging to fin n are unique. This is in turn used to prove finite
extensional choice: if we have a surjection to a finite subset there
is an extensional right inverse. This is the last result of the file.

\section{The file {\sf subsetspacked.v}}
\label{sec:filen-packed}

The question appeared if it is good or bad with a file which is as
`flat' as {\sf subsets.v}. I therefore produced another file which
uses {\sf subsets.v} but is more `packed'. The first step was of
course to define the notion of equivalence relation
\begin{verbatim}
Record eq_rel (A:Type)(R:bin A):Type :=
{  eq_rel_refl:refl R;
   eq_rel_sym:sym R ;
   eq_rel_trans:trans R }.
\end{verbatim}
and prove things like the extensional equality being an equivalence
relation. Precisely as desired the lemmas from the previous file can
be applied directly:
\begin{verbatim}
Lemma extequal_eq: forall (A:Type), eq_rel (extequal (carrier:=A)).
Proof.
split.
    apply extequal_refl.
  apply extequal_sym.
apply extequal_trans.
Qed.
\end{verbatim}

In the same spirit we define equivalence relations on subsets:
\begin{verbatim}
Record eq_part (A:Type)(U:subset A)(R:bin_on U): CProp :=
{
   eq_part_pirr:   proofirr R ;
   eq_part_refl:   refl_part R;
   eq_part_sym:    sym_part R;
   eq_part_tr:     trans_part R
}.
\end{verbatim}
This definition will later lead to problems, because several of my
lemmas require that a relation is reflexive, but we don't have that,
but only that it is an equivalence relation. For Coq it is not
self-evident that equivalence relations are reflexive, so we have to
teach this to Coq:
\begin{verbatim}
Hint Immediate eq_part_pirr : subsets.
Hint Immediate eq_part_refl : subsets.
Hint Immediate eq_part_sym : subsets.
Hint Immediate eq_part_tr : subsets.
\end{verbatim}
The ending `:subsets' makes the hints put in a data base called
`subsets'. We can then use these hints with a tactic
''sauto'' which is defined in Ltac as follows:
\begin{verbatim}
Ltac sauto := auto with subsets.
\end{verbatim}

We now prove that the reflexive closure of an equivalence relation on
a subset is a total equivalence relation. Again, we may use our lemmas:
\begin{verbatim}
Lemma reflexive_closure_of_eq_eq: forall (A:Type)(U:subset A)(R:bin_on U),
   eq_part R -> eq_rel (reflexive_closure R).
Proof.
intros A U R [Href Hsym Htr Hext].
split.
    apply reflexive_closure_refl.
  apply reflexive_closure_of_sym_sym; assumption.
apply reflexive_closure_of_pirr_trans_trans; assumption.
Qed.

Thereafter you find a proof that ''same'' in an equivalence relation,
I omit this.

The definition of surjectivity follows Bishop:
\begin{verbatim}
Record surjective (A B:Type) (U:subset A) (V:subset B) (RV:bin_on V) (f:U~>B)
: CProp := {
   fun_i: into V f ;
   inv : V ~> A ;
   inv_i: into U inv ;
   isinv_pf : isinv RV f inv }.
\end{verbatim}

Finally, we collect carrier, subset and equivalence relation in a
record. Personally I think this idea is absurd, because it looks to me
the result is very similar to setoids so it is unclear why one should
do anything new at all. It is, however, instructive to see how it works.
\begin{verbatim}
Record quotient : Type :=
{
   q_carrier: Type ;
   q_subset:  subset q_carrier ;
   q_rel:     bin_on q_subset ;
   q_eq:      eq_part q_rel
}.
\end{verbatim}

The rest of this file is more or less a repetition of subsets.v, so I
don't say more about it here but refer to the file itself.



\section{Conclusion}
\label{sec:sammanfattning}

The file {\sf subsets.v} follows the paradigm of
subsets-as-propositional-functions. Advantages are the ones one would
expect. Unexpected (to me) disadvantages is that you often get many
subgoals to prove. However simple, they have to be taken care of (by
the tactic sauto, for instance). For example: instead of the goal that a relation
is an equivalence relation one gets four goals to prove: reflexivity,
symmetry, transitivity, proof irrelevance. 




\bibliographystyle{jesper}
\bibliography{all}

\

\

See also the files {\sf subsets.v} and {\sf subsetspacked.v}.

\end{document}