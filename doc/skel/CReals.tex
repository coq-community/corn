% $Id$

\section{The Reals}
We give a constructive axiomatization of the reals $\RR$. The
intention is that the axioms can be instantiated by any specific
construction of $\RR$. In our axiomatization, the reals form a
constructive ordered field, for which Cauchy-completeness and the
axiom of Archimedes hold.

\begin{definition} A {\em structure for real numbers\/} is a
constructive abelian ordered field 
$\langle \RR, 0, 1, +, *, -, ^{-1}, =, <, \noto\rangle$ that
\begin{enumerate}
%\item $\langle \RR, 0, 1, +, *, -, ^{-1}, =, <, \noto\rangle$ is an abelian
%constructive ordered field,
%\item $<$ is transitive, irreflexive, anti-symmetric,
%\item $+$ respects $<$, i.e.\ $\forall x,y [x<y \rightarrow\forall
%z[x+z<y+z]]$,
%\item $*$ respects $0<$, i.e.\ $\forall x,y [0<x \wedge 0<y
%\rightarrow 0< x y]$,
%\item $\forall x,y[ x \noto y  \leftrightarrow  (x < y \vee y< x)]$,     
\item is Cauchy-complete:
$$\forall x_1, x_2,\ldots 
        (\forall \epsilon>0 \exists N\in\NN \forall m>N(-\epsilon <x_m - x_N <
        \epsilon))
        \rightarrow
        \exists x [x = \lim_{n\rightarrow\infty} x_n].$$
\item satisfies the Axiom of Archimedes:
$$\forall x\exists n\in\NN[x<n].$$
\end{enumerate}
\end{definition}

\begin{remark}[to the Definition]
 In the definition we actually use $\NN$ to denote both $\NN$
  itself (in 1) as the {\em image\/} of
  $\NN$ in $\RR$ (in 2) via the function $f$ that maps $0\in\NN$ to $0\in\RR$
  and $S(x) \in \NN$ to $f(x) +1$. So, in 1, we quantify over the set
  of functions from $\NN$ to $\RR$. In 2, the axiom really reads
  $\forall x\exists n\in\NN[x<f(n)]$, with $f$ the injection of $\NN$
  into $\RR$.
\end{remark}




\weg{\subsection{$\NN$, $\ZZ$ and $\QQ$}

\begin{definition} Similarly to $\NN$ we define
$\ZZ$ and $\QQ$ as follows. 
\begin{enumerate}
\item $\ZZ$ is the smallest subset of $\RR$
containing $0$, $1$ and closed under $+$ and $-$. 
\item $\QQ$ is the
smallest subset of $\RR$ containing $0$, $1$ and closed under $+$,
$-$ and $^{-1}$ (the latter restricted to elements that are $\noto
0$).
\end{enumerate}
\end{definition}

\begin{lemma}
$\ZZ$ and $\QQ$ are closed under $*$.
\end{lemma}

\begin{proof}
By induction, using distributivity and Lemma \ref{lemuninv}.
\end{proof}

\begin{lemma}[Standard forms of naturals, integers and
rationals]\label{lemstand}
\begin{enumerate}
\item Every $x\in\NN$ can be written as
$$(\ldots((0+1)+1)\ldots +1).$$
\item Every $x\in\ZZ$ can be written as
$$ p-q , \mbox{ with } p,q\in \NN.$$
\item Every $x\in\QQ$ can be written as
$$p   q^{-1}, \mbox{ with } p,q\in \ZZ, q \neq 0.$$
\end{enumerate}
\end{lemma}

\begin{proof}
By induction. \qed
\end{proof}

\begin{lemma} Equality on $\NN$, $\ZZ$ and $\QQ$ are decidable, i.e.\
\begin{eqnarray*}
\forall x,y[x\in\NN\wedge y\in \NN &\implies& x =y \vee x\neq y],\\
\forall x,y[x\in\ZZ\wedge y\in \ZZ &\implies& x =y \vee x\neq y],\\
\forall x,y[x\in\QQ\wedge y\in \QQ &\implies& x =y \vee x\neq y],\\
\end{eqnarray*}
\end{lemma}

\begin{proof}
First use Lemma \ref{lemstand} to create a standard form. Then, for $\ZZ$,
$p-q = r-s$ iff $p+s = r+q$ in $\NN$.
For $\QQ$, $p   q^{-1} = r  s^{-1}$ iff $p   s = q   r$ in $\ZZ$. \qed
\end{proof}

\begin{definition}[Absolute value]\label{defabszq}
For $x\in\ZZ$ we define the absolute value of $x$, $|x|$ by first
considering the standardform of $x$, $p-q$, and then defining
\begin{eqnarray*} 
|x|&:=& p-q, \mbox{ if } p\geq q,\\
|x| &:=& q-p, \mbox{ if } p< q.
\end{eqnarray*}
For $x\in\QQ$ we define the absolute value of $x$, $|x|$ by first
considering the standardform of $x$, $p q^{-1}$, and then defining
$$|x|:= |p| |q|^{-1}.$$
\end{definition}

\begin{lemma}\label{lemmulabsval}
For $x, y \in \QQ$,
$$ |x| |y| = |xy|.$$
\end{lemma}

\begin{proof}
We first prove the Lemma for integers. There are four cases of which we treat one:
suppose $x = p-q$ with $p\geq q$ and $y= r-s$ with $r<s$. Then $|x y| = |(p-q) (r-s)|
= |(p-q) r -(p-q) s| = (p-q) s - (p-q) r = = (p-q) (s-r) = |x| |y|$. 
(Use Lemma \ref{lemmulpresgeq}.) Now for the rationals suppose $x = p q^{-1}$ and 
$y = r   s^{-1}$. Then $|x   y| = |p r| (|q s|)^{-1} = |p| |q|^{-1} 
|r|   |s|^{-1} = |x| |y|$.
\end{proof}

\begin{lemma}\label{lemtriangle}
For $x, y \in \QQ$,
$$|x + y| \leq |x|+|y|.$$
\end{lemma}

\begin{proof}
We first prove the Lemma for integers. There are four cases of which we treat one: 
suppose $x = p-q$ with $p\geq q$ and $y= r-s$ with $r<s$. Then $|x+y| =|p-q+r-s| \leq 
|p-q+s-r| = |x| +|y|$.
For the rationals, suppose $x = p q^{-1}$ and $y = r   s^{-1}$. Then $|x+y| = |(ps + rq) (qs)^{-1}|
=  |ps + rq| |qs|^{-1} \leq (|ps| + |rq|) |qs|^{-1} = |p   q^{-1}| + |r s^{-1}| = 
|x|+|y|$. \qed
\end{proof}

\begin{corollary}\label{corabsq}
For $x, y \in \QQ$,
$$||x| - |y|| \leq |x-y|.$$
\end{corollary}

\begin{proof} Distinguish between $|x|\geq |y|$ and $|x|< |y|$. In the
first case, $||x| - |y|| = |x|-|y|\leq |x-y|$, by $|x| \leq
|y|+|x-y|$. In the second case, $||x| - |y|| = |y|-|x|\leq |y-x| =
|x-y|$, by $|y| \leq |x|+|y-x|$. \qed
\end{proof}
}
\subsection{Properties of the real numbers}
\begin{lemma}\label{lemnonstand}
There are no non-standard real numbers, i.e.
$$\forall n\in \NN[ -\frac{1}{n} < x <\frac{1}{n}] \implies x = 0,$$
for all $x$. 
\end{lemma}

\begin{proof}
Suppose $x \noto 0$ towards a contradiction. (Then $\neg
(x\noto 0)$ and hence $x = 0$.) Then $x^{-1}$ exists and by the Axiom
of Archimedes we find a $n\in \NN$ such that $\frac{1}{x} < n$. But
then either $\frac{1}{n} < x$ or $x <\frac{-1}{n}$ (distinguishing
cases according to $x>0$ or $x<0$ and using Lemma
\ref{leminvless}. \qed
\end{proof}

\begin{definition} A sequence of reals $x_1, x_2,\ldots$ is called a
{\em Cauchy sequence\/} if 
$$\forall \epsilon>0 \exists N\in\NN \forall
m>N(-\epsilon < x_m - x_N <  \epsilon).$$ 
\end{definition}

\begin{lemma}\label{lemCauchy} A sequence of reals $x_1, x_2,\ldots$ is
a Cauchy-sequence iff
$$\forall k\in\NN \exists N\in\NN \forall m>N(-\frac{1}{k} <x_m - x_N <
        \frac{1}{k}).$$
\end{lemma}

\begin{proof} The implication from left to right is immediate, as
$\frac{1}{k} >0$. The reverse implication uses the axiom of
Archimedes. Assume $\forall k\in\NN \exists N\in\NN \forall
m>N(-\frac{1}{k} <x_m - x_N <
        \frac{1}{k})$.
Let $\epsilon > 0$. Then $ \epsilon^{-1} \in \RR$, so
there is a $k\in\NN$ such that $\epsilon^{-1} < k$ and hence
$\epsilon > \frac{1}{k}$. Now we find $N$ by our assumption. \qed
\end{proof}

\weg{
\begin{lemma}\label{lemrealseqrat}
For every $x\in \RR$ there exists a sequence of rational numbers $q_0,
q_1, q_2, \ldots$ such that
$$ x = \lim_{n\rightarrow\infty} q_n.$$
\end{lemma}

\begin{proof} Let $x\in \RR$. Then there are $m, n\in \NN$ such that
$x<n$ and $-x <m$, hence $-m < x < n$. We construct two sequences of
rational numbers  $p_0, p_1, p_2, \ldots$ and $q_0, q_1, q_2, \ldots$
such that
\begin{enumerate}
\item $\forall i\in \NN [ p_i < x < q_i]$,      
\item $\forall i\in \NN [ q_{i+1} - p_{i+1} = \frac{2}{3}(q_{i} - p_{i})]$.
\end{enumerate}
Then $q_0, q_1, q_2, \ldots$ is a Cauchy sequence and
$\lim_{n\rightarrow \infty} q_n = x$. The construction of $p_0, p_1,
p_2, \ldots$ and $q_0, q_1, q_2, \ldots$ is as follows. Define $p_0 := -m$,
$q_0 := n$ and if $p_i$ and $q_i$ have already been defined such that
$p_i < x < q_i$, consider  
\begin{eqnarray*}
l &:=& \frac{2p_i + q_i}{3},\\
r &:=& \frac{p_i + 2 q_i}{3}.
\end{eqnarray*}
Now $l < r$, so we can define
\begin{eqnarray*}
q_{i+1} := r, && p_{i+1} := p_i, \mbox{ if }x<r,\\
q_{i+1} := q_i, && p_{i+1} := l, \mbox{ if }l<x.
\end{eqnarray*}
Then $q_{i+1} - p_{i+1} = \frac{2}{3}(q_{i} - p_{i})$
and $p_{i+1} < x < q_{i+1}$. \qed
\end{proof}
}

\begin{lemma}\label{lemapy}
Given $x,y\in\RR$ and $\epsilon>0$, there exists $x'\in\RR$ such that
$$-\epsilon < x-x'< \epsilon \wedge x'\noto y.$$ 
\end{lemma}

\begin{proof}
$y< x+\frac{\epsilon}{2} \vee y> x-\frac{\epsilon}{2}$. In the first
case take $x' := x+\frac{\epsilon}{2}$, in the second case take $x' :=
x-\frac{\epsilon}{2}$.
\qed
\end{proof}

We now define the {\em maximum\/} of two real numbers. This is not
straightfoward, because we have no trichotomy. (Classically, the
maximum can be defined in an ordered field, but constructively that is
in general not the case: one needs the Cauchy property.) In a
situation where the reals are constructed out of the rationals, say,
$x = (x_i)_{i\in \NN}$, one can use the maximum of two rationals
($\max(x_i,y_i)$) to define a Cauchy sequence of
the maximum of $x$ and $y$, namely $(\max(x_i,y_i)_{i\in \NN}$.
Here we can not do that. Instead when defining the maximum of $x$ and
$y$ we first have to define an auxiliary sequence of reals
$(y_i)_{i\in\NN}$ that has $y$ as a limit and such that $x\noto y_i$
for all $i$.

\begin{definition}\label{defmax}
We construct a sequence $(y_i)_{i\in\NN}$ such that
$$\forall i\in \NN[-\frac{1}{i}< y-y_i <\frac{1}{i} \wedge y_i \noto
x].$$ This is is possible, due to Lemma \ref{lemapy}. Note that
$(y_i)_{i\in\NN}$ is a Cauchy sequence and $y=\lim_{i\rightarrow \infty}
y_i$. Now define the sequence $(s_i)_{i\in\NN}$ by
$$s_i :=
\left\{ \begin{array}{rcl}
        x &\mbox{if}& x>y_i,\\
        y_i &\mbox{if}& x<y_i,\\
       \end{array}\right.$$
Now $(s_i)_{i\in\NN}$ is a Cauchy sequence and we define
$$\maxx (x ,y) := \lim_{i\rightarrow \infty} s_i.$$
\end{definition}

\begin{lemma}\label{lemmaxngt}
$\forall x,y\in\RR[\neg(\maxx(x,y)>x \wedge \maxx(x,y)>y)]$.
\end{lemma}
\begin{proof}
From the Definition of \maxx.\qed
\end{proof}

\begin{lemma}\label{lemmaxcomm}
\maxx\ is commutative, i.e.\ $\forall x,y\in\RR[\maxx(x,y)=\maxx(y,x)]$.
\end{lemma}

\begin{lemma}\label{lemmaxisupb}
\maxx\ gives an upperbound, i.e.\ $\forall x,y\in\RR[\maxx(x,y)\geq x 
\wedge \maxx(x,y)\geq y]$.
\end{lemma}

\begin{lemma}\label{lemmaxislub}
\maxx\ give a least upperbound, i.e.\ $\forall x,y,z\in\RR[z\geq x 
\wedge z\geq y \rightarrow z\geq\maxx(x,y)]$. 
\end{lemma}

\begin{proof}
Suppose $z< \maxx(x,y)$. Then  $z<y \vee y<\maxx(x,y)$ and 
$z<x \vee x<\maxx(x,y)$. If $y<\maxx(x,y)$, then $x<\maxx(x,y)$ 
contradicts Lemma \ref{lemmaxngt}, so $z<y\vee z<x$, contradicting 
$z\geq x \wedge z\geq y$. So $z\geq\maxx(x,y)$.
\qed
\end{proof}

\begin{lemma} \label{lemmaxgeq}
$\forall x,y\in\RR[ x\leq y \leftrightarrow \maxx(x,y) = y]$.
\end{lemma}

\begin{proof}
  For $\rightarrow$: $\maxx(x,y)\geq y$ by Lemma \ref{lemmaxisupb} and
  $\maxx(x,y) \leq y$ by Lemma \ref{lemmaxislub}.\\
  For $\leftarrow$: suppose $x>y$. Then $x\geq y$ and hence
  $\maxx(x,y) = x$ by the previous. Hence $x=y$, contradiction. So
  $x\leq y$.  \qed
\end{proof}

\begin{definition}\label{defabs}
For $x\in\RR$, we define
$$|x| := \maxx(x, -x).$$
\end{definition}

\begin{lemma} \label{lemabsid}
$\forall x\in \RR[x\geq 0 \rightarrow |x|=x].$
\end{lemma}

\begin{proof}
If $x\geq 0$, then 
$-x \leq 0$, hence $-x\leq x$. So $\maxx(x,-x)= x$ by Lemma 
\ref{lemmaxgeq}. 
\qed
\end{proof}

\begin{lemma}\label{lemabseps}
$\forall x,y,r\in\RR [|x-y| \leq r \leftrightarrow x-r \leq y\leq x+r]$.
\end{lemma}

\begin{proof}
  Immediate using the intermediate equivalent statement $x-y\leq r
  \wedge -x+y \leq r$.  \qed
\end{proof}

\begin{lemma}\label{lemtriangle}
$\forall x,y\in\RR[ |x+y| \leq |x|+|y|]$.
\end{lemma}

\begin{proof}
  $\maxx(x,-x)+\maxx(y,-y) \geq x+y$ and $\maxx(x,-x)+\maxx(y,-y) \geq
  -x-y$.  Hence $|x|+|y| = \maxx(x,-x)+\maxx(y,-y) \geq \maxx(x+y,
  -x-y) = |x+y|$.  \qed
\end{proof}

\begin{lemma}\label{lemabsmin}
$\forall x,y,z,r,q\in\RR[ |x-y|\leq r \wedge |y-z|\leq q \rightarrow |x-z|\leq r+q$. 
\end{lemma}

\begin{proof}
$|x-z| =|x-y+y-z| \leq |x-y|+|y-z| \leq r+q$.
\qed
\end{proof}

\weg{
\begin{remark}\label{remtris}
The construction of the two sequences $p_0, p_1, p_2, \ldots$ and
$q_0, q_1, q_2, \ldots$ in the proof above is a standard technique. We
will refer to it as the {\em method of succesive trisection}. If we
want to define a Cauchy sequence with limit $x$, this method works in
general if we have a fixed $p_0$ and $q_0$ such that $p_0 < x < q_0$
and we can decide whether $\frac{2p + q}{3} <x$ or $x< \frac{p+3q}{3}$
for every $p,q$ with $p_0 <p <q<q_0$.
\end{remark}


\begin{lemma}
If a sequence of rationals $q_1, q_2, \ldots$ is a Cauchy sequence,
then so is $ |q_1|, |q_2|, \ldots$.
\end{lemma}

\begin{proof} Use Corollary \ref{corabsq}. \qed
\end{proof}

\begin{definition}[Absolute value]\label{defabsr}
For $x\in\RR$ we define the absolute value of $x$, $|x|$ by first
considering a Cauchy sequence $q_0, q_1, \ldots$ of rationals with
limit $x$, and then defining
$$|x|:= \lim_{n\rightarrow\infty}|q_n|.$$
\end{definition}
}

\begin{lemma}\label{lemlimop}
If  $x_0, x_1, \ldots$ and $y_0, y_1, \ldots$ are  Cauchy sequences
with limits (respectively) $x$ and  $y$, then
\begin{eqnarray*}
\lim_{n\rightarrow\infty} x_n +\lim_{n\rightarrow\infty} y_n &=& 
\lim_{n\rightarrow\infty} (x_n + y_n),\\ 
(\lim_{n\rightarrow\infty} x_n)  (\lim_{n\rightarrow\infty} y_n) &=& 
\lim_{n\rightarrow\infty} (x_n   y_n),\\ 
|\lim_{n\rightarrow\infty} x_n| &=& 
\lim_{n\rightarrow\infty} |x_n|,\\ 
-\lim_{n\rightarrow\infty} x_n  &=& 
\lim_{n\rightarrow\infty} (- x_n),\\ 
(\lim_{n\rightarrow\infty} x_n)^{-1}  &=& 
\lim_{n\rightarrow\infty} (x_n)^{-1}, 
\end{eqnarray*}
where the latter is only defined if $\forall i\in\NN[x_i\noto 0]$. 
Furthermore, if $x_i \leq y_i$ 
for all $i$, then $x\leq y$.
\end{lemma}

In the following, for $n\in\NN$, $y^n$ denotes the $n$-times
multiplication of $y$. 

\begin{definition}\label{defrootRR}
For $x\geq 0$ and $n\in\NN^+$, we define $\sqrt[n]{x}$. First we
notice that $(x+1)^n >x$ (proof by induction on $n$). Define the
sequences $(p_i)_{i\in\NN}$ and $(q_i)_{i\in\NN}$ as follows.
\begin{eqnarray*}
p_0 &:=& 0,\\
q_0 &:=& x+1,\\
p_{i+1} &:=& \left\{\begin{array}{rcl}
                p_i &\mbox{if}&(\frac{2p_i + q_i}{3})^n<x,\\
                \frac{2p_i + q_i}{3} &\mbox{if}&(\frac{p_i + 2q_i}{3})^n>x
                \end{array}\right.\\
q_{i+1} &:=& \left\{\begin{array}{rcl}
                q_i &\mbox{if}&(\frac{p_i + 2q_i}{3})^n>x,\\
                \frac{p_i + 2q_i}{3} &\mbox{if}&(\frac{2p_i + q_i}{3})^n<x
                \end{array}\right.
\end{eqnarray*}
Then we have the following.
\begin{enumerate}
\item $\forall i\in \NN [ p_i < x < q_i]$,      
\item $\forall i\in \NN [ q_{i+1} - p_{i+1} = \frac{2}{3}(q_{i} - p_{i})]$.
\end{enumerate}
So, $(q_i)_{i\in\NN}$ is a Cauchy sequence and we define
$$\sqrt[n]{x}:= \lim_{i\rightarrow \infty} q_i.$$ 
\end{definition}

\begin{lemma}\label{lemrtpos}
$\forall x\in\RR\forall n\in\NN^+[x\geq 0 \implies \sqrt[n]{x} \geq 0].$
\end{lemma}

\begin{lemma}\label{lemrtmult}
$\forall x,y\in\RR\forall n\in\NN^+[x,y\geq 0 \implies
\sqrt[n]{x}\sqrt[n]{y} = \sqrt[n]{xy}].$
\end{lemma}

\begin{lemma}\label{lemroot}
$$\forall x \forall n\in\NN^+ [x\geq 0 \implies (\sqrt[n]{x})^n = x].$$
\end{lemma}

\begin{lemma}
 If $x\geq 0$, then there is a unique $y\geq 0$ such that $y^2 = x$.
\end{lemma}

\begin{proof}
  Suppose we have $y$ and $z$ ($y,z\geq 0$) such that $y^2 = x = z^2$
  and suppose $y\noto z$.  Using Lemma ?? we conclude that $y=z \vee
  y= -z$, hence $y= -z$.  Also $y\noto 0 \vee z\noto 0$, so
  $(y>0\wedge z<0)\vee(y<0 \wedge z>0)$. Contradiction. So $y=z$.
  \qed
\end{proof}

\weg{
\begin{lemma}\label{lemdecQroot}
Equality on the set $\{ x\,|\, \exists y\in\QQ\exists k\in\NN[y =
x^k]\}$ is decidable.
\end{lemma}
}


